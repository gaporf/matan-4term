\documentclass{article}

\usepackage[T2A]{fontenc}
\usepackage[utf8]{inputenc}
\usepackage[russian]{babel}
\parindent 0pt
\parskip 8pt
\usepackage{setspace}
\usepackage{etaremune}
\usepackage{amsmath}
\usepackage{amssymb}
\usepackage{amsfonts}
\usepackage[left=2.3cm, right=2.3cm, top=2.7cm, bottom=2.7cm, bindingoffset=0cm]{geometry}
\usepackage{latexsym}
\usepackage[unicode, pdftex]{hyperref}
\usepackage{xcolor}
\usepackage{graphicx}
\usepackage{mathtools}
\graphicspath{ {./images/} }

\doublespacing

\everymath{\displaystyle}

\begin{document}

\newcommand{\R}[0]{\mathbb{R}}
\newcommand{\RM}[0]{\mathbb{R}^m}
\newcommand{\dist}[0]{\mathrm{dist}}
\newcommand{\rang}[0]{\mathrm{rang} $\ $}
\newcommand{\grad}[0]{\mathrm{grad} $\ $}
\newcommand{\Lin}[0]{\mathrm{Lin} $\ $}

\tableofcontents

\newpage 

\part{Интеграл по мере}

\newpage

    \section{Интеграл ступенчатой функции}
    
        $f = \sum\limits_{k = 1}^n \lambda_k \cdot \chi_{E_k}$, $f \geqslant 0$, где $E_k \in \mathcal{A}$~--- допустимое разбиение, тогда интеграл ступенчатой функции $f$ на множестве $X$ есть
                
        $\int\limits_{X} f d \mu = \int\limits_{X} f(x) d \mu(x) = \sum\limits_{k = 1}^n \lambda_k \mu E_k$ 
        
        Дополнительно будем считать, что $0 \cdot \infty = \infty \cdot 0 = 0$.
                
        \subsection{Свойства}
                
            \begin{itemize}
                
                \item Интеграл не зависит от допустимого разбиения:
                    
                    $f = \sum \alpha_j \chi_{F_j} = \sum\limits_{k,\, j} \lambda_k \chi_{E_k \cap F_j}$, тогда $\int F = \sum \lambda_k \mu E_k = \sum\limits_{k} \lambda_k \sum\limits_j \mu (E_k \cap F_j) = \sum \alpha_j \mu F_i = \int F$;
                        
                \item $f \leqslant g$, то $\int\limits_{X} f d \mu \leqslant \int\limits_{X} g d \mu$.
                
            \end{itemize}
            
    \newpage
    
    \section{Интеграл неотрицательной измеримой функции}
    
        $f \geqslant 0$, измерима, тогда интеграл неотрицательной измеримой функции $f$ есть
        
        $\int\limits_{X} f d \mu = \sup\limits_{\substack{\text{$g$ - ступ.} \\ 0 \leqslant g \leqslant f}} \left( \int\limits_{X} g d \mu \right)$.
            
        \subsection{Свойства}
                
            \begin{itemize}
                
                \item Для ступенчатой функции $f$ (при $f \geqslant 0$) это определение даёт тот же интеграл, что и для ступенчатой функции;
                    
                \item $0 \leqslant \int\limits_{X} f \leqslant +\infty$;
                    
                \item $0 \leqslant g \leqslant f$, $g$~--- ступенчатая, $f$~--- измеримая, тогда $\int\limits_{X} g \leqslant \int\limits_{X} f$.
                    
            \end{itemize}
                
    \newpage
    
    \section{Суммируемая функция}
    
        $f$~--- измеримая, $f_+$ и $f_-$~--- срезки, тогда если $\int\limits_{X} f_+$ или $\int\limits_{X} f_-$~--- конечен, тогда интеграл суммируемой функции есть
            
        $\int\limits_{X} f d \mu = \int\limits_{X} f_+ - \int\limits_{X} f_-$. 
                
        Если $\int\limits_{X} f \neq \pm \infty$, то говорят, что $f$~--- \textit{суммируемая}, а также $\int |f|$~--- конечен ($|f| = f_+ + f_-$).
                
        \subsection{Свойство}
                
            Если $f \geqslant 0$~--- измерима, то это определение даёт тот же интеграл, что и интеграл измеримой неотрицательной функции.
            
    \newpage
                
    \section{Интеграл суммируемой функции}
        
        $E \subset X$~--- измеримое множество, $f$~--- измеримо на $X$, тогда интеграл $f$ по множеству $E$ есть
        
        $\int\limits_{E} f d \mu := \int\limits_{X} f \chi_E d \mu$. 
        
        $f$~--- суммируемая на $E$ если $\int\limits_{E} f+-$ и $\int\limits_{E} f_-$~--- конечны одновременно.
            
        \subsection{Свойства} 
                
            \begin{itemize}
                    
                \item $f = \sum \lambda_k \chi_{E_k}$, то $\int\limits_{E} f = \sum \lambda_k \mu \left( E_k \cap E \right)$;
                
                \item $f \geqslant 0$~--- измерима, тогда $\int\limits_{E} f d \mu = \sup\limits_{\substack{\text{$g$ - ступ.} \\ 0 \leqslant g \leqslant f}} \left( \int\limits_{X} g d \mu \right)$.
                        
            \end{itemize}
                    
    \newpage
    
    $(X, \mathcal{A}, \mu)$~--- произвольное пространство с мерой.
    
    $\mathcal{L}^0 (X)$~--- множество измеримых почти везде конечных функций.
        
    \section{Простейшие свойства интеграла Лебега}
    
        \begin{enumerate}
        
            \item \textit{Монотонность}: 
            
                $f \leqslant g \Rightarrow \int\limits_{E} f \leqslant \int\limits_{E} g$.
            
                \subsection{Доказательство}
                
                    \begin{itemize}
                    
                        \item $\sup\limits_{\substack{\text{$\widetilde{f}$ - ступ.} \\ 0 \leqslant \widetilde{f} \leqslant f}} \left( \int\limits_{X} \widetilde{f} d \mu \right) \leqslant \sup\limits_{\substack{\text{$\widetilde{g}$ - ступ.} \\ 0 \leqslant \widetilde{g} \leqslant g}} \left( \int\limits_{X} \widetilde{g} d \mu \right)$;
                        
                        \item $f$ и $g$~--- произвольные, то работаем со срезками, и $f_+ \leqslant g_+$, а $f_- \geqslant g_-$, тогда очевидно и для интегралов.
                        
                    \end{itemize}
            
            \item $\int\limits_{E} 1 \cdot d \mu = \mu E$, $\int\limits_{E} 0 \cdot d \mu = 0$.
            
                \subsection{Доказательство}
                
                    По определению.
            
            \item $\mu E = 0$, $f$~--- измерима, тогда $\int\limits_{E} f = 0$.
            
                \subsection{Доказательство}
                
                    \begin{itemize}
                    
                        \item $f$~--- ступенчатая, то по определению интеграла для ступенчатых функций получаем $0$;
                        
                        \item $f \geqslant 0$~--- измеримая, то по определению интеграла для измеримых неотрицательных функций также получаем $0$;
                        
                        \item $f$~--- любая, то разбиваем на срезки $f_+$ и $f_-$ и снова получаем $0$.
                        
                    \end{itemize}
                    
            \item 
            
                \begin{enumerate}
                
                    \item $\int -f = - \int f$;
                    \item $\forall c > 0 : \int cf = c \int f$.
                
                \end{enumerate}
                
                \subsection{Доказательство}
                
                    \begin{itemize}
                    
                        \item $(-f)_+ = f_-$ и $(-f)_ = f_+$ и $\int -f = f_- - f_+ = - \int f$.
                    
                        \item $f \geqslant 0$~--- очевидно, $\sup\limits_{\substack{\text{$g$ - ступ.} \\ 0 \leqslant g \leqslant cf}} \left( \int g \right) = c \sup\limits_{\substack{\text{$g$ - ступ.} \\ 0 \leqslant g \leqslant f}} \left( \int g \right)$.
                        
                    \end{itemize}
                    
            \item Пусть существует $\int\limits_{E} f d \mu$, тогда $\left| \int\limits_{E} f \right| \leqslant \int\limits_{E} |f|$.
            
                \subsection{Доказательство}
                
                    $- |f| \leqslant f \leqslant |f|$,
                    
                    $- \int\limits_{E} |f| \leqslant \int\limits_{E} f \leqslant \int\limits_{E} |f|$.
                    
            \item $f$~--- измерима на $E$, $\mu E < +\infty$, $\forall x \in E : a \leqslant f(x) \leqslant b$. Тогда 
            
                $a \mu E \leqslant \int\limits_{E} f \leqslant b \mu E$.
                
                \subsection{Доказательство}
                    
                    $\int\limits_{E} a \leqslant \int\limits_{E} f \leqslant \int\limits_{E} b$,
                    
                    $a \mu E \leqslant \int\limits_{E} f \leqslant b \mu E$.
                
        \end{enumerate}
        
    \newpage
    
    \section{Счетная аддитивность интеграла (по множеству)}
    
        \subsection{Лемма}
    
            $A = \bigsqcup A_i$, где $A$, $A_i$~--- измеримы, $g \geqslant 0$~--- ступенчатые. Тогда
        
            $\int\limits_{A} g d \mu = \sum\limits_{i = 1}^{+\infty} \int\limits_{A_i} g d \mu$.
            
            \subsubsection{Доказательство}
        
                $g = \sum \lambda_k \chi_{E_k}$.
            
                $\int\limits_A g d \mu = \sum \lambda_k \mu (A \cap E_k) = \sum\limits_{k} \lambda_k \sum\limits_{i} \mu (A_i \cap E_k) = \sum\limits_i \left( \sum\limits_k \lambda_k \mu (A_i \cap E_k ) \right) = \sum\limits_i \int\limits_{A_i} g d \mu$.
            
        \subsection{Теорема}
    
            $f : C \rightarrow \overline{R}$, $f \geqslant 0$~--- измеримая на $A$, $A$~--- измерима, $A = \bigsqcup A_i$, все $A_i$~--- измеримы. Тогда
        
            $\int\limits_{A} f d \mu = \sum\limits_{i} \int\limits_{A_i} f d \mu$
            
            \subsubsection{Доказательство}
        
                \begin{itemize}
            
                    \item $\leqslant$
                
                        $g$~--- ступенчатая, $0 \leqslant g \leqslant f$, тогда $\int\limits_A g = \sum \int\limits_{A_i} g \leqslant \sum \int\limits_{A_i} f$. Осталось перейти к $\sup$.
                    
                    \item $\geqslant$
                
                        $A = A_1 \sqcup A_2$, $\sum \lambda_k \chi_{E_k} = g_1 \leqslant f \chi_{A_1}$, $g_2 \leqslant f \cdot \chi_{A_2} = \sum \lambda_k \chi_{E_k}$, $g_1 + g_2 \leqslant f \cdot \chi_{A}$
                    
                        $\int\limits_{A_1} g_1 + \int\limits_{A_2} g_2 = \int\limits_{A} g_1 + g_2$.
                    
                        переходим к $\sup$ $g_1$ и $g_2$
                    
                        $\int\limits_{A_1} f + \int\limits_{A_2} f \leqslant \int\limits_{A} f$
                    
                        по индукции разобьём для $A = A_1 \sqcup A_2 \sqcup \ldots \sqcup A_n$, $A = \bigsqcup\limits^{+\infty}_{i = 1} A_i$ и $A = A_1 \sqcup A_2 \sqcup \ldots \sqcup A_n \sqcup B_n$, где $B_n = \bigsqcup\limits_{i \geqslant n + 1} A_i$, тогда
                    
                        $\int\limits_{A} \geqslant \sum\limits^n_{i = 1} \int\limits_{A_i} f + \int\limits_{B} f \geqslant \sum\limits^n_{i = 1} \int\limits_{A_i} f \Rightarrow \int\limits_{A}f \geqslant \sum\limits^{+\infty}_{i = 1} \int\limits_{A_i} f$
                    
                \end{itemize}
            
        \subsection{Следствие}
    
            $f \geqslant 0$~--- измеримая, $\nu : \mathcal{A} \rightarrow \overline{\mathbb{R}}_+$, $\nu E = \int\limits_{E} f d \mu$. Тогда $\nu$~--- мера.
            
        \subsection{Следствие 2}
    
            $A = \bigsqcup\limits_{i = 1}^{+\infty} A_i$, $f$~--- суммируемая на $A$, тогда 
        
            $\int\limits_{A} f = \sum\limits_{i} \int\limits_{A_i} f$.
        
\newpage

\part{Предельный переход под знаком интеграла}

\newpage

    \section{Теорема Леви}
    
        $(X, \mathcal{A}, \mu)$, $f_n$~--- измерима, $\forall n : 0 \leqslant f_n(x) \leqslant f_{n + 1} (x)$ при почти всех $x$.
        
        $f(x) = \lim\limits_{n \rightarrow +\infty} f_n(x)$ при почти всех $x$. Тогда
        
        $\lim\limits_{n \rightarrow +\infty} \int\limits_{X} f_n(x) d \mu = \int\limits_{X} f d \mu$.
        
        \subsubsection{Доказательство}
        
            $f$~--- измерима как предел измеримых функций.
            
            \begin{itemize}
            
                \item $\leqslant$
                
                    $f_n(x) \leqslant f(x)$ почти везде, тогда $\forall n : \int\limits_{X} f_n(x) d \mu \leqslant \int\limits_{X} f d \mu$, откуда следует, что и предел интегралов не превосходит интеграл предела.
                    
                \item $\geqslant$
                
                    Достаточно доказать, что для любой ступенчатой функции $g : 0 \leqslant g \leqslant f$ верно $\lim \int\limits_{X} f_n \geqslant \int\limits_{X} g$.
                    
                    Достаточно доказать, что $\forall c \in (0, 1)$ верно $\lim \int\limits_{X} f_n \geqslant c \int\limits_{X} g$.
                    
                    $E_n := X \left( f_n \geqslant cg \right)$, $E_n \subset E_{n + 1} \subset \ldots$.
                    
                    $\bigcup E_n = X$, т.к. $c < 1$, то $c g(x) < f(x)$, $f_n(x) \rightarrow f(x) \Rightarrow f_n$ попадёт в ''зазор'' $c g(x) < f(x)$.
                    
                    $\int\limits_{X} f_n \geqslant \int\limits_{E_n} f_n \geqslant \int\limits_{E_n} c g = c \int\limits_{E_n} g$,
                    
                    $\lim\limits_{n \rightarrow +\infty} \int\limits_{X} f_n \geqslant \lim\limits_{n \rightarrow +\infty} c \int\limits_{E_n} g = c \int\limits_{X} g$, потому что это непрерывность снизу меры $A \mapsto \int\limits_{A} g$.
                    
            \end{itemize}
    
    \newpage
    
    \subsection{Теорема}
    
        Пусть $f$, $g$~--- измеримы на $E$, $f \geqslant 0$, $g \geqslant 0$. Тогда $\int\limits_{E} f + g = \int\limits_{E} f + \int\limits_{E} g$.
        
        \subsubsection{Доказательство}
        
            Если $f$, $g$~--- ступенчатые, то очевидно.
            
            Разберём общий случай. Существуют ступенчатые функции $f_n : 0 \leqslant f_n \leqslant f_{n + 1} \leqslant \ldots \leqslant f$, и $g_n : 0 \leqslant g_n \leqslant g_{n + 1} \leqslant \ldots \leqslant g$, и $f_n(x) \rightarrow f(x)$ и $g_n(x) \rightarrow g(x)$. Тогда
            
            $\int\limits_{E} f_n + g_n = \int\limits_{E} f_n + \int\limits_{E} g_n$, сделаем предельный переход, значит при $n \rightarrow +\infty$
            
            $\int\limits_{E} f + g = \int\limits_ f + \int\limits_{E} g$
            
        \subsubsection{Следствие}
        
            Пусть $f$, $g$~--- суммируемые на множестве $E$, тогда $f + g$ тоже суммируема и $\int\limits_{E} f + g = \int\limits_{E} f + \int\limits_{E} g$.
            
            \textit{Доказательство}
            
                $(f + g)_{\pm} \leqslant | f + g | \leqslant |f| + |g|$.
                
                $h := f + g$,
                
                $h_+ - h_- = f_+ - f_- + g_+ - g_-$,
                
                $h_+ + f_- + g_- = h_- + f_+ + g_+$,
                
                $\int h_+ + \int f_- + \int g_- = \int h_- + \int f_+ \int g_+$,
                
                $\int h_+ - \int h_- = \int f_+ - \int f_- + \int g_+ - \int g_-$, тогда
                
                $\int h = \int f + \int g$.
                
    \subsection{Определение}
    
        $\mathcal{L} (X)$~--- множество суммируемых функций. Это линейное пространство.
        
        Интеграл: $\mathcal{L}(X) \rightarrow \mathbb{R}$~--- это линейная функция, но красивее говорить линейный функционал.
        
        $f_1, \ldots, f_n \in \mathcal{L}(X)$, $\alpha_1, \ldots, \alpha_n \in \mathbb{R}$, тогда $\alpha_1 f_1 + \ldots + \alpha_n f_n \in \mathcal{L}(x)$.
        
        $\int\limits_{X} f = I(f)$, $\int\limits_X \alpha_1 f_1 + \ldots \alpha_n f_n = \alpha_1 \int\limits_{X} f_1 + \ldots + \alpha_n \int\limits_{X} f_n$
        
        $I(\alpha_1 f_1 + \ldots + \alpha_n f_n) = I(\alpha_1 f_1) + \ldots + I(\alpha_n f_n)$.
        
    \subsection{Теорема об интегрировании положительных рядов}
    
        $u_n \geqslant 0$ почти везде, измеримы на $E$. Тогда
        
        $\int\limits_{E} \left( \sum\limits^{+\infty}_{i = 1} u_n \right) d \mu = \sum\limits^{+\infty}_{i \int= 1} \int\limits_{E} u_n d \mu$.
        
        \subsubsection{Доказательство}
        
            Очевидно по теореме Леви.
            
            $S(x) = \sum\limits^{+\infty}_{n = 1} u_n(x)$ и $p \leqslant S_N \leqslant S_{N + 1} \leqslant \ldots$ и $S_N \rightarrow S(X)$.
            
            $\lim\limits_{n \rightarrow +\infty} \int\limits_{E} S_N = \int\limits_{E} S$
            
            $\lim \sum\limits^n_{k = 1} \int\limits_{E} u_k(x) = \int\limits_{E} S(x) d \mu$.
            
        \subsubsection{Следствие}
        
            $u_n$~--- измеримая функция, $\sum\limits^{+\infty}_{n = 1} \int\limits_{E} | u_n | < +\infty$. Тогда
            
            $\sum u_n$~--- абсолютно сходится почти везде на $E$.
            
            \textit{Доказательство}
            
                $S(x) = \sum\limits^{+\infty}_{n = 1} | u_n(x) |$
                
                $\int\limits_{E} S(x) = \sum\limits^{+\infty}_{n = 1} \int\limits | u_n(x) | < +\infty$, значит $S(x)$ конечна почти всюду.
                
                $S(x) = +\infty$ при $x \in B$, $\mu B > 0$, $S(x) \geqslant n \cdot \chi_{B} \int\limits_{E} S(x) \geqslant n \cdot \mu B$.
                
        
\end{document}
