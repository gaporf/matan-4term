\documentclass{article}

\usepackage[T2A]{fontenc}
\usepackage[utf8]{inputenc}
\usepackage[russian]{babel}
\parindent 0pt
\parskip 8pt
\usepackage{setspace}
\usepackage{etaremune}
\usepackage{amsmath}
\usepackage{amssymb}
\usepackage{amsfonts}
\usepackage[left=2.3cm, right=2.3cm, top=2.7cm, bottom=2.7cm, bindingoffset=0cm]{geometry}
\usepackage{latexsym}
\usepackage[unicode, pdftex]{hyperref}
\usepackage{xcolor}
\usepackage{graphicx}
\usepackage{mathtools}
\graphicspath{ {./images/} }

\doublespacing

\everymath{\displaystyle}

\begin{document}

\newcommand{\R}[0]{\mathbb{R}}
\newcommand{\RM}[0]{\mathbb{R}^m}
\newcommand{\dist}[0]{\mathrm{dist}}
\newcommand{\rang}[0]{\mathrm{rang} $\ $}
\newcommand{\grad}[0]{\mathrm{grad} $\ $}
\newcommand{\Lin}[0]{\mathrm{Lin} $\ $}

\tableofcontents

\newpage 

\part{Интеграл по мере}

\newpage

    \section{Интеграл ступенчатой функции}
    
        $f = \sum\limits_{k = 1}^n \lambda_k \cdot \chi_{E_k}$, $f \geqslant 0$, где $E_k \in \mathcal{A}$~--- допустимое разбиение, тогда интеграл ступенчатой функции $f$ на множестве $X$ есть
                
        $\int\limits_{X} f d \mu = \int\limits_{X} f(x) d \mu(x) = \sum\limits_{k = 1}^n \lambda_k \mu E_k$ 
        
        Дополнительно будем считать, что $0 \cdot \infty = \infty \cdot 0 = 0$.
                
        \subsection{Свойства}
                
            \begin{itemize}
                
                \item Интеграл не зависит от допустимого разбиения:
                    
                    $f = \sum \alpha_j \chi_{F_j} = \sum\limits_{k,\, j} \lambda_k \chi_{E_k \cap F_j}$, тогда $\int F = \sum \lambda_k \mu E_k = \sum\limits_{k} \lambda_k \sum\limits_j \mu (E_k \cap F_j) = \sum \alpha_j \mu F_i = \int F$;
                        
                \item $f \leqslant g$, то $\int\limits_{X} f d \mu \leqslant \int\limits_{X} g d \mu$.
                
            \end{itemize}
            
    \newpage
    
    \section{Интеграл неотрицательной измеримой функции}
    
        $f \geqslant 0$, измерима, тогда интеграл неотрицательной измеримой функции $f$ есть
        
        $\int\limits_{X} f d \mu = \sup\limits_{\substack{\text{$g$ - ступ.} \\ 0 \leqslant g \leqslant f}} \left( \int\limits_{X} g d \mu \right)$.
            
        \subsection{Свойства}
                
            \begin{itemize}
                
                \item Для ступенчатой функции $f$ (при $f \geqslant 0$) это определение даёт тот же интеграл, что и для ступенчатой функции;
                    
                \item $0 \leqslant \int\limits_{X} f \leqslant +\infty$;
                    
                \item $0 \leqslant g \leqslant f$, $g$~--- ступенчатая, $f$~--- измеримая, тогда $\int\limits_{X} g \leqslant \int\limits_{X} f$.
                    
            \end{itemize}
                
    \newpage
    
    \section{Суммируемая функция}
    
        $f$~--- измеримая, $f_+$ и $f_-$~--- срезки, тогда если $\int\limits_{X} f_+$ или $\int\limits_{X} f_-$~--- конечен, тогда интеграл суммируемой функции есть
            
        $\int\limits_{X} f d \mu = \int\limits_{X} f_+ - \int\limits_{X} f_-$. 
                
        Если $\int\limits_{X} f \neq \pm \infty$, то говорят, что $f$~--- \textit{суммируемая}, а также $\int |f|$~--- конечен ($|f| = f_+ + f_-$).
                
        \subsection{Свойство}
                
            Если $f \geqslant 0$~--- измерима, то это определение даёт тот же интеграл, что и интеграл измеримой неотрицательной функции.
            
    \newpage
                
    \section{Интеграл суммируемой функции}
        
        $E \subset X$~--- измеримое множество, $f$~--- измеримо на $X$, тогда интеграл $f$ по множеству $E$ есть
        
        $\int\limits_{E} f d \mu := \int\limits_{X} f \chi_E d \mu$. 
        
        $f$~--- суммируемая на $E$ если $\int\limits_{E} f+-$ и $\int\limits_{E} f_-$~--- конечны одновременно.
            
        \subsection{Свойства} 
                
            \begin{itemize}
                    
                \item $f = \sum \lambda_k \chi_{E_k}$, то $\int\limits_{E} f = \sum \lambda_k \mu \left( E_k \cap E \right)$;
                
                \item $f \geqslant 0$~--- измерима, тогда $\int\limits_{E} f d \mu = \sup\limits_{\substack{\text{$g$ - ступ.} \\ 0 \leqslant g \leqslant f}} \left( \int\limits_{X} g d \mu \right)$.
                        
            \end{itemize}
                    
    \newpage
    
    $(X, \mathcal{A}, \mu)$~--- произвольное пространство с мерой.
    
    $\mathcal{L}^0 (X)$~--- множество измеримых почти везде конечных функций.
        
    \section{Простейшие свойства интеграла Лебега}
    
        \begin{enumerate}
        
            \item \textit{Монотонность}: 
            
                $f \leqslant g \Rightarrow \int\limits_{E} f \leqslant \int\limits_{E} g$.
            
                \subsection{Доказательство}
                
                    \begin{itemize}
                    
                        \item $\sup\limits_{\substack{\text{$\widetilde{f}$ - ступ.} \\ 0 \leqslant \widetilde{f} \leqslant f}} \left( \int\limits_{X} \widetilde{f} d \mu \right) \leqslant \sup\limits_{\substack{\text{$\widetilde{g}$ - ступ.} \\ 0 \leqslant \widetilde{g} \leqslant g}} \left( \int\limits_{X} \widetilde{g} d \mu \right)$;
                        
                        \item $f$ и $g$~--- произвольные, то работаем со срезками, и $f_+ \leqslant g_+$, а $f_- \geqslant g_-$, тогда очевидно и для интегралов.
                        
                    \end{itemize}
            
            \item $\int\limits_{E} 1 \cdot d \mu = \mu E$, $\int\limits_{E} 0 \cdot d \mu = 0$.
            
                \subsection{Доказательство}
                
                    По определению.
            
            \item $\mu E = 0$, $f$~--- измерима, тогда $\int\limits_{E} f = 0$.
            
                \subsection{Доказательство}
                
                    \begin{itemize}
                    
                        \item $f$~--- ступенчатая, то по определению интеграла для ступенчатых функций получаем $0$;
                        
                        \item $f \geqslant 0$~--- измеримая, то по определению интеграла для измеримых неотрицательных функций также получаем $0$;
                        
                        \item $f$~--- любая, то разбиваем на срезки $f_+$ и $f_-$ и снова получаем $0$.
                        
                    \end{itemize}
                    
            \item 
            
                \begin{enumerate}
                
                    \item $\int -f = - \int f$;
                    \item $\forall c > 0 : \int cf = c \int f$.
                
                \end{enumerate}
                
                \subsection{Доказательство}
                
                    \begin{itemize}
                    
                        \item $(-f)_+ = f_-$ и $(-f)_ = f_+$ и $\int -f = f_- - f_+ = - \int f$.
                    
                        \item $f \geqslant 0$~--- очевидно, $\sup\limits_{\substack{\text{$g$ - ступ.} \\ 0 \leqslant g \leqslant cf}} \left( \int g \right) = c \sup\limits_{\substack{\text{$g$ - ступ.} \\ 0 \leqslant g \leqslant f}} \left( \int g \right)$.
                        
                    \end{itemize}
                    
            \item Пусть существует $\int\limits_{E} f d \mu$, тогда $\left| \int\limits_{E} f \right| \leqslant \int\limits_{E} |f|$.
            
                \subsection{Доказательство}
                
                    $- |f| \leqslant f \leqslant |f|$,
                    
                    $- \int\limits_{E} |f| \leqslant \int\limits_{E} f \leqslant \int\limits_{E} |f|$.
                    
            \item $f$~--- измерима на $E$, $\mu E < +\infty$, $\forall x \in E : a \leqslant f(x) \leqslant b$. Тогда 
            
                $a \mu E \leqslant \int\limits_{E} f \leqslant b \mu E$.
                
                \subsection{Доказательство}
                    
                    $\int\limits_{E} a \leqslant \int\limits_{E} f \leqslant \int\limits_{E} b$,
                    
                    $a \mu E \leqslant \int\limits_{E} f \leqslant b \mu E$.
                
        \end{enumerate}
        
    \newpage
    
    \section{Счетная аддитивность интеграла (по множеству)}
    
        \subsection{Лемма}
    
            $A = \bigsqcup A_i$, где $A$, $A_i$~--- измеримы, $g \geqslant 0$~--- ступенчатые. Тогда
        
            $\int\limits_{A} g d \mu = \sum\limits_{i = 1}^{+\infty} \int\limits_{A_i} g d \mu$.
            
            \subsubsection{Доказательство}
        
                $g = \sum \lambda_k \chi_{E_k}$.
            
                $\int\limits_A g d \mu = \sum \lambda_k \mu (A \cap E_k) = \sum\limits_{k} \lambda_k \sum\limits_{i} \mu (A_i \cap E_k) = \sum\limits_i \left( \sum\limits_k \lambda_k \mu (A_i \cap E_k ) \right) = \sum\limits_i \int\limits_{A_i} g d \mu$.
            
        \subsection{Теорема}
    
            $f : C \rightarrow \overline{R}$, $f \geqslant 0$~--- измеримая на $A$, $A$~--- измерима, $A = \bigsqcup A_i$, все $A_i$~--- измеримы. Тогда
        
            $\int\limits_{A} f d \mu = \sum\limits_{i} \int\limits_{A_i} f d \mu$
            
            \subsubsection{Доказательство}
        
                \begin{itemize}
            
                    \item $\leqslant$
                
                        $g$~--- ступенчатая, $0 \leqslant g \leqslant f$, тогда $\int\limits_A g = \sum \int\limits_{A_i} g \leqslant \sum \int\limits_{A_i} f$. Осталось перейти к $\sup$.
                    
                    \item $\geqslant$
                
                        $A = A_1 \sqcup A_2$, $\sum \lambda_k \chi_{E_k} = g_1 \leqslant f \chi_{A_1}$, $g_2 \leqslant f \cdot \chi_{A_2} = \sum \lambda_k \chi_{E_k}$, $g_1 + g_2 \leqslant f \cdot \chi_{A}$
                    
                        $\int\limits_{A_1} g_1 + \int\limits_{A_2} g_2 = \int\limits_{A} g_1 + g_2$.
                    
                        переходим к $\sup$ $g_1$ и $g_2$
                    
                        $\int\limits_{A_1} f + \int\limits_{A_2} f \leqslant \int\limits_{A} f$
                    
                        по индукции разобьём для $A = A_1 \sqcup A_2 \sqcup \ldots \sqcup A_n$, $A = \bigsqcup\limits^{+\infty}_{i = 1} A_i$ и $A = A_1 \sqcup A_2 \sqcup \ldots \sqcup A_n \sqcup B_n$, где $B_n = \bigsqcup\limits_{i \geqslant n + 1} A_i$, тогда
                    
                        $\int\limits_{A} \geqslant \sum\limits^n_{i = 1} \int\limits_{A_i} f + \int\limits_{B} f \geqslant \sum\limits^n_{i = 1} \int\limits_{A_i} f \Rightarrow \int\limits_{A}f \geqslant \sum\limits^{+\infty}_{i = 1} \int\limits_{A_i} f$
                    
                \end{itemize}
            
        \subsection{Следствие}
    
            $f \geqslant 0$~--- измеримая, $\nu : \mathcal{A} \rightarrow \overline{\mathbb{R}}_+$, $\nu E = \int\limits_{E} f d \mu$. Тогда $\nu$~--- мера.
            
        \subsection{Следствие 2}
    
            $A = \bigsqcup\limits_{i = 1}^{+\infty} A_i$, $f$~--- суммируемая на $A$, тогда 
        
            $\int\limits_{A} f = \sum\limits_{i} \int\limits_{A_i} f$.
        
\newpage

\part{Предельный переход под знаком интеграла}

\newpage

    \section{Теорема Леви}
    
        $(X, \mathcal{A}, \mu)$, $f_n$~--- измерима, $\forall n : 0 \leqslant f_n(x) \leqslant f_{n + 1} (x)$ при почти всех $x$.
        
        $f(x) = \lim\limits_{n \rightarrow +\infty} f_n(x)$ при почти всех $x$. Тогда
        
        $\lim\limits_{n \rightarrow +\infty} \int\limits_{X} f_n(x) d \mu = \int\limits_{X} f d \mu$.
        
        \subsection{Доказательство}
        
            $f$~--- измерима как предел измеримых функций.
            
            \begin{itemize}
            
                \item $\leqslant$
                
                    $f_n(x) \leqslant f(x)$ почти везде, тогда $\forall n : \int\limits_{X} f_n(x) d \mu \leqslant \int\limits_{X} f d \mu$, откуда следует, что и предел интегралов не превосходит интеграл предела.
                    
                \item $\geqslant$
                
                    Достаточно доказать, что для любой ступенчатой функции $g : 0 \leqslant g \leqslant f$ верно $\lim \int\limits_{X} f_n \geqslant \int\limits_{X} g$.
                    
                    Достаточно доказать, что $\forall c \in (0, 1)$ верно $\lim \int\limits_{X} f_n \geqslant c \int\limits_{X} g$.
                    
                    $E_n := X \left( f_n \geqslant cg \right)$, $E_n \subset E_{n + 1} \subset \ldots$.
                    
                    $\bigcup E_n = X$, т.к. $c < 1$, то $c g(x) < f(x)$, $f_n(x) \rightarrow f(x) \Rightarrow f_n$ попадёт в ''зазор'' $c g(x) < f(x)$.
                    
                    $\int\limits_{X} f_n \geqslant \int\limits_{E_n} f_n \geqslant \int\limits_{E_n} c g = c \int\limits_{E_n} g$,
                    
                    $\lim\limits_{n \rightarrow +\infty} \int\limits_{X} f_n \geqslant \lim\limits_{n \rightarrow +\infty} c \int\limits_{E_n} g = c \int\limits_{X} g$, потому что это непрерывность снизу меры $A \mapsto \int\limits_{A} g$.
                    
            \end{itemize}
    
    \newpage
    
    \section{Линейность интеграла Лебега}
    
        Пусть $f$, $g$~--- измеримы на $E$, $f \geqslant 0$, $g \geqslant 0$. Тогда $\int\limits_{E} f + g = \int\limits_{E} f + \int\limits_{E} g$.
        
        \subsection{Доказательство}
        
            Если $f$, $g$~--- ступенчатые, то очевидно.
            
            Разберём общий случай. Существуют ступенчатые функции $f_n : 0 \leqslant f_n \leqslant f_{n + 1} \leqslant \ldots \leqslant f$, и $g_n : 0 \leqslant g_n \leqslant g_{n + 1} \leqslant \ldots \leqslant g$, и $f_n(x) \rightarrow f(x)$ и $g_n(x) \rightarrow g(x)$. Тогда
            
            $\int\limits_{E} f_n + g_n = \int\limits_{E} f_n + \int\limits_{E} g_n$, сделаем предельный переход, значит при $n \rightarrow +\infty$
            
            $\int\limits_{E} f + g = \int\limits_{E} f + \int\limits_{E} g$
            
        \subsection{Следствие}
        
            Пусть $f$, $g$~--- суммируемые на множестве $E$, тогда $f + g$ тоже суммируема и $\int\limits_{E} f + g = \int\limits_{E} f + \int\limits_{E} g$.
            
            \subsubsection{Доказательство}
            
                $(f + g)_{\pm} \leqslant | f + g | \leqslant |f| + |g|$.
                
                $h := f + g$,
                
                $h_+ - h_- = f_+ - f_- + g_+ - g_-$,
                
                $h_+ + f_- + g_- = h_- + f_+ + g_+$,
                
                $\int h_+ + \int f_- + \int g_- = \int h_- + \int f_+ \int g_+$,
                
                $\int h_+ - \int h_- = \int f_+ - \int f_- + \int g_+ - \int g_-$, тогда
                
                $\int h = \int f + \int g$.
    
    \newpage
        
    \section{Теорема об интегрировании положительных рядов}
    
        $u_n \geqslant 0$ почти везде, измеримы на $E$. Тогда
        
        $\int\limits_{E} \left( \sum\limits^{+\infty}_{i = 1} u_n \right) d \mu = \sum\limits^{+\infty}_{n = 1} \int\limits_{E} u_n d \mu$.
        
        \subsection{Доказательство}
        
            Очевидно по теореме Леви.
            
            $S(x) = \sum\limits^{+\infty}_{n = 1} u_n(x)$ и $p \leqslant S_N \leqslant S_{N + 1} \leqslant \ldots$ и $S_N \rightarrow S(X)$.
            
            $\lim\limits_{n \rightarrow +\infty} \int\limits_{E} S_N = \int\limits_{E} S$,
            
            $\lim\limits_{n \rightarrow +\infty} \sum\limits^n_{k = 1} \int\limits_{E} u_k(x) = \int\limits_{E} S(x) d \mu$.
            
        \subsection{Следствие}
        
            $u_n$~--- измеримая функция, $\sum\limits^{+\infty}_{n = 1} \int\limits_{E} | u_n | < +\infty$. Тогда
            
            $\sum u_n$~--- абсолютно сходится почти везде на $E$.
            
            \subsubsection{Доказательство}
            
                $S(x) = \sum\limits^{+\infty}_{n = 1} | u_n(x) |$
                
                $\int\limits_{E} S(x) = \sum\limits^{+\infty}_{n = 1} \int\limits | u_n(x) | < +\infty$, значит $S(x)$ конечна почти всюду.
    
    \newpage
    
    \section{Абсолютная непрерывность интеграла}
    
        $f$~--- суммируемая функция, тогда верно:
        
        $$\forall \varepsilon > 0 : \exists \delta > 0 : \forall E \in \mathcal{A} : \mu E < \delta : \left| \int\limits_{E} f \right| < \varepsilon$$.
        
        \subsection{Доказательство}
        
            $X_n = X \left( f \geqslant n \right)$, $X_n \supset X_{n + 1} \supset \ldots$ и $\mu \left( \bigcap\limits^{+\infty}_{n = 1} X_n \right) = 0$.
            
            Тогда $\forall \varepsilon > 0 : \exists n_{\varepsilon} : \int\limits_{X_{n_{\varepsilon}}} |f| < \frac{\varepsilon}{2}$ ($A \mapsto \int\limits_{A} |f|$~--- мера, тогда $\int\limits_{\bigcap X_n} |f| = 0$ и по непрерывности меры сверху).
            
            $\delta := \frac{\varepsilon}{2 n_{\varepsilon}}$, берём $E : \mu E < \delta$.
            
            $\left| \int\limits_{E} f \right| \leqslant \int\limits_{E} |f| = \int\limits_{E \cap X_{n_{\varepsilon}}} |f| + \int\limits_{E \setminus X_{n_{\varepsilon}}} |f| \leqslant \int\limits_{X_{n_{\varepsilon}}} |f| + n_{\varepsilon} \mu E < \frac{\varepsilon}{2} + n_{\varepsilon} \frac{\varepsilon}{2 n_{\varepsilon}} = \varepsilon$.
            
        \subsection{Следствие}
        
            $e_n$~--- измеримое множество, $\mu e_n \rightarrow 0$, $f$~--- суммируемая. Тогда $\int\limits_{e_n} f \rightarrow 0$.
 
    \newpage
    
    \section{02.03.2020}
    
        $f_n \rightrightarrows f$ по мере то же самое, что и $\mu X( |f_n - f| \geqslant \varepsilon) \rightarrow 0$. Ещё есть способ $\int\limits_X |f_n - f| d \mu \rightarrow 0$. Можно ли вывести хоть какую-нибудь импликацию.
        
        $\Rightarrow$ нельзя, пример: $f_n(x) = \dfrac{1}{nx}$ в $(\mathbb{R}, \lambda)$, тогда $f_n \rightrightarrows 0$ по мере. а $\int \left| \dfrac{1}{nx} \right| d \mu = +\infty$.
        
        $\Leftarrow$ можно: $\mu X ( |f_n - f| \geqslant \varepsilon) = \int\limits_{x_n} 1 d \mu \leqslant \int\limits_{x_n} \dfrac{|f_n - f|}{\varepsilon} d \mu \leqslant \dfrac{1}{\varepsilon} \int\limits_X |f_n - f| \rightarrow 0$.
        
        Хотим доказать подобие $f_n \rightarrow f$, то $\int f_n \rightarrow \int f$.
        
        \subsection{Теорема Лебега о мажорированной сходимости}
        
            $f_n$, $f$~--- измеримые, почти везде конечные функции. $f_n \xRightarrow[\mu]{} f$. Также существует $g$, что:
            
            \begin{enumerate}
            
                \item $\forall n : |f_n| \leqslant g$ почти везде;
                
                \item $g$~--- суммируема на $X$ ($g$~--- мажоранта).
                
            \end{enumerate}
        
            Тогда $\int\limits_X |f_n - f| d \mu \rightarrow 0$, и тем более $\int\limits_X f_n \rightarrow \int\limits_X f$.
            
            \subsubsection{Доказательство}
            
                $f_n$~--- суммируема в силу первого утверждения про $g$, $f$~--- суммируема по следствию теоремы Рисса. Тем более $\left| \int\limits_X f_n - \int\limits_X f \right| \leqslant \left| \int\limits_X f_n - f \right| \leqslant \int |f_n  - f|$.
                
                \begin{enumerate}
                
                    \item $\mu X < +\infty$. Фиксируем $\varepsilon > 0$. $X_n := X(|f_n - f| \geqslant \varepsilon)$, $\mu X_n \rightarrow 0$.
                    
                        $\int\limits_X |f_n - f| = \int\limits_{x_n} + \int\limits_{x_n^c} \leqslant \int\limits_{x_n} 2g + \int\limits_{x_n^c} \varepsilon_0 \leqslant \int\limits_{x_n} 2g + \int\limits_x \varepsilon < \varepsilon (1 + \mu X)$. (при больших $n$ выражение $\int\limits_{x_n} 2g \leqslant \varepsilon$).
                        
                    \item $\mu X = +\infty$, $\varepsilon > 0$. 
                    
                        Утверждение: $\exists A$~--- измеримое, $\mu A$~--- конечное, $\int\limits_{X \setminus A} g < \varepsilon$.
                    
                        \textit{Доказательство}
                        
                            $\int G = \sup \left\{ \int g_n : h - \text{ступенчатая функция} 0 \leqslant h \leqslant g \right\}$
                            
                        $\exists h_0 : \int\limits_X g - \int\limits_X h_0 < \varepsilon$, $A := \mathrm{supp \ } h_0$. (где supp~--- носитель (support))
                        
                        $\int\limits_{X \setminus A} g + \int\limits_A g - h_0 < \varepsilon$.
                        
                        $\int\limits_X |f_n - f| = \int\limits_A + \int\limits_{X \setminus A} \leqslant \int\limits_A |f_n - f| + 2 \varepsilon < 3 \varepsilon$ при больших $n$.
                        
                \end{enumerate}
                
        \subsection{Теорема Лебега о мажорированной сходимости почти везде}
        
            $(X, \mathcal{A}, \mu)$, $f_n$, $f$~--- измеримые, $f_n \rightarrow f$~--- почти везде.
            
            Существует такая $g$, что:
            
            \begin{enumerate}
            
                \item $|f_n| \leqslant g$ почти везде;
                
                \item $g$~--- суммируема.
                
            \end{enumerate}
            
            \subsubsection{Доказательство}
            
                $f_n$, $f$~--- суммируемая, тем более~--- как и раньше.
                
                $h_n := \sup( |f_n - f|, |f_{n + 1} - f|, \ldots )$, $h_n$ убывает. $0 \leqslant h_n \leqslant 2g$.
                
                $\lim\limits_{n \rightarrow +\infty} h_n(x) = \overline{\lim} |f_n - f| = 0$ почти везде.
                
                $2g - h \geqslant 0$, возрастают, тогда по теореме Леви $\int\limits_X 2g - h \rightarrow \int\limits_X 2g$, значит $\int\limits_X h_n \rightarrow 0$, тогда $\int\limits_X |f_n - f| \leqslant \int\limits_X h_n \rightarrow 0$.
                
        \subsection{Теорема Фату}
        
            $(X, \mathcal{A}, \mu$, $f_n \geqslant 0$~--- измеримые, $f_n \rightarrow f$ почти везде. Если $\exists C > 0$, что $\forall n : \int\limits_X f_n \leqslant C$, то $\int\limits_X f \leqslant C$.
            
            \subsubsection{Замечание}
            
                Вообще говоря $\int\limits_X f_n \not\rightarrow \int\limits_X f$.
                
            \subsubsection{Доказательство}
            
                $g_n = \int (f_n, f_{n + 1}, \ldots)$.
                
                $g_n$ возрастает, $g_n \rightarrow f$ почти везде. $\lim g_n = \underline{\lim} f_n = f$ почти везде.
                
                $\int\limits_X g_n \leqslant \int\limits_X f_n \leqslant C$, тогда $\int F \leqslant C$.
                
            Примерчик
            
            $f_n = n \cdot \chi_{[0, \frac{1}{n}} \rightarrow 0$ почти везде.
            
            $\int\limits_{\mathbb{R}} f_n = 1$, $\int f = 0$.
            
            Положительность важна:
            
            $f_n \geqslant 0$, тогда $\int -f_n \leqslant -1$, но $\int f = 0 \geqslant -1$.
            
            \subsubsection{Следствие}
            
                $f_n \xRightarrow[\mu]{} f$ ($f_{n_k} \rightarrow f$).
                
            \subsubsection{Следствие 2}
            
                $f_n \geqslant 0$, измеримая. Тогда
                
                $\int\limits_X \underline{\lim} f_n \leqslant \underline{\lim} \int\limits_X f_n$.
                
                \text{Доказательство}
                
                    $\int\limits_X g_n \leqslant \int\limits_X f_n \leqslant C$.
                
                    Берём $n_k$
                    
                    $\underline{\lim} \left( \int\limits_X f_n \right) = \lim\limits_{k \rightarrow +\infty} \left( \int\limits_X f_{n_k} \right)$.
                    
                    $\int\limits_X f_{n_k} \rightarrow \lim \left( \int\limits_X f_n \right)$, а $\int\limits_x g_n \rightarrow \int\limits_X \underline{\lim} f_n$.
\newpage

\part{Произведение мер}

\newpage

    \section{Произведение мер}
    
        $(X, \mathcal{A}, \mu)$ и $(Y, \mathcal{B}, \nu)$~--- пространства с мерой.
   
        $\mathcal{A} \times \mathcal{B} = \left\{ A \times B, A \in \mathcal{A}, B \in \mathcal{B} \right\}$~--- семейство подмножеств в $X \times Y$.
        
        $\mathcal{A}$, $\mathcal{B}$~--- полукольца, значит и $\mathcal{A} \times \mathcal{B}$~--- полукольцо.
                
        $\mathcal{A} \times \mathcal{B}$~--- полукольцо \textit{измеримых прямоугольников} (на самом деле это не всегда так).
            
            
        Тогда введём меру на $A \times B$~--- $\mu_0 (A \times B) = \mu(A) \cdot \nu(B)$.
        
        Обозначим $(X \times Y, A \otimes B, \mu \times \nu)$ как произведение пространств с мерой.
        
    \newpage
    
    \section{Теорема о произведении мер}
        
        \begin{enumerate}
        
            \item $\mu_0$~--- мера на полукольце $\mathcal{A} \times \mathcal{B}$;
            
            \item $\mu$, $\nu$~--- $\sigma$-конечное, значит $\mu_0$~--- $\sigma$-конечное.
            
        \end{enumerate}
        
        \subsection{Доказательство}
        
            \begin{enumerate}
            
                \item 

                    Проверим счётную аддитивность $\mu_0$. $\chi_{A \times B} (x, y) = \chi_A(x) \cdot \chi_B(y)$, $(x, y) \in X \times Y$.
            
                    $P = \bigsqcup\limits_{\text{сч.}} P_k$~--- измеримые прямоугольники. $P = A \times B$ и $P_k = A_k \times B_k$, $\chi_P = \sum \chi_{P_k}$.
            
                    $\chi_A(x) \chi_B(y) = \sum\limits_k \chi_{A_k}(x) \chi_{B_k}(y)$. Интегрируем по $\nu$ (по пространству $Y$).
            
                    $\chi_A(x) \cdot \nu(B) = \sum \chi_{A_k}(x) \nu (B_k)$. Интегрируем по $\mu$.
            
                    $\mu A \cdot \nu B = \sum \mu A_k \cdot \nu B_k$.
            
                \item
                
                    $X = \bigcup X_k$, $Y = \bigcup Y_j$, где $\mu X_k$ и $\nu Y_j$~--- конечные, $X \times Y = \bigcup\limits_{k, j} X_k \times Y_j$.
    
                    $\left( \mathbb{R}^m, \mathcal{M}^m, \lambda_m \right)$ и $\left( \mathbb{R}^n, \mathcal{M}^n, \lambda_n \right)$.
    
                    $\left( X \times Y, \mathcal{A} \otimes \mathcal{B}, \mu_0 \right)$, где $\mathcal{A} \times \mathcal{B}$~--- полукольцо.
    
                    Запускаем теорему о продолжении меры.
    
                    $\rightsquigarrow \left( X \times Y, \mathcal{A} \otimes \mathcal{B}, \mu \right)$, где $\mathcal{A} \times \mathcal{B}$~--- $\sigma$-алгебра.
    
                    $\mu$, $\nu$~--- $\sigma$-конечная, следовательно продолжение определено однозначно.
            
            \end{enumerate}
            
        \subsection{Замечание}
    
            Произведение мер ассоциативно.
    
        \subsection{Дополнительная теорема (без доказательства)}
    
            $\lambda_{m + n}$ есть произведение мер $\lambda_m$ и $\lambda_n$.
        
    \newpage
    
    \section{Сечения множества}
    
        $X$, $Y$ и $C \subset X \times Y$, $C_x = \left\{ y \in Y : (x, y) \in C \right\} \subset Y$~--- сечение множества $C$, аналогично определим $C^y = \left\{ x \in X : (x, y) \in C \right\}$.
    
        Допустимы объедения, пересечения и т.п.
    
    \newpage

    \section{Принцип Кавальери}
    
        $(X, \mathcal{A}, \mu)$ и $(Y, \mathcal{B}, \nu)$, а также $\mu$, $\nu$~--- $\sigma$-конечные и полные. 
        
        $m = \mu \times \nu$, $C \in \mathcal{A} \otimes \mathcal{B}$. Тогда:
        
        \begin{enumerate}
        
            \item при почти всех $x \in X$ сечение $C_x \in \mathcal{B}$;
            
            \item $x \mapsto \nu (C_x)$~--- измерима (почти везде) на $X$;
            
            \item $m C = \int\limits_{X} \nu (C_x) d \mu(x)$.
            
        \end{enumerate}
        
        \subsection{Замечание}
        
            \begin{enumerate}
            
                \item $C$~--- измеримая $\not\Rightarrow$ что $\forall x : C_x$~--- измеримое.
                
                \item $\forall x$, $\forall y$, $C_x$, $C^y$~--- измеримы $\not\Rightarrow$ что $C$~--- измеримо (пример можно взять из Серпинскиго).
                
            \end{enumerate}
            
        \subsection{Доказательство}
        
            $D$~--- класс множеств $X \times Y$, для который принцип Кавальери верен.
            
            \begin{enumerate}
            
                \item $D \times \mathcal{B} \subset D$, $C = A \times B$, $C_x = 
                                                                        \begin{cases}
                                                                            B & x \in A \\ 
                                                                            \varnothing & x \notin A
                                                                        \end{cases}$.
                
                    $x \longmapsto C_x: \nu B \cdot \chi_A(x)$.
                    
                    $\int\limits_{X} \nu B \chi_A(x) d \mu(x) = \mu A \cdot \nu B = m C$.
                    
                \item $E_i$~--- дизъюнктные, $E_i \in D$. Тогда $\bigsqcup E_i \in D$.
                
                    $(E_i)_x$~--- измеримые при почти всех $x$. 
                    
                    При почти всех $x$ все сечения $(E_i)_x$, $i = 1, 2, \ldots$~--- измеримые.
                    
                    $E_x = \bigsqcup (E_i)_x$~--- измеримые при почти всех $x$.
                    
                    $\nu E_x = \sum \nu (E_i)_x$, значит $x \mapsto \nu E_x$ измеримая функция.
                    
                    $\int\limits_{X} \nu E_x d \mu = \int\limits_{X} \sum \nu (E_i)_x d \mu = \sum \int\limits_{X} \nu (E_i)_x d \mu = \sum m E_i = m E$
                
                \item $E_i \in D$, $\ldots \supset E_i \supset E_{i + 1} \supset \ldots$, $E = \bigcap\limits^{+\infty}_{i = 1} E_i$, $m E_i < +\infty$. Тогда $E \in D$.
                
                    $\int\limits_{X} \nu (E_i)_x d\mu = m E_i < +\infty \Rightarrow \nu (E_i)_x$~--- почти везде конечны.
                    
                    $(E_i)_x \supset (E_{i + 1})_x \supset \ldots$, $E_x = \bigcap\limits^{+\infty}_{i = 1} (E_i)_x \Rightarrow E_x$~--- измеримое при почти всех $x$.
                    
                    При почти всех $x$ (для тех $x$, для который $\nu (E_i)_x$~--- конечные сразу все $i$ или при $i = 1$), поэтому можно утверждать, что $\nu E_x = \lim\limits_{i \rightarrow +\infty} \nu (E_i)_x \Rightarrow x\mapsto \nu E_X$~--- измерима.
                    
                    $\int\limits_{X} \nu E_x d \mu = \int\limits_{X} \lim (\nu E_i)_x = \lim\limits_{i \rightarrow +\infty} \int\limits_{X} \nu (E_i)_x d \mu = \lim m E_i = m E$ (по непрерывности сверху меры $m$).
                    
                    Перестановка пределов доказывается из теоремы Лебега, которую ещё не доказывали $|\nu (E_i)_x | \leqslant \nu (E_1)_x$~--- суммируемая функция.
                    
                    Мы доказали, что если $A_{ij} \in \mathcal{A} \times \mathcal{B}$, то $\bigcap\limits_{j} \left( \bigcup\limits_{i} A_{ij} \right) \in D$.
        
                    $m E = \inf \left( \sum m P_k, \ E \subset \bigcup P_k \right)$.
    
                \item $m E = 0 \Rightarrow E \in D$. $H = \bigcap\limits_{j} \bigcup\limits_{i} P_{ij}$, $m H = 0$ ($P_{ij} \in \mathcal{A} \times \mathcal{B}$), тогда $E \subset H$ ($H \in D$).
                
                    $0 = m H = \int\limits_{X} \nu H_x d \mu \Rightarrow \nu H_x = 0$ при почти всех $x$, но $E_x \subset H_x \Rightarrow$ при почти всех $x$ $\nu E_x = 0$, значит и $\int \nu E_x = 0 = m E$.
                    
                \item $C \in \mathcal{A} \otimes \mathcal{B}$, $m C < +\infty \Rightarrow C \in D$.
                
                    Для множества $C$ существует множество $e$, что $m e = 0$ и $H = \bigcap \bigcup P_{ij}$ и $C = H \setminus e$, $C_x = H_x \setminus e_x$ и $m C = m H$.
                    
                    $\nu e_x = 0$ при почти всех $x$, значит $\nu C_x = \nu H_x - \nu e_x$ при почти всех $x$.
                    
                    $\int\limits_{X} \nu C_x d \mu = \int\limits_{X} \nu H_x - \nu e_x = \int\limits_{X} \nu H_x - \int\limits_{X} \nu e_x = mH = m C$.
                    
                \item $C$~--- произвольное, $m$-измеримое множество, $X = \bigsqcup X_k$ и $Y = \bigsqcup Y_j$, тогда $C = \bigsqcup\limits_{i, j} \left( C \bigcap \left( X_i \times Y_j \right) \right) \in D$ по пункту $2$. ($\mu X_k$, $\mu Y_j$~--- конечные).
                
            \end{enumerate}
        
        \subsection{Следствие}
        
            $C \in Q \otimes B$, $P_1(C) := \left\{ x : C_x \neq \varnothing \right\}$, тогда если $P_1(C)$~--- измеримое в $X$, тогда $m C = \int\limits_{P_1(C)} \nu C_x d \mu x$.
        
        \subsection{Замечание}
        
            Из того, что $C$ измеримое $\not\Rightarrow$ что его проекция измерима.

    \newpage
    
    \section{Совпадение определенного интеграла и интеграла Лебега}
        
        $f : [a, b] \rightarrow \mathbb{R}$, непрерывное. Тогда $\int\limits^b_a f(x) dx = \int\limits_{[a, b]} f d \lambda_1$.
            
        \subsection{Доказательство}
            
            Достаточно доказать для $f \geqslant 0$. 
                
            $f$~--- непрерывно $\Rightarrow C = \Pi \Gamma \left(f, [a, b] \right)$ измеримо в $\mathbb{R}^2$ (почти очевидно).
                
            $C_x = [0, f(x)]$ (или $\varnothing$) $\Rightarrow$ измеримость $\lambda_1 C_x = f(x)$.
                
            $\int\limits^b_a f(x) dx = \lambda_2 \left( \Pi \Gamma \left( f, [a, b] \right) \right) = \int\limits_{[a, b]} f(x) d \lambda_1 (x)$.
                
        \subsection{Замечание}
            
            $f \geqslant 0$ измеримое, значит $\lambda_2 \Pi \Gamma (f, [a, b]) = \int\limits_{[a, b]} f(x) d \lambda_2(x)$.
            
            $f : X \times Y \rightarrow \overline{\mathbb{R}}$, $C \in X \times Y$, $C_x$, $f_x : C_x \rightarrow \mathbb{R}$, т.е. $y \mapsto f(x, y)$, аналогично $f^y : C^y \rightarrow \overline{\mathbb{R}}$.
    
    \newpage
    
    \section{Теорема Тонелли}
    
        $(X, \mathcal{A}, \mu)$, $(Y, \mathcal{B}, \nu)$ и $\mu$, $\nu$~--- $\sigma$-конечные и полные, а также $m = \mu \times \nu$.
        
        $f : X \times Y \rightarrow \overline{\mathbb{R}}$, $f \geqslant 0$, измеримая. Тогда
        
        \begin{enumerate}
        
            \item при почти всех $x$ функция $f_x$~--- измерима почти везде на $Y$ (аналогично при почти всех $y$ функция $f^y$ также измерима на $X$);
            
            \item $x \mapsto \varphi(x) = \int\limits_Y f_x(y) d \nu (y) = \int\limits_Y f(x, y) d \nu (y)$~--- измерима почти везде на $X$ (аналогично $y \mapsto \psi(y) = \int\limits_X f(x, y) d \mu(x)$~--- измерима почти везде на $Y$);
            
            \item $\int\limits_{X \times Y} f(x, y) d \mu = \int\limits_{Y} \left( \int\limits_{X} f (x, y) d \mu (x) \right) d \nu (y) = \int\limits_X \left( \int\limits_Y f(x, y) d \nu (y) \right) d \mu (x)$.
            
        \end{enumerate}
        
        \subsection{Доказательство}
        
            \begin{enumerate}
            
                \item $f = \chi_c$, $C \subset X \times Y$, измеримая. $f_x = \chi_{C_x} (y)$. $C_x$~--- измеримое при почти всех $x \Rightarrow f_x$~--- измеримая при почти всех $x$.
                
                    $\varphi(x) = \int\limits_{Y} \chi_{C_x} (y) d \nu (y) = \nu (C_x)$ ($x \mapsto \nu C_x$~--- измерима по принципу Кавальери).
                    
                    $\int\limits_{X} \varphi(x) = \int\limits_{X} \nu C_X = m C = \int\limits_{X \times Y} \chi_C d m$.
                
                \item $f = \sum\limits_{\text{кон.}} a_k \chi_{C_k}$, $f \geqslant 0$.
                
                    $f_x = \sum a_k \chi_{(C_k)_x} (y)$.
                    
                    $x \mapsto \int f_x(y) d \nu(y) = \sum a_k \nu (C_k)_x$~--- измеримая (отдельные слагаемые~--- измеримые, значит и вся сумма измеримая).
                    
                    $\int\limits_X \left( \int\limits_{Y} f_x(y) d \nu \right) d \mu = \sum a_k \int\limits_X \nu (C_k)_x d \mu = \sum a_k m C_k = \int\limits_{X \times Y} f d m$
                    
                \item $f \geqslant 0$, $g_n$~--- ступенчатые, что $\ldots \leqslant g_n \leqslant g_{n + 1} \leqslant \ldots$, $\lim\limits_{n \rightarrow +\infty} g_n = f$.
                
                    $f_x = \lim\limits_{n \rightarrow +\infty} (g_n)_x$~--- измерима как предел измеримых функций.
                    
                    $\varphi(x) = \int\limits_Y f_x(y) d \nu (y) = \lim\limits_{n \rightarrow +\infty} \int\limits_{Y} g_n d \nu = \lim\limits_{n \rightarrow +\infty} \varphi_n(x)$, значит $\varphi(x)$ измерима из-за измеримости $\varphi_n$ (Теорема Леви).
                
                    $g_n \leqslant g_{n + 1} \leqslant \ldots \Rightarrow \varphi_n(x) \leqslant \varphi_{n + 1} (x) \leqslant \ldots$
                    
                    $\int\limits_{X} \varphi(x) = \lim\limits_{n \rightarrow +\infty} \int\limits_{X} \varphi_n(x) = \lim\limits_{n \rightarrow} \int\limits_{X \times Y} g_n d m = \int\limits_{X \times Y} f dm$ (по теореме Леви)
                    
            \end{enumerate}
            
        \textit{Везде должна быть приговорка \glqqпри почти всех $x$\grqq }.
        
    \newpage
    
    \section{Теорема Фубини}
    
        $(X, \mathcal{A}, \mu)$, $(Y, \mathcal{B}, \nu)$ и $\mu$, $\nu$~--- $\sigma$-конечные и полные.
        
        $f : X \times Y \rightarrow \overline{\mathbb{R}}$, суммируемая. Тогда
        
        \begin{enumerate}
        
            \item при почти всех $x$ функция $f_x$~--- суммируемая почти везде на $Y$ (аналогично при почти всех $y$ функция $f^y$ также измерима на $X$).
            
            \item $x \mapsto \varphi(x) = \int\limits_Y f_x(y) d \nu (y) = \int\limits_Y f(x, y) d \nu (y)$~--- суммируемая почти везде на $X$ (аналогично $y \mapsto \psi(y) = \int\limits_X f(x, y) d \mu(x)$~--- суммируемая почти везде на $Y$).
            
            \item $\int\limits_{X \times Y} f(x, y) d \mu = \int\limits_{Y} \left( \int\limits_{X} f (x, y) d \mu (x) \right) d \nu (y) = \int\limits_X \left( \int\limits_Y f(x, y) d \nu (y) \right) d \mu (x)$
            
        \end{enumerate}
        
        \textit{без доказательства} 
    
        \subsubsection{Следствие}
        
            $\int\limits_C f = \int\limits_{X \times Y} f \chi_C = \int\limits_X \left( \int\limits_Y f \cdot \chi_C \right) d \mu = \int\limits_{P_1(C)} \left( \int\limits_{C_x} f(x, y) d \nu(y) \right) d \mu (x)$.
            
            $P_1(C)$~--- проекция, измеримая, $\left\{ x : C_x \neq \varnothing \right\}$.
    
    \newpage
    
    \section{Какая-то нужная штука для лекции 02.03.2020, потом удалю}
    
        $B(0, 1) \subset \mathbb{R}^m$, Хотим найти $\lambda_m B(0, 1) = \alpha_m$.
    
        $\lambda_m B(0, R) = \alpha_m \cdot R^M$.
    
        $x_1^2 + x_2^2 + \ldots + x_m^2 \leqslant 1$.
    
        интеграл обычного кружочка: $\int \chi_B d \lambda_2 = \int\limits^1_{-1} \int\limits^{\sqrt{1 - x^2}}_{-\sqrt{1 - x^2}} 1 dy dy dx = \int\limits^1_{-1} 2 \sqrt{1 - x^2} dx = \pi$
    
        $\alpha_m = \int\limits_{\mathbb{R}^m} \chi_B = \int\limits^1_{-1} \left( \int\limits_{B(0, \sqrt{1 - x_1^2}) \subset \mathbb{R}^{m -1}} 1 d \nu \right) dx_1 = \int\limits^1_{-1} (1 - x_1^2)^{\frac{m - 1}{2}} \alpha_{m - 1} d x_1$.
    
        $B(x, y) = \int\limits^1_0 t^{x - 1} (1 - t)^{y - 1} dt$.
    
        $B(x, y) = \dfrac{\Gamma(x) \Gamma(y)}{\Gamma(x + y)}$, $\Gamma(n) = (n - 1)!$, $\Gamma(x + 1) = \Gamma(x) \cdot x$.
    
        Тогда объём шара в $\mathbb{R}^m$ равен $\alpha_{m - 1} 2 \int\limits^1_0 (1 - t)^{\frac{m - 1}{2}} t^{-\frac{1}{2}} dt = B(\frac{1}{2}, \frac{m + 1}{2}) \alpha_{m - 1}$. Тогда объём шара можно переписать как $\dfrac{\Gamma (\frac{1}{2}) \Gamma (\frac{m + 1}{2})}{\Gamma (\frac{m}{2} + 1) \alpha_{m - 1}}$.
    
\newpage

\part{Замена переменных в интеграле}

\newpage

    $(X, \mathcal{A}, \mu)$ и $(Y, \mathcal{B}, )$ (пространство и алгебру изобрели, а меру нет)
    
    $\Phi : X \rightarrow Y$, $\forall B \in \mathcal{B}$, тогда $\Phi^{-1}(B)$~--- измеримое ($\in \mathcal{A}$)
    
    Утверждение: $\Phi^{-1}(\mathcal{B})$~--- $\sigma$-алгебра (упражнение).
    
    Определение: $\nu : \mathcal{B} \rightarrow \overline{\mathbb{R}}$, $E \in \mathcal{B}$, $\nu E := \mu(\Phi^{-1}(E))$~--- это мера на $\mathcal{B}$~--- образ $\mu$ при отображении $\Phi$.
    
    И кстати: $\nu E = \int\limits_{\Phi^{-1}(E)} 1 d \mu$.
    
    $\nu( \bigsqcup B_i) = \mu \left( \Phi^{-1}(\bigsqcup B_i) \right) = \mu \left( \bigsqcup \Phi^{-1}(B_i) \right) = \sum \mu \Phi^{-1}(B_i) \sum \nu B_i$.
    
    $f$~--- измерима относительна $\mathcal{B}$. Тогда $f \circ \Phi$~--- измерима относительна $\mathcal{A}$. $X \left( f \left( \Phi(x) \right) < a \right) = \Phi^{-1} \left( Y (f < a) \right)$
    
    Третье замечание: $\omega : X \rightarrow \overline{\mathbb{R}}$, $\omega \geqslant 0$, измеримая. $\nu(B) := \int\limits_{\Phi^{-1}(B)} \omega d \mu$~--- мера, которая назначает взвешенный образ меры $\omega$~--- вес.
    
    \subsection{Интегрирование по взвешенному образу меры}
    
        $\Phi : X \rightarrow Y$~--- измеримое отображение, $\omega : X \rightarrow \overline{\mathbb{R}}$, $\omega \geqslant 0$~---измеримая на $X$. $\omega$~--- взвешенный образ меры $\nu$. Тогда
        
        $\forall f \geqslant 0$~--- измеримой на $Y$. $f \circ \Phi$~--- измерима на $X$ и выполняется следующее свойство:
        
            $\int\limits_Y f(y) d \nu(y) = \int\limits_x f (\Phi(x)) \omega(x) d \mu(x)$.
            
        То же верно для случая $f$~--- суммируемая.
    
        \subsubsection{Доказательство}
        
            \begin{enumerate}
            
                \item $f = \chi_B$, $B \in \mathcal{B}$. Тогда $f \circ \Phi(x) = \begin{cases} 1 & \Phi(X) \in B \\ 0 & \Phi(x) \notin B \end{cases} = \chi_{\Phi^{-1}(B)}$.
                
                    Доказывать нечего (и весёлый смайлик) $\nu B = \int\limits_{\Phi(B)} \omega d \mu$;
                    
                \item $f$~-- ступенчатая, для каждой ступеньки~--- правда, и по линейности интеграла получаем результат;
                
                \item $f \geqslant 0$~--- измеримая. Теорема об аппроксимизации измеримых функций ступенчатыми плюс предельный переход по теореме Леви;
                
                \item $f$~--- измеримая, значит $|f|$~--- всё верно.
                
            \end{enumerate}
            
        \subsubsection{Следствие}
        
            $f$~--- суммируема на $Y$, $B \in \mathcal{B}$, $\int\limits_B f d \nu(y) = \int\limits_{\Phi^{-1}}(B) f \circ \Phi w d \mu$.
            
            Частный случай: $X = Y$, $\mathcal{A} = \mathcal{B}$, $\Phi = \mathrm{id}$, $\omega \geqslant 0$~--- измерима.
            
            $\nu B = \int\limits_B \omega(x) d \mu(x)$, говорят, что $\omega$~--- плотность меры $\nu$ относительно меры $\mu$.
            
            $\int\limits_X f(x) d \nu(x) = \int\limits_X f(x) \omega(x) d \mu(x)$.
            
    \subsection{Критерий плотности}
    
        $(X, \mathcal{A}, \mu)$, $\nu$~--- ещё одна мера на $\mathcal{A}$, $\omega \geqslant 0$~--- измеримая. Тогда
        
        $\omega$~--- плотность $\nu$ относительна $\mu$ $\Leftrightarrow \forall A \in \mathcal{A}$ верное $\inf\limits_{A} \omega \cdot \mu A \leqslant \nu A \leqslant \sup\limits_{A} \omega \cdot \mu A$ ($0 \cdot \infty = 0$).
        
        \subsubsection{Доказательство}
        
            \begin{itemize}
            
                \item $\Rightarrow$ Очевидно (интеграл $\mu A$ обладает этими свойствами из-за плотностей);
                
                \item $\Leftarrow$ Считаем, что $\omega > 0$. Для $\omega = 0$ получаем: $e := X(\omega = 0)$, $\nu e = 0 = \int\limits_e \omega d \mu$, тогда $\nu (A) = \int\limits_{A} \omega d \mu$, $x \setminus e$ $\omega > 0$. Теперь пусть $\omega > 0$, то $q \in (0, 1)$. $A_j := A(q^j \leqslant \omega \leqslant q^{j - 1})$, $j \in \mathbb{Z}$, $A = \bigsqcup\limits_{j \in \mathbb{Z}} A_j$.
                
                    $q^j \mu A_j \leqslant \nu A_j \leqslant q^{j - 1} \mu A_j$.
                    
                    $q^j \mu A_j \leqslant \int\limits_{A_j} \omega d \mu \leqslant q^{j - 1} \mu A_j$.
                    
                    $q \int\limits_A \omega d \mu = q \sum \int\limits_{A_j} \leqslant \sum q^j \mu A_j \leqslant \nu A \leqslant \dfrac{1}{q} \sum q^j \mu A_j \leqslant \dfrac{1}{q} \int\limits_A \omega$.
                    
                    Устремим $q \rightarrow 1$ и получим доказательство равенства.
                    
            \end{itemize}
            
    \subsection{Лемма}
    
        $f$, $g$~--- суммируемые на $X$, $\forall A$~--- измеримых $\int\limits_A f = \int\limits_a g$. Тогда $f = g$ почти везде.
        
        \subsubsection{Доказательство}
        
            $h = f - g$, $\forall A$~--- измеримых, $\int\limits_A h = 0$.
            
            $A_+ = X(h \geqslant 0)$, $A_- = X(h < 0)$, $A_+ \bigcap A_- = \varnothing$.
            
            $\int\limits_{A_+} |h| = \int\limits_{A_+} h = 0$.
            
            $\int\limits_{A_-} |h| = - \int\limits_{A_-} h = 0$.
            
            $X = A_+ \bigsqcup A_-$, $\int\limits_{X} |h| = 0$.
            
    $L$ и $A$, тогда $\lambda (L(A)) = | \det L | \lambda A$.
    
    \subsection{Лемма об образе малых кубических ячеек}
    
        $\Phi : O \subset \mathbb{R}^m \rightarrow \mathbb{R}^m$, $a \in O$. $\Phi$~--- дифференцируема $G$ в окрестности точки $a$, $\det \Phi'(a) \neq 0$. Пусть $c > | \det \Phi'(a) |$. Тогда
        
        существует такое $\delta > 0$, что для любого куба $Q \subset B(a, \delta)$, $a \in Q$ верно, что $c \lambda \Phi(Q)$.
        
        \subsubsection{Доказательство}
        
            $L := \Phi'(a)$~--- обратимое линейное отображение линейное отображение. $\Phi(x) = \Phi(a) + L (x - a) + o(x - a)$.
            
            $a + L^{-1}(\Phi(x) - \Phi(a)) = x + o(x - a)$ (увеличили в константу, поэтому о маленькое остаётся о маленьким).
            
            $\forall \varepsilon > 0$ можно записать шар $B_{\varepsilon}(a)$, что при $x \in B_{\varepsilon}(a)$ $\left| \psi(x) - x \right| < \dfrac{\varepsilon}{\sqrt{m}} |x - a|$.
            
            $Q \subset B_{\varepsilon}$, $a \in Q$~--- куб со стороной $h$, при $x \in Q : | \psi(x) - x| < \varepsilon h$. $|x_i - a_i| \leqslant h$.
            
            $x$, $y \in Q$, тогда $\left| \psi(x)_i - \psi(y)_i \right| = \left| \psi(x)_i - x_i \right| + \left| \psi(y)_i - y_i \right| + |x_i - y_i| \leqslant | \psi(x) - x | + |\psi(y) - y| + h < (1 + 2 \varepsilon)h$.
            
            $\psi(Q)$~--- содержится в кубе со стороной $(1 + 2 \varepsilon) h$, тогда $\lambda \psi(Q) \leqslant (1 + 2 \varepsilon)^m \lambda Q$.
            
            $\lambda \Phi(Q) \leqslant (1 + 2 \varepsilon)^m | \det L | \lambda Q$.
            
            $< C \lambda Q$.
            
            Берём $\varepsilon : (1 + 2 \varepsilon) | \det L | < C$, где $\delta$~--- радиус $B_{\varepsilon} (a)$.
            
    $\lambda A = \inf\limits_{\text{G - открытое}, A \subset G} \lambda G$
            
    \subsection{Лемма 2}
    
        $f : \underset{\text{откр.}}{O} \subset \mathbb{R}^m \rightarrow \mathbb{R}$, $O$~--- непрерывное. $A$~--- измеримое, $A \subset Q \subset \overline{Q} \subset O$.
        
        Тогда $\int\limits_{A \subset G \text{открытое}} \left( \lambda (G) \sup\limits_{G} f \right) = \lambda A \sup\limits_{A} f$.
        
    \subsection{Теорема 2}
    
        $\Phi : O \subset \mathbb{R}^m \rightarrow \mathbb{R}^m$~--- диффеоморфизм. $A \in \mathcal{M}^m$, $A \subset O$. Тогда
        
        $\lambda \Phi(A) = \int\limits_{A} \left| \det \Phi'(a) \right| d \lambda$.
        
        \subsubsection{Доказательство}
        
            $\nu A := \lambda \Phi(A)$. Верно ли, что $J_{\Phi} (x) := \left| \det \Phi'(x) \right|$~--- это плотность $\nu$ по отношению к $\mu$.
            
            Достаточно проверить, что $\forall A$ верно: $\inf\limits_{A} J_{\Phi} \cdot \lambda A \leqslant \nu A \leqslant \sup\limits_{A} J_{\Phi} \cdot \lambda A$.
            
            Достаточно проверить правое неравенство. Левое~--- правое для $\Phi^{-1}$ и $\widetilde{A} = \Phi(A)$.
            
            $\lambda \Phi^{-1} \left( \widetilde{A} \right) \leqslant \sum J_{\Phi^{-1}} \cdot \lambda \widetilde{A}$.
            
            $\lambda A \leqslant \sup \left| \det (\Phi^{-1})' \right| \lambda \Phi(A)$.
            
            $\sup \dfrac{1}{\left| \det \Phi' \right|}$
            
            $\dfrac{1}{\inf \left| \det \Phi' \right|}$
            
            \begin{enumerate}
            
                \item $A$~--- кубическая ячейка, $\overline{A} \subset O$. От противного: пусть оказалось, что $\lambda Q \sup J_{\Phi} < \nu Q$. Возьмём $c > \sup\limits_{Q} J_{\Phi}$, так, что $\lambda Q \cdot c < \nu Q$. Значит существует такая часть $Q_i$, что $\lambda Q_i \cdot c < \nu Q_i$. $\lambda Q_n \cdot c < nu Q_n$, $a = \bigcap \overline{Q_n}$, накроем точку $a$ этим кубиков. $c > \left| \det \Phi'(a) \right|$, тогда $\nu Q_n = \lambda \Phi(Q_n)$. Получили, что $\lambda \Phi(Q_n) > c \lambda Q_n$, а по лемме нужно наоборот.
            
                \item Оценка $\nu A \leqslant \sup J_{\Phi} \lambda A$, верна для случая, когда $A$~--- открытое множество.
                
                    $\nu Q \leqslant \sup\limits_{A} J_{\Phi} \lambda Q$.
                    
                    Суммируя по $Q$: $\nu A \leqslant \sup\limits_{A} J_{\Phi} \lambda A$.
                    
                    Что было в лемме (и что мы потеряли):
                    
                    $\inf\limits_{A \subset G} \left( \lambda G \cdot \sup\limits_{G} f \right) = \lambda A \cdot \sup\limits_{A} f$.
                    
                    $G$~--- открытое, тогда
                    
                    $\nu G \leqslant \sup\limits_{G} J_{\Phi} \cdot \lambda G$.
                    
                    $\nu A \leqslant \nu G \leqslant \lambda \lambda A \sup\limits_{A} f$.
                    
            \end{enumerate}
            
\end{document}
