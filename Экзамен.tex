\documentclass{article}

\usepackage[T2A]{fontenc}
\usepackage[utf8]{inputenc}
\usepackage[russian]{babel}
\parindent 0pt
\parskip 8pt
\usepackage{setspace}
\usepackage{etaremune}
\usepackage{amsmath}
\usepackage{amssymb}
\usepackage{amsfonts}
\usepackage[left=2.3cm, right=2.3cm, top=2.7cm, bottom=2.7cm, bindingoffset=0cm]{geometry}
\usepackage{latexsym}
\usepackage[unicode, pdftex]{hyperref}
\usepackage{xcolor}
\usepackage{graphicx}
\usepackage{mathtools}
\graphicspath{ {./images/} }

\doublespacing

\everymath{\displaystyle}

\begin{document}

\newcommand{\R}[0]{\mathbb{R}}
\newcommand{\RM}[0]{\mathbb{R}^m}
\newcommand{\dist}[0]{\mathrm{dist}}
\newcommand{\rang}[0]{\mathrm{rang} $\ $}
\newcommand{\grad}[0]{\mathrm{grad} $\ $}
\newcommand{\Lin}[0]{\mathrm{Lin} $\ $}

\tableofcontents

\newpage 

\part{Интеграл по мере}

\newpage

    \section{Интеграл ступенчатой функции}
    
        $f = \sum\limits_{k = 1}^n \lambda_k \cdot \chi_{E_k}$, $f \geqslant 0$, где $E_k \in \mathcal{A}$~--- допустимое разбиение, тогда интеграл ступенчатой функции $f$ на множестве $X$ есть
                
        $\int\limits_{X} f d \mu = \int\limits_{X} f(x) d \mu(x) = \sum\limits_{k = 1}^n \lambda_k \mu E_k$ 
        
        Дополнительно будем считать, что $0 \cdot \infty = \infty \cdot 0 = 0$.
                
        \subsection{Свойства}
                
            \begin{itemize}
                
                \item Интеграл не зависит от допустимого разбиения:
                    
                    $f = \sum \alpha_j \chi_{F_j} = \sum\limits_{k,\, j} \lambda_k \chi_{E_k \cap F_j}$, тогда $\int F = \sum \lambda_k \mu E_k = \sum\limits_{k} \lambda_k \sum\limits_j \mu (E_k \cap F_j) = \sum \alpha_j \mu F_i = \int F$;
                        
                \item $f \leqslant g$, то $\int\limits_{X} f d \mu \leqslant \int\limits_{X} g d \mu$.
                
            \end{itemize}
            
    \newpage
    
    \section{Интеграл неотрицательной измеримой функции}
    
        $f \geqslant 0$, измерима, тогда интеграл неотрицательной измеримой функции $f$ есть
        
        $\int\limits_{X} f d \mu = \sup\limits_{\substack{\text{$g$ - ступ.} \\ 0 \leqslant g \leqslant f}} \left( \int\limits_{X} g d \mu \right)$.
            
        \subsection{Свойства}
                
            \begin{itemize}
                
                \item Для ступенчатой функции $f$ (при $f \geqslant 0$) это определение даёт тот же интеграл, что и для ступенчатой функции;
                    
                \item $0 \leqslant \int\limits_{X} f \leqslant +\infty$;
                    
                \item $0 \leqslant g \leqslant f$, $g$~--- ступенчатая, $f$~--- измеримая, тогда $\int\limits_{X} g \leqslant \int\limits_{X} f$.
                    
            \end{itemize}
                
    \newpage
    
    \section{Суммируемая функция}
    
        $f$~--- измеримая, $f_+$ и $f_-$~--- срезки, тогда если $\int\limits_{X} f_+$ или $\int\limits_{X} f_-$~--- конечен, тогда интеграл суммируемой функции есть
            
        $\int\limits_{X} f d \mu = \int\limits_{X} f_+ - \int\limits_{X} f_-$. 
                
        Если $\int\limits_{X} f \neq \pm \infty$, то говорят, что $f$~--- \textit{суммируемая}, а также $\int |f|$~--- конечен ($|f| = f_+ + f_-$).
                
        \subsection{Свойство}
                
            Если $f \geqslant 0$~--- измерима, то это определение даёт тот же интеграл, что и интеграл измеримой неотрицательной функции.
            
    \newpage
                
    \section{Интеграл суммируемой функции}
        
        $E \subset X$~--- измеримое множество, $f$~--- измеримо на $X$, тогда интеграл $f$ по множеству $E$ есть
        
        $\int\limits_{E} f d \mu := \int\limits_{X} f \chi_E d \mu$. 
        
        $f$~--- суммируемая на $E$ если $\int\limits_{E} f+-$ и $\int\limits_{E} f_-$~--- конечны одновременно.
            
        \subsection{Свойства} 
                
            \begin{itemize}
                    
                \item $f = \sum \lambda_k \chi_{E_k}$, то $\int\limits_{E} f = \sum \lambda_k \mu \left( E_k \cap E \right)$;
                
                \item $f \geqslant 0$~--- измерима, тогда $\int\limits_{E} f d \mu = \sup\limits_{\substack{\text{$g$ - ступ.} \\ 0 \leqslant g \leqslant f}} \left( \int\limits_{X} g d \mu \right)$.
                        
            \end{itemize}
                    
    \newpage
    
    $(X, \mathcal{A}, \mu)$~--- произвольное пространство с мерой.
    
    $\mathcal{L}^0 (X)$~--- множество измеримых почти везде конечных функций.
        
    \section{Простейшие свойства интеграла Лебега}
    
        \begin{enumerate}
        
            \item \textit{Монотонность}: 
            
                $f \leqslant g \Rightarrow \int\limits_{E} f \leqslant \int\limits_{E} g$.
            
                \subsection{Доказательство}
                
                    \begin{itemize}
                    
                        \item $\sup\limits_{\substack{\text{$\widetilde{f}$ - ступ.} \\ 0 \leqslant \widetilde{f} \leqslant f}} \left( \int\limits_{X} \widetilde{f} d \mu \right) \leqslant \sup\limits_{\substack{\text{$\widetilde{g}$ - ступ.} \\ 0 \leqslant \widetilde{g} \leqslant g}} \left( \int\limits_{X} \widetilde{g} d \mu \right)$;
                        
                        \item $f$ и $g$~--- произвольные, то работаем со срезками, и $f_+ \leqslant g_+$, а $f_- \geqslant g_-$, тогда очевидно и для интегралов.
                        
                    \end{itemize}
            
            \item $\int\limits_{E} 1 \cdot d \mu = \mu E$, $\int\limits_{E} 0 \cdot d \mu = 0$.
            
                \subsection{Доказательство}
                
                    По определению.
            
            \item $\mu E = 0$, $f$~--- измерима, тогда $\int\limits_{E} f = 0$.
            
                \subsection{Доказательство}
                
                    \begin{itemize}
                    
                        \item $f$~--- ступенчатая, то по определению интеграла для ступенчатых функций получаем $0$;
                        
                        \item $f \geqslant 0$~--- измеримая, то по определению интеграла для измеримых неотрицательных функций также получаем $0$;
                        
                        \item $f$~--- любая, то разбиваем на срезки $f_+$ и $f_-$ и снова получаем $0$.
                        
                    \end{itemize}
                    
            \item 
            
                \begin{enumerate}
                
                    \item $\int -f = - \int f$;
                    \item $\forall c > 0 : \int cf = c \int f$.
                
                \end{enumerate}
                
                \subsection{Доказательство}
                
                    \begin{itemize}
                    
                        \item $(-f)_+ = f_-$ и $(-f)_ = f_+$ и $\int -f = f_- - f_+ = - \int f$.
                    
                        \item $f \geqslant 0$~--- очевидно, $\sup\limits_{\substack{\text{$g$ - ступ.} \\ 0 \leqslant g \leqslant cf}} \left( \int g \right) = c \sup\limits_{\substack{\text{$g$ - ступ.} \\ 0 \leqslant g \leqslant f}} \left( \int g \right)$.
                        
                    \end{itemize}
                    
            \item Пусть существует $\int\limits_{E} f d \mu$, тогда $\left| \int\limits_{E} f \right| \leqslant \int\limits_{E} |f|$.
            
                \subsection{Доказательство}
                
                    $- |f| \leqslant f \leqslant |f|$,
                    
                    $- \int\limits_{E} |f| \leqslant \int\limits_{E} f \leqslant \int\limits_{E} |f|$.
                    
            \item $f$~--- измерима на $E$, $\mu E < +\infty$, $\forall x \in E : a \leqslant f(x) \leqslant b$. Тогда 
            
                $a \mu E \leqslant \int\limits_{E} f \leqslant b \mu E$.
                
                \subsection{Доказательство}
                    
                    $\int\limits_{E} a \leqslant \int\limits_{E} f \leqslant \int\limits_{E} b$,
                    
                    $a \mu E \leqslant \int\limits_{E} f \leqslant b \mu E$.
                
        \end{enumerate}
        
    \newpage
    
    \section{Счетная аддитивность интеграла (по множеству)}
    
        \subsection{Лемма}
    
            $A = \bigsqcup A_i$, где $A$, $A_i$~--- измеримы, $g \geqslant 0$~--- ступенчатые. Тогда
        
            $\int\limits_{A} g d \mu = \sum\limits_{i = 1}^{+\infty} \int\limits_{A_i} g d \mu$.
            
            \subsubsection{Доказательство}
        
                $g = \sum \lambda_k \chi_{E_k}$.
            
                $\int\limits_A g d \mu = \sum \lambda_k \mu (A \cap E_k) = \sum\limits_{k} \lambda_k \sum\limits_{i} \mu (A_i \cap E_k) = \sum\limits_i \left( \sum\limits_k \lambda_k \mu (A_i \cap E_k ) \right) = \sum\limits_i \int\limits_{A_i} g d \mu$.
            
        \subsection{Теорема}
    
            $f : C \rightarrow \overline{R}$, $f \geqslant 0$~--- измеримая на $A$, $A$~--- измерима, $A = \bigsqcup A_i$, все $A_i$~--- измеримы. Тогда
        
            $\int\limits_{A} f d \mu = \sum\limits_{i} \int\limits_{A_i} f d \mu$
            
            \subsubsection{Доказательство}
        
                \begin{itemize}
            
                    \item $\leqslant$
                
                        $g$~--- ступенчатая, $0 \leqslant g \leqslant f$, тогда $\int\limits_A g = \sum \int\limits_{A_i} g \leqslant \sum \int\limits_{A_i} f$. Осталось перейти к $\sup$.
                    
                    \item $\geqslant$
                
                        $A = A_1 \sqcup A_2$, $\sum \lambda_k \chi_{E_k} = g_1 \leqslant f \chi_{A_1}$, $g_2 \leqslant f \cdot \chi_{A_2} = \sum \lambda_k \chi_{E_k}$, $g_1 + g_2 \leqslant f \cdot \chi_{A}$
                    
                        $\int\limits_{A_1} g_1 + \int\limits_{A_2} g_2 = \int\limits_{A} g_1 + g_2$.
                    
                        переходим к $\sup$ $g_1$ и $g_2$
                    
                        $\int\limits_{A_1} f + \int\limits_{A_2} f \leqslant \int\limits_{A} f$
                    
                        по индукции разобьём для $A = A_1 \sqcup A_2 \sqcup \ldots \sqcup A_n$, $A = \bigsqcup\limits^{+\infty}_{i = 1} A_i$ и $A = A_1 \sqcup A_2 \sqcup \ldots \sqcup A_n \sqcup B_n$, где $B_n = \bigsqcup\limits_{i \geqslant n + 1} A_i$, тогда
                    
                        $\int\limits_{A} \geqslant \sum\limits^n_{i = 1} \int\limits_{A_i} f + \int\limits_{B} f \geqslant \sum\limits^n_{i = 1} \int\limits_{A_i} f \Rightarrow \int\limits_{A}f \geqslant \sum\limits^{+\infty}_{i = 1} \int\limits_{A_i} f$
                    
                \end{itemize}
            
        \subsection{Следствие}
    
            $f \geqslant 0$~--- измеримая, $\nu : \mathcal{A} \rightarrow \overline{\mathbb{R}}_+$, $\nu E = \int\limits_{E} f d \mu$. Тогда $\nu$~--- мера.
            
        \subsection{Следствие 2}
    
            $A = \bigsqcup\limits_{i = 1}^{+\infty} A_i$, $f$~--- суммируемая на $A$, тогда 
        
            $\int\limits_{A} f = \sum\limits_{i} \int\limits_{A_i} f$.
        
\newpage

\part{Предельный переход под знаком интеграла}

\newpage

    \section{Теорема Леви}
    
        $(X, \mathcal{A}, \mu)$, $f_n$~--- измерима, $\forall n : 0 \leqslant f_n(x) \leqslant f_{n + 1} (x)$ при почти всех $x$.
        
        $f(x) = \lim\limits_{n \rightarrow +\infty} f_n(x)$ при почти всех $x$. Тогда
        
        $\lim\limits_{n \rightarrow +\infty} \int\limits_{X} f_n(x) d \mu = \int\limits_{X} f d \mu$.
        
        \subsection{Доказательство}
        
            $f$~--- измерима как предел измеримых функций.
            
            \begin{itemize}
            
                \item $\leqslant$
                
                    $f_n(x) \leqslant f(x)$ почти везде, тогда $\forall n : \int\limits_{X} f_n(x) d \mu \leqslant \int\limits_{X} f d \mu$, откуда следует, что и предел интегралов не превосходит интеграл предела.
                    
                \item $\geqslant$
                
                    Достаточно доказать, что для любой ступенчатой функции $g : 0 \leqslant g \leqslant f$ верно $\lim \int\limits_{X} f_n \geqslant \int\limits_{X} g$.
                    
                    Достаточно доказать, что $\forall c \in (0, 1)$ верно $\lim \int\limits_{X} f_n \geqslant c \int\limits_{X} g$.
                    
                    $E_n := X \left( f_n \geqslant cg \right)$, $E_n \subset E_{n + 1} \subset \ldots$.
                    
                    $\bigcup E_n = X$, т.к. $c < 1$, то $c g(x) < f(x)$, $f_n(x) \rightarrow f(x) \Rightarrow f_n$ попадёт в ''зазор'' $c g(x) < f(x)$.
                    
                    $\int\limits_{X} f_n \geqslant \int\limits_{E_n} f_n \geqslant \int\limits_{E_n} c g = c \int\limits_{E_n} g$,
                    
                    $\lim\limits_{n \rightarrow +\infty} \int\limits_{X} f_n \geqslant \lim\limits_{n \rightarrow +\infty} c \int\limits_{E_n} g = c \int\limits_{X} g$, потому что это непрерывность снизу меры $A \mapsto \int\limits_{A} g$.
                    
            \end{itemize}
    
    \newpage
    
    \section{Линейность интеграла Лебега}
    
        Пусть $f$, $g$~--- измеримы на $E$, $f \geqslant 0$, $g \geqslant 0$. Тогда $\int\limits_{E} f + g = \int\limits_{E} f + \int\limits_{E} g$.
        
        \subsection{Доказательство}
        
            Если $f$, $g$~--- ступенчатые, то очевидно.
            
            Разберём общий случай. Существуют ступенчатые функции $f_n : 0 \leqslant f_n \leqslant f_{n + 1} \leqslant \ldots \leqslant f$, и $g_n : 0 \leqslant g_n \leqslant g_{n + 1} \leqslant \ldots \leqslant g$, и $f_n(x) \rightarrow f(x)$ и $g_n(x) \rightarrow g(x)$. Тогда
            
            $\int\limits_{E} f_n + g_n = \int\limits_{E} f_n + \int\limits_{E} g_n$, сделаем предельный переход, значит при $n \rightarrow +\infty$
            
            $\int\limits_{E} f + g = \int\limits_{E} f + \int\limits_{E} g$
            
        \subsection{Следствие}
        
            Пусть $f$, $g$~--- суммируемые на множестве $E$, тогда $f + g$ тоже суммируема и $\int\limits_{E} f + g = \int\limits_{E} f + \int\limits_{E} g$.
            
            \subsubsection{Доказательство}
            
                $(f + g)_{\pm} \leqslant | f + g | \leqslant |f| + |g|$.
                
                $h := f + g$,
                
                $h_+ - h_- = f_+ - f_- + g_+ - g_-$,
                
                $h_+ + f_- + g_- = h_- + f_+ + g_+$,
                
                $\int h_+ + \int f_- + \int g_- = \int h_- + \int f_+ \int g_+$,
                
                $\int h_+ - \int h_- = \int f_+ - \int f_- + \int g_+ - \int g_-$, тогда
                
                $\int h = \int f + \int g$.
    
    \newpage
        
    \section{Теорема об интегрировании положительных рядов}
    
        $u_n \geqslant 0$ почти везде, измеримы на $E$. Тогда
        
        $\int\limits_{E} \left( \sum\limits^{+\infty}_{i = 1} u_n \right) d \mu = \sum\limits^{+\infty}_{n = 1} \int\limits_{E} u_n d \mu$.
        
        \subsection{Доказательство}
        
            Очевидно по теореме Леви.
            
            $S(x) = \sum\limits^{+\infty}_{n = 1} u_n(x)$ и $p \leqslant S_N \leqslant S_{N + 1} \leqslant \ldots$ и $S_N \rightarrow S(X)$.
            
            $\lim\limits_{n \rightarrow +\infty} \int\limits_{E} S_N = \int\limits_{E} S$,
            
            $\lim\limits_{n \rightarrow +\infty} \sum\limits^n_{k = 1} \int\limits_{E} u_k(x) = \int\limits_{E} S(x) d \mu$.
            
        \subsection{Следствие}
        
            $u_n$~--- измеримая функция, $\sum\limits^{+\infty}_{n = 1} \int\limits_{E} | u_n | < +\infty$. Тогда
            
            $\sum u_n$~--- абсолютно сходится почти везде на $E$.
            
            \subsubsection{Доказательство}
            
                $S(x) = \sum\limits^{+\infty}_{n = 1} | u_n(x) |$
                
                $\int\limits_{E} S(x) = \sum\limits^{+\infty}_{n = 1} \int\limits | u_n(x) | < +\infty$, значит $S(x)$ конечна почти всюду.
    
\newpage

\part{17.02.2020}

    \subsection{Теорема}
    
        $x_m \in \mathbb{R}$ и $\sum a_n$~--- абс. сходится.
    
        Тогда $\sum \frac{a_n}{\sqrt{| x - x_m |}}$ абсолютно сходится при почти всех $x$.
    
        \subsubsection{Доказательство}
            
            $\int\limits^{A}_{-A} \frac{| a_n |}{\sqrt{ | x - x_m |}} \leqslant | a_n | \int\limits^{A - x_m}_{-A - x_m} \frac{dx}{\sqrt{x}} \leqslant |a_n| \int\limits^{A}_{-A} \frac{dx}{\sqrt{x}} = 4 \sqrt{A} |a_n|$, а ряд $\sum 4 \sqrt A |a_n|$~--- сходится.
    
    \subsection{Абсолютная непрерывность интеграла}
    
        $f$~--- суммируемая функция, тогда
        
        $\forall \varepsilon > 0 : \exists \delta > 0 : \forall E \in \mathcal{A} : \mu E < \delta : \left| \int\limits_{E} f \right| < \varepsilon$.
        
        \subsubsection{Доказательство}
        
            $X_n = X \left( f \geqslant n \right)$, $X_n \supset X_{n + 1} \supset \ldots$, $\mu \left( \bigcap\limits^{+\infty}_{n = 1} X_n \right) = 0$.
            
            Тогда $\forall \varepsilon > 0 : \exists n_{\varepsilon} : \int\limits_{X_{n_{\varepsilon}}} |f| < \frac{\varepsilon}{2}$ ($A \mapsto \int\limits_{A} |f|$~--- мера, тогда $\int\limits_{\bigcap X_n} |f| = 0$ и по непрерывности меры сверху).
            
            $\delta := \frac{\varepsilon}{2 n_{\varepsilon}}$, берём $E : \mu E < \delta$.
            
            $\left| \int\limits_{E} f \right| \leqslant \int\limits_{E} |f| = \int\limits_{E \cap X_{n_{\varepsilon}}} |f| + \int\limits_{E \setminus X_{n_{\varepsilon}}} |f| \leqslant \int\limits_{X_{n_{\varepsilon}}} |f| + n_{\varepsilon} \mu E < \frac{\varepsilon}{2} + n_{\varepsilon} \frac{\varepsilon}{2 n_{\varepsilon}} = \varepsilon$.
            
        \subsubsection{Следствие}
        
            $e_n$~--- измеримное множество, $\mu e_n \rightarrow 0$, $f$~--- суммируемая. Тогда $\int\limits_{e_n} f \rightarrow 0$.
            
\part{Произведение мер}

    $(X, \mathcal{A}, \mu)$ и $(Y, \mathcal{B}, \nu)$.
   
    $\mathcal{A} \times \mathcal{B} = \left\{ A \times B, A \in \mathcal{A}, B \in \mathcal{B} \right\}$~--- семейство подмножеств в $X \times Y$.
        
    \subsection{Лемма}
            
        $\mathcal{A}$, $\mathcal{B}$~--- полукольцо, значит и $\mathcal{A} \times \mathcal{B}$~--- полукольцо.
                
    $\mathcal{A} \times \mathcal{B}$~--- полукольцо \textit{измеримых прямоугольников} (на самом деле это не всегда так).
            
    $\mu_0 \left( A \times B \right) = \mu A \cdot \mu B$.
    
    \subsection{Теорема}
        
        \begin{enumerate}
        
            \item $\mu_0$~--- мера на полукольце $\mathcal{A} \times \mathcal{B}$;
            
            \item $\mu$, $\nu$~--- $\sigma$-конечное, значит $\mu_0$~--- $\sigma$-конечное.
            
        \end{enumerate}
        
        \subsubsection{Доказательство}
        
            Проверим счётную аддитивность $\mu_0$. $\chi_{A \times B} (x, y) = \chi_A(x) \cdot \chi_B(y)$, $(x, y) \in X \times Y$.
            
            $P = \bigsqcup\limits_{\text{сч.}} P_k$~--- измеримые прямоугольники. $P = A \times B$ и $P_k = A_k \times B_k$, $\chi_P = \sum \chi_{P_k}$.
            
            $\chi_A(x) \chi_B(y) = \sum\limits_k \chi_{A_k}(x) \chi_{B_k}(y)$. Интегрируем по $\nu$ (по пространству $Y$).
            
            $\chi_A(x) \cdot \nu(B) = \sum \chi_{A_k}(x) \nu (B_k)$. Интегрируем по $\mu$.
            
            $\mu A \nu B = \sum \mu A_k \cdot \nu B_k$.
            
            $X = \bigcup X_k$, $Y = \bigcup Y_j$, где $\mu X_k$ и $\nu Y_j$~--- конечные, $X \times Y = \bigcup\limits_{k, j} X_k \times Y_j$.
    
    $\left( \mathbb{R}^m, \mathcal{M}^m, \lambda_m \right)$ и $\left( \mathbb{R}^n, \mathcal{M}^n, \lambda_n r\right)$.
    
    $\left( X \times Y, \mathcal{A} \times \mathcal{B}, \mu_0 \right)$, где $\mathcal{A} \times \mathcal{B}$~--- полукольцо.
    
    Запускаем теорему о продолжении меры
    
    $\rightsquigarrow \left( X \times Y, \mathcal{A} \otimes \mathcal{B}, \mu \right)$ (крестик в кружочке), где $\mathcal{A} \times \mathcal{B}$~--- $\sigma$-алгебра.
    
    $\mu$, $\nu$~--- $\sigma$-конечная, следовательно продолжение определено однозначно.
            
    \subsection{Замечание}
    
        произведение мер ассоциативна.
    
    \subsection{Теорема}
    
        $\lambda_{m + n}$ если произведение мер $\lambda_m$ и $\lambda_n$.
        
        без доказательства.
        
    $X$, $Y$ $C \subset X \times Y$, $C_x = \left\{ y \in Y : (x, y) \in C \right\} \subset Y$~--- сечение множества $C$, $C^y = \left\{ x \in X : (x, y) \in C \right\}$.
    
    Допустимы и объедения, пересечения и т.п.
    
    \subsection{Принцип Кавальери}
    
        $(X, \mathcal{A}, \mu)$ и $(Y, \mathcal{B}, \nu)$, $\mu$, $\nu$~--- $\sigma$-конечные, полные. 
        
        $m = \mu \times \nu$, $C \in \mathcal{A} \otimes \mathcal{B}$. Тогда
        
        \begin{enumerate}
        
            \item при почти всех $x \in X$ сечение $C_x \in \mathcal{B}$;
            
            \item $x \mapsto \nu (C_x)$~--- измерима* на $X$;
            
            \item $m C = \int\limits_{X} \nu (C_x) d \mu(x)$.
            
        \end{enumerate}
        
        *Рассматривается почти везде.
        
        \subsubsection{Замечание}
        
            \begin{enumerate}
            
                \item $C$~--- измеримая, не следует, что $\forall x : C_x$~--- измеримое.
                
                \item $\forall x$, $\forall y$, $C_x$, $C^y$~--- измеримы, но не следует, что $C$~--- измеримо (из Серпинскиго).
                
            \end{enumerate}
            
        \subsubsection{Доказательство}
        
            $D$~--- класс множеств $X \times Y$, для который принцип Кавальери верен.
            
            \begin{enumerate}
            
                \item $D \times \mathcal{B} \subset D$, $C = A \times B$, $C_x = B, x \in A, \varnothing, x \notin A$ (сделать красиво).
                
                    $x \mapsto X_x: \nu B \cdot \chi_A(x)$.
                    
                    $\int\limits_{X} \nu B \chi_A(x) d \mu(x) = \mu A \nu B = m C$.
                    
                \item $E_i$~--- дизъюнктные, $E_i \in D$. Тогда $\bigsqcup E_i \in D$.
                
                    $(E_i)_x$~--- измерное при почти всех $x$. 
                    
                    При почти всех $x$ все сечения $(F_i)_x$, $i = 1, 2, \ldots$~--- измеримое.
                    
                    $E_x = \bigsqcup (E_i)_x$~--- измеримое при почти всех $x$.
                    
                    $\nu E_x = \sum \nu (E_i)_x$, значит $x \mapsto \nu E_x$ измеримая функция.
                    
                    $\int\limits_{X} \nu E_x d \mu = \int\limits_{X} \sum \nu (E_i)_x d \mu = \sum \int\limits_{X} \nu (E_i)_x d \mu = \sum m E_i = m E$
                
                \item $E_i \in D$, $\ldots \supset E_i \supset E_{i + 1} \supset \ldots$, $E = \bigcap\limits^{+\infty}_{i = 1} E_i$, $m E_i < +\infty$. Тогда $E \in D$.
                
                    $\int\limits_{X} \nu (E_i)_x d\mu = m E_i < +\infty \Rightarrow \nu (E_i)_x$~--- почти везде конечны.
                    
                    $(E_i)_x \supset (E_{i + 1})_x \supset \ldots$, $E_x = \bigcap\limits^{+\infty}_{i = 1} (E_i)_x \Rightarrow E_x$~--- измеримое при почти всех $x$.
                    
                    при почти всех $x$ (для тех $x$, для который $\nu (E_i)_x$~--- конечные сразу все $i$ или при $i = 1$), поэтому можно утверждать, что $\nu E_x = \lim\limits_{i \rightarrow +\infty} \nu (E_i)_x \Rightarrow x\mapsto \nu E_X$~--- измерима.
                    
                    $\int\limits_{X} \nu E_x d \mu = \int\limits_{X} \lim (\nu E_i)_x = \lim\limits_{i \rightarrow +\infty} \int\limits_{X} \nu (E_i)_x d \mu = \lim m E_i = m E$ (по непрерывности сверху меры $m$).
                    
                    Перестановка пределов доказывается из теоремы Лебега, которую ещё не доказывали $|\nu (E_i)_x | \leqslant \nu (E_1)_x$~--- суммируемая функция.
                    
                    Кажется, что мы доказали, что если $A_{ij} \in \mathcal{A} \times \mathcal{B}$, то $\bigcap\limits_{j} \left( \bigcup\limits_{i} A_{ij} \right) \in D$.
        
                    $m E = \inf \left( \sum m P_k, \ E \subset \bigcup P_k \right)$.
    
                \item $m E = 0 \Rightarrow E \in D$. $H = \bigcap\limits_{j} \bigcup\limits_{i} P_{ij}$, $m H = 0$ ($P_{ij} \in \mathcal{A} \times \mathcal{B}$), тогда $E \subset H$ ($H \in D$).
                
                    $0 = m H = \int\limits_{X} \nu H_x d \mu \Rightarrow \nu H_x = 0$ при почти всех $x$, но $E_x \subset H_x \Rightarrow$ при почти всех $x$ $\nu E_x = 0$, значит и $\int \nu E_x = 0 = m E$.
                    
                \item $C \in \mathcal{A} \otimes \mathcal{B}$, $m C < +\infty \Rightarrow C \in D$.
                
                    Для множества $C$ существует множество $e$, что $m e = 0$ и $H = \bigcap \bigcup P_{ij}$ и $C = H \setminus e$, $C_x = H_x \setminus e_x$ и $m C = m H$.
                    
                    $\nu e_x = 0$ при почти всех $x$, значит $\nu C_x = \nu H_x - \nu e_x$ при почти всех $x$.
                    
                    $\int\limits_{X} \nu C_x d \mu = \int\limits_{X} \nu H_x - \nu e_x = \int\limits_{X} \nu H_x - \int\limits_{X} \nu e_x = mH = m C$.
                    
                \item $C$~--- произведение, $m$ измеримое множество, $X = \bigsqcup X_k$ и $Y = \bigsqcup Y_j$, тогда $C = \bigsqcup\limits_{i, j} \left( C \bigcap \left( X_i \times Y_j \right) \right) \in D$ по пункту $2$. ($\mu X_k$, $\mu Y_j$~--- конечные).
                
            \end{enumerate}
        
        \subsubsection{Следствие}
        
            $C \in Q \otimes B$, $P_1(C) := \left\{ x : C_x \neq \varnothing \right\}$, тогда если $P_1(C)$~--- измеримое в $X$, тогда $m C = \int\limits_{P_1(C)} \nu C_x d \mu x)$.
        
        \subsubsection{Замечание}
        
            Из того, что $C$ измеримое, не следует, что его проекция измерима.

        \subsubsection{Следствие}
        
            $f : [a, b] \rightarrow \mathbb{R}$, непрерывное. Тогда $\int\limits^b_a f(x) dx = \int\limits_{[a, b]} f d \lambda_1$.
            
            \textit{Доказательство}
            
                Достаточно для $f \geqslant 0$. 
                
                $f$~--- непрерывно $\Rightarrow C = \Pi \Gamma \left(f, [a, b] \right)$ измеримо в $\mathbb{R}^2$ (почти очевидно).
                
                $C_x = [0, f(x)]$ (или $\varnothing$) $\Rightarrow$ измеримость $\lambda_1 C_x = f(x)$.
                
                $\int\limits^b_a f(x) dx = \lambda_2 \left( \Pi \Gamma \left( f, [a, b] \right) \right) = \int\limits_{[a, b]} f(x) d \lambda_1 (x)$.
                
        \subsubsection{Замечание}
            
            $f \geqslant 0$ измеримое, значит $\lambda_2 \Pi \Gamma (f, [a, b]) = \int\limits_{[a, b]} f(x) d \lambda_2(x)$.
            
    $f : X \times Y \rightarrow \overline{\mathbb{R}}$, $C \in X \times Y$, $C_x$, $f_x : C_x \rightarrow \mathbb{R}$, т.е. $y \mapsto f(x, y)$, аналогично $f^y : C^y \rightarrow \overline{\mathbb{R}}$.
    
    \subsection{Теорема Тонелли}
    
        $(X, \mathcal{A}, \mu)$, $(Y, \mathcal{B}, \nu)$, $\mu$, $\nu$~--- $\sigma$-конечные, полные. $m = \mu \times \nu$.
        
        $f : X \times Y \rightarrow \overline{\mathbb{R}}$, $f \geqslant 0$, измеримое. Тогда
        
        \begin{enumerate}
        
            \item при почти всех $x$ функция $f_x$~--- измерима почти везде на $Y$ [аналогично при почти всех $y$ функция $f^y$ также измерима на $X$].
            
            \item $x \mapsto \varphi(x) = \int\limits_Y f_x(y) d \nu (y) = \int\limits_Y f(x, y) d \nu (y)$~--- измерима* на $X$ [аналогично $y \mapsto \psi(y) = \int\limits_X f(x, y) d \mu(x)$~--- измерима* на $Y$].
            
            \item $\int\limits_{X \times Y} f(x, y) d \mu = \int\limits_{Y} \left( \int\limits_{X} f (x, y) d \mu (x) \right) d \nu (y) = \int\limits_X \left( \int\limits_Y f(x, y) d \nu (y) \right) d \mu (x)$
            
        \end{enumerate}
    
        *Почти везде
        
        \subsubsection{Доказательство}
        
            \begin{enumerate}
            
                \item $f = \chi_c$, $C \subset X \times Y$, измеримая. $f_x = \chi_{C_x} (y)$. $C_x$~--- измеримое при почти всех $x \Rightarrow f_x$~--- измеримая при почти всех $x$.
                
                    $\varphi(x) = \int\limits_{Y} \chi_{C_x} (y) d \nu (y) = \nu (C_x)$ ($x \mapsto \nu C_x$~--- измерима по принципу Кавальери).
                    
                    $\int\limits_{X} \varphi(x) = \int\limits_{X} \nu C_X = m C = \int\limits_{X \times Y} \chi_C d m$.
                
                \item $f = \sum\limits_{\text{кон.}} a_k \chi_{C_k}$, $f \geqslant 0$.
                
                    $f_x = \sum a_k \chi_{(C_k)_x} (y)$.
                    
                    $x \mapsto \int f_x(y) d \nu(y) = \sum a_k \nu (C_k)_x$~--- измеримая (отдельные слагаемые~--- измеримые, значит и вся сумма измеримая).
                    
                    $\int\limits_X \left( \int\limits_{Y} f_x(y) d \nu \right) d \mu = \sum a_k \int\limits_X \nu (C_k)_x d \mu = \sum a_k m C_k = \int\limits_{X \times Y} f d m$
                    
                \item $f \geqslant 0$, $g_n$~--- ступенчатые, что $\ldots \leqslant g_n \leqslant g_{n + 1} \leqslant \ldots$, $\lim\limits_{n \rightarrow +\infty} g_n = f$.
                
                    $f_x = \lim\limits_{n \rightarrow +\infty} (g_n)_x$~--- измерима как предел измеримых функций.
                    
                    $\varphi(x) = \int\limits_Y f_x(y) d \nu (y) = \lim\limits_{n \rightarrow +\infty} \int\limits_{Y} g_n d \nu = \lim\limits_{n \rightarrow +\infty} \varphi_n(x)$, значит $\varphi(x)$ измерима из-за измеримости $\varphi_n$ (Теорема Леви).
                
                    $g_n \leqslant g_{n + 1} \leqslant \ldots \Rightarrow \varphi_n(x) \leqslant \varphi_{n + 1} (x) \leqslant \ldots$
                    
                    $\int\limits_{X} \varphi(x) = \lim\limits_{n \rightarrow +\infty} \int\limits_{X} \varphi_n(x) = \lim\limits_{n \rightarrow} \int\limits_{X \times Y} g_n d m = \int\limits_{X \times Y} f dm$ (по теореме Леви)
                    
            \end{enumerate}
            
        Везде должна быть приговорка ''при почти всех $x$''.
        
    \subsection{Теорема Фубини}
    
        $(X, \mathcal{A}, \mu)$, $(Y, \mathcal{B}, \nu)$, $\mu$, $\nu$~--- $\sigma$-конечные, полные.
        
        $f : X \times Y \rightarrow \overline{\mathbb{R}}$, суммируемая. Тогда
        
        \begin{enumerate}
        
            \item при почти всех $x$ функция $f_x$~--- суммируемая почти везде на $Y$ [аналогично при почти всех $y$ функция $f^y$ также измерима на $X$].
            
            \item $x \mapsto \varphi(x) = \int\limits_Y f_x(y) d \nu (y) = \int\limits_Y f(x, y) d \nu (y)$~--- суммируемая* на $X$ [аналогично $y \mapsto \psi(y) = \int\limits_X f(x, y) d \mu(x)$~--- суммируемая* на $Y$].
            
            \item $\int\limits_{X \times Y} f(x, y) d \mu = \int\limits_{Y} \left( \int\limits_{X} f (x, y) d \mu (x) \right) d \nu (y) = \int\limits_X \left( \int\limits_Y f(x, y) d \nu (y) \right) d \mu (x)$
            
        \end{enumerate}
        
        *почти везде
        
        без доказательства
    
        \subsubsection{Следствие}
        
            $\int\limits_C f = \int\limits_{X \times Y} f \chi_C = \int\limits_X \left( \int\limits_Y f \cdot \chi_C \right) d \mu = \int\limits_{P_1(C)} \left( \int\limits_{C_x} f(x, y) d \nu(y) \right) d \mu (x)$.
            
            $P_1(C)$~--- проекция, измеримая, $\left\{ x : C_x \neq \varnothing \right\}$.
            
    $B(0, 1) \subset \mathbb{R}^m$, Хотим найти $\lambda_m B(0, 1) = \alpha_m$.
    
    $\lambda_m B(0, R) = \alpha_m \cdot R^M$.
    
    $x_1^2 + x_2^2 + \ldots + x_m^2 \leqslant 1$.
    
    интеграл обычного кружочка: $\int \chi_B d \lambda_2 = \int\limits^1_{-1} \int\limits^{\sqrt{1 - x^2}}_{-\sqrt{1 - x^2}} 1 dy dy dx = \int\limits^1_{-1} 2 \sqrt{1 - x^2} dx = \pi$
    
    $\alpha_m = \int\limits_{\mathbb{R}^m} \chi_B = \int\limits^1_{-1} \left( \int\limits_{B(0, \sqrt{1 - x_1^2}) \subset \mathbb{R}^{m -1}} 1 d \nu \right) dx_1 = \int\limits^1_{-1} (1 - x_1^2)^{\frac{m - 1}{2}} \alpha_{m - 1} d x_1$.
    
    $B(x, y) = \int\limits^1_) t^{x - 1} (1 - t)^{y - 1} dt$.
    
    $B(x, y) = \dfrac{\Gamma(x) \Gamma(y)}{\Gamma(x + y)}$, $\Gamma(n) = (n - 1)!$, $\Gamma(x + 1) = \Gamma(x) \cdot x$.
    
    Тогда объём шара в $\mathbb{R}^m$ равен $\alpha_{m - 1} 2 \int\limits^1_0 (1 - t)^{\frac{m - 1}{2}} t^{-\frac{1}{2}} dt = B(\frac{1}{2}, \frac{m + 1}{2}) \alpha_{m - 1}$. Тогда объём шара можно переписать как $\dfrac{\Gamma (\frac{1}{2}) \Gamma (\frac{m + 1}{2})}{\Gamma (\frac{m}{2} + 1) \alpha_{m - 1}}$.
    
    
\end{document}
