\documentclass{article}

\usepackage[T2A]{fontenc}
\usepackage[utf8]{inputenc}
\usepackage[russian]{babel}
\parindent 0pt
\parskip 8pt
\usepackage{setspace}
\usepackage{etaremune}
\usepackage{amsmath}
\usepackage{amssymb}
\usepackage{amsfonts}
\usepackage[left=2.3cm, right=2.3cm, top=2.7cm, bottom=2.7cm, bindingoffset=0cm]{geometry}
\usepackage{latexsym}
\usepackage[unicode, pdftex]{hyperref}
\usepackage{xcolor}
\usepackage{graphicx}
\usepackage{mathtools}
\usepackage{MnSymbol,wasysym}
\graphicspath{ {./images/} }

\doublespacing

\everymath{\displaystyle}

\begin{document}

\newcommand{\R}[0]{\mathbb{R}}
\newcommand{\RM}[0]{\mathbb{R}^m}
\newcommand{\dist}[0]{\mathrm{dist}}
\newcommand{\rang}[0]{\mathrm{rang} $\ $}
\newcommand{\grad}[0]{\mathrm{grad} $\ $}
\newcommand{\Lin}[0]{\mathrm{Lin} $\ $}

\tableofcontents

\newpage 

\part{Интеграл по мере}

\newpage

    \section{Интеграл ступенчатой функции}
    
        $f = \sum\limits_{k = 1}^n \lambda_k \cdot \chi_{E_k}$, $f \geqslant 0$, где $E_k \in \mathcal{A}$~--- допустимое разбиение, тогда интеграл ступенчатой функции $f$ на множестве $X$ есть
                
        $\int\limits_{X} f d \mu = \int\limits_{X} f(x) d \mu(x) = \sum\limits_{k = 1}^n \lambda_k \mu E_k$ 
        
        Дополнительно будем считать, что $0 \cdot \infty = \infty \cdot 0 = 0$.
                
        \subsection{Свойства}
                
            \begin{itemize}
                
                \item Интеграл не зависит от допустимого разбиения:
                    
                    $f = \sum \alpha_j \chi_{F_j} = \sum\limits_{k,\, j} \lambda_k \chi_{E_k \cap F_j}$, тогда $\int F = \sum \lambda_k \mu E_k = \sum\limits_{k} \lambda_k \sum\limits_j \mu (E_k \cap F_j) = \sum \alpha_j \mu F_i = \int F$;
                        
                \item $f \leqslant g$, то $\int\limits_{X} f d \mu \leqslant \int\limits_{X} g d \mu$.
                
            \end{itemize}
            
    \newpage
    
    \section{Интеграл неотрицательной измеримой функции}
    
        $f \geqslant 0$, измерима, тогда интеграл неотрицательной измеримой функции $f$ есть
        
        $\int\limits_{X} f d \mu = \sup\limits_{\substack{\text{$g$ - ступ.} \\ 0 \leqslant g \leqslant f}} \left( \int\limits_{X} g d \mu \right)$.
            
        \subsection{Свойства}
                
            \begin{itemize}
                
                \item Для ступенчатой функции $f$ (при $f \geqslant 0$) это определение даёт тот же интеграл, что и для ступенчатой функции;
                    
                \item $0 \leqslant \int\limits_{X} f \leqslant +\infty$;
                    
                \item $0 \leqslant g \leqslant f$, $g$~--- ступенчатая, $f$~--- измеримая, тогда $\int\limits_{X} g \leqslant \int\limits_{X} f$.
                    
            \end{itemize}
                
    \newpage
    
    \section{Суммируемая функция}
    
        $f$~--- измеримая, $f_+$ и $f_-$~--- срезки, тогда если $\int\limits_{X} f_+$ или $\int\limits_{X} f_-$~--- конечен, тогда интеграл суммируемой функции есть
            
        $\int\limits_{X} f d \mu = \int\limits_{X} f_+ - \int\limits_{X} f_-$. 
                
        Если $\int\limits_{X} f \neq \pm \infty$, то говорят, что $f$~--- \textit{суммируемая}, а также $\int |f|$~--- конечен ($|f| = f_+ + f_-$).
                
        \subsection{Свойство}
                
            Если $f \geqslant 0$~--- измерима, то это определение даёт тот же интеграл, что и интеграл измеримой неотрицательной функции.
            
    \newpage
                
    \section{Интеграл суммируемой функции}
        
        $E \subset X$~--- измеримое множество, $f$~--- измеримо на $X$, тогда интеграл $f$ по множеству $E$ есть
        
        $\int\limits_{E} f d \mu := \int\limits_{X} f \chi_E d \mu$. 
        
        $f$~--- суммируемая на $E$ если $\int\limits_{E} f+-$ и $\int\limits_{E} f_-$~--- конечны одновременно.
            
        \subsection{Свойства} 
                
            \begin{itemize}
                    
                \item $f = \sum \lambda_k \chi_{E_k}$, то $\int\limits_{E} f = \sum \lambda_k \mu \left( E_k \cap E \right)$;
                
                \item $f \geqslant 0$~--- измерима, тогда $\int\limits_{E} f d \mu = \sup\limits_{\substack{\text{$g$ - ступ.} \\ 0 \leqslant g \leqslant f}} \left( \int\limits_{X} g d \mu \right)$.
                        
            \end{itemize}
                    
    \newpage
    
    $(X, \mathcal{A}, \mu)$~--- произвольное пространство с мерой.
    
    $\mathcal{L}^0 (X)$~--- множество измеримых почти везде конечных функций.
        
    \section{Простейшие свойства интеграла Лебега}
    
        \begin{enumerate}
        
            \item \textit{Монотонность}: 
            
                $f \leqslant g \Rightarrow \int\limits_{E} f \leqslant \int\limits_{E} g$.
            
                \subsection{Доказательство}
                
                    \begin{itemize}
                    
                        \item $\sup\limits_{\substack{\text{$\widetilde{f}$ - ступ.} \\ 0 \leqslant \widetilde{f} \leqslant f}} \left( \int\limits_{X} \widetilde{f} d \mu \right) \leqslant \sup\limits_{\substack{\text{$\widetilde{g}$ - ступ.} \\ 0 \leqslant \widetilde{g} \leqslant g}} \left( \int\limits_{X} \widetilde{g} d \mu \right)$;
                        
                        \item $f$ и $g$~--- произвольные, то работаем со срезками, и $f_+ \leqslant g_+$, а $f_- \geqslant g_-$, тогда очевидно и для интегралов.
                        
                    \end{itemize}
            
            \item $\int\limits_{E} 1 \cdot d \mu = \mu E$, $\int\limits_{E} 0 \cdot d \mu = 0$.
            
                \subsection{Доказательство}
                
                    По определению.
            
            \item $\mu E = 0$, $f$~--- измерима, тогда $\int\limits_{E} f = 0$.
            
                \subsection{Доказательство}
                
                    \begin{itemize}
                    
                        \item $f$~--- ступенчатая, то по определению интеграла для ступенчатых функций получаем $0$;
                        
                        \item $f \geqslant 0$~--- измеримая, то по определению интеграла для измеримых неотрицательных функций также получаем $0$;
                        
                        \item $f$~--- любая, то разбиваем на срезки $f_+$ и $f_-$ и снова получаем $0$.
                        
                    \end{itemize}
                    
            \item 
            
                \begin{enumerate}
                
                    \item $\int -f = - \int f$;
                    \item $\forall c > 0 : \int cf = c \int f$.
                
                \end{enumerate}
                
                \subsection{Доказательство}
                
                    \begin{itemize}
                    
                        \item $(-f)_+ = f_-$ и $(-f)_ = f_+$ и $\int -f = f_- - f_+ = - \int f$.
                    
                        \item $f \geqslant 0$~--- очевидно, $\sup\limits_{\substack{\text{$g$ - ступ.} \\ 0 \leqslant g \leqslant cf}} \left( \int g \right) = c \sup\limits_{\substack{\text{$g$ - ступ.} \\ 0 \leqslant g \leqslant f}} \left( \int g \right)$.
                        
                    \end{itemize}
                    
            \item Пусть существует $\int\limits_{E} f d \mu$, тогда $\left| \int\limits_{E} f \right| \leqslant \int\limits_{E} |f|$.
            
                \subsection{Доказательство}
                
                    $- |f| \leqslant f \leqslant |f|$,
                    
                    $- \int\limits_{E} |f| \leqslant \int\limits_{E} f \leqslant \int\limits_{E} |f|$.
                    
            \item $f$~--- измерима на $E$, $\mu E < +\infty$, $\forall x \in E : a \leqslant f(x) \leqslant b$. Тогда 
            
                $a \mu E \leqslant \int\limits_{E} f \leqslant b \mu E$.
                
                \subsection{Доказательство}
                    
                    $\int\limits_{E} a \leqslant \int\limits_{E} f \leqslant \int\limits_{E} b$,
                    
                    $a \mu E \leqslant \int\limits_{E} f \leqslant b \mu E$.
                
        \end{enumerate}
        
    \newpage
    
    \section{Счетная аддитивность интеграла (по множеству)}
    
        \subsection{Лемма}
    
            $A = \bigsqcup A_i$, где $A$, $A_i$~--- измеримы, $g \geqslant 0$~--- ступенчатые. Тогда
        
            $\int\limits_{A} g d \mu = \sum\limits_{i = 1}^{+\infty} \int\limits_{A_i} g d \mu$.
            
            \subsubsection{Доказательство}
        
                $g = \sum \lambda_k \chi_{E_k}$.
            
                $\int\limits_A g d \mu = \sum \lambda_k \mu (A \cap E_k) = \sum\limits_{k} \lambda_k \sum\limits_{i} \mu (A_i \cap E_k) = \sum\limits_i \left( \sum\limits_k \lambda_k \mu (A_i \cap E_k ) \right) = \sum\limits_i \int\limits_{A_i} g d \mu$.
            
        \subsection{Теорема}
    
            $f : C \rightarrow \overline{R}$, $f \geqslant 0$~--- измеримая на $A$, $A$~--- измерима, $A = \bigsqcup A_i$, все $A_i$~--- измеримы. Тогда
        
            $\int\limits_{A} f d \mu = \sum\limits_{i} \int\limits_{A_i} f d \mu$
            
            \subsubsection{Доказательство}
        
                \begin{itemize}
            
                    \item $\leqslant$
                
                        $g$~--- ступенчатая, $0 \leqslant g \leqslant f$, тогда $\int\limits_A g = \sum \int\limits_{A_i} g \leqslant \sum \int\limits_{A_i} f$. Осталось перейти к $\sup$.
                    
                    \item $\geqslant$
                
                        $A = A_1 \sqcup A_2$, $\sum \lambda_k \chi_{E_k} = g_1 \leqslant f \chi_{A_1}$, $g_2 \leqslant f \cdot \chi_{A_2} = \sum \lambda_k \chi_{E_k}$, $g_1 + g_2 \leqslant f \cdot \chi_{A}$
                    
                        $\int\limits_{A_1} g_1 + \int\limits_{A_2} g_2 = \int\limits_{A} g_1 + g_2$.
                    
                        переходим к $\sup$ $g_1$ и $g_2$
                    
                        $\int\limits_{A_1} f + \int\limits_{A_2} f \leqslant \int\limits_{A} f$
                    
                        по индукции разобьём для $A = A_1 \sqcup A_2 \sqcup \ldots \sqcup A_n$, $A = \bigsqcup\limits^{+\infty}_{i = 1} A_i$ и $A = A_1 \sqcup A_2 \sqcup \ldots \sqcup A_n \sqcup B_n$, где $B_n = \bigsqcup\limits_{i \geqslant n + 1} A_i$, тогда
                    
                        $\int\limits_{A} \geqslant \sum\limits^n_{i = 1} \int\limits_{A_i} f + \int\limits_{B} f \geqslant \sum\limits^n_{i = 1} \int\limits_{A_i} f \Rightarrow \int\limits_{A}f \geqslant \sum\limits^{+\infty}_{i = 1} \int\limits_{A_i} f$
                    
                \end{itemize}
            
        \subsection{Следствие}
    
            $f \geqslant 0$~--- измеримая, $\nu : \mathcal{A} \rightarrow \overline{\mathbb{R}}_+$, $\nu E = \int\limits_{E} f d \mu$. Тогда $\nu$~--- мера.
            
        \subsection{Следствие 2}
    
            $A = \bigsqcup\limits_{i = 1}^{+\infty} A_i$, $f$~--- суммируемая на $A$, тогда 
        
            $\int\limits_{A} f = \sum\limits_{i} \int\limits_{A_i} f$.
        
\newpage

\part{Предельный переход под знаком интеграла}

\newpage

    \section{Теорема Леви}
    
        $(X, \mathcal{A}, \mu)$, $f_n$~--- измерима, $\forall n : 0 \leqslant f_n(x) \leqslant f_{n + 1} (x)$ при почти всех $x$.
        
        $f(x) = \lim\limits_{n \rightarrow +\infty} f_n(x)$ при почти всех $x$. Тогда
        
        $\lim\limits_{n \rightarrow +\infty} \int\limits_{X} f_n(x) d \mu = \int\limits_{X} f d \mu$.
        
        \subsection{Доказательство}
        
            $f$~--- измерима как предел измеримых функций.
            
            \begin{itemize}
            
                \item $\leqslant$
                
                    $f_n(x) \leqslant f(x)$ почти везде, тогда $\forall n : \int\limits_{X} f_n(x) d \mu \leqslant \int\limits_{X} f d \mu$, откуда следует, что и предел интегралов не превосходит интеграл предела.
                    
                \item $\geqslant$
                
                    Достаточно доказать, что для любой ступенчатой функции $g : 0 \leqslant g \leqslant f$ верно $\lim \int\limits_{X} f_n \geqslant \int\limits_{X} g$.
                    
                    Достаточно доказать, что $\forall c \in (0, 1)$ верно $\lim \int\limits_{X} f_n \geqslant c \int\limits_{X} g$.
                    
                    $E_n := X \left( f_n \geqslant cg \right)$, $E_n \subset E_{n + 1} \subset \ldots$.
                    
                    $\bigcup E_n = X$, т.к. $c < 1$, то $c g(x) < f(x)$, $f_n(x) \rightarrow f(x) \Rightarrow f_n$ попадёт в ''зазор'' $c g(x) < f(x)$.
                    
                    $\int\limits_{X} f_n \geqslant \int\limits_{E_n} f_n \geqslant \int\limits_{E_n} c g = c \int\limits_{E_n} g$,
                    
                    $\lim\limits_{n \rightarrow +\infty} \int\limits_{X} f_n \geqslant \lim\limits_{n \rightarrow +\infty} c \int\limits_{E_n} g = c \int\limits_{X} g$, потому что это непрерывность снизу меры $A \mapsto \int\limits_{A} g$.
                    
            \end{itemize}
    
    \newpage
    
    \section{Линейность интеграла Лебега}
    
        Пусть $f$, $g$~--- измеримы на $E$, $f \geqslant 0$, $g \geqslant 0$. Тогда $\int\limits_{E} f + g = \int\limits_{E} f + \int\limits_{E} g$.
        
        \subsection{Доказательство}
        
            Если $f$, $g$~--- ступенчатые, то очевидно.
            
            Разберём общий случай. Существуют ступенчатые функции $f_n : 0 \leqslant f_n \leqslant f_{n + 1} \leqslant \ldots \leqslant f$, и $g_n : 0 \leqslant g_n \leqslant g_{n + 1} \leqslant \ldots \leqslant g$, и $f_n(x) \rightarrow f(x)$ и $g_n(x) \rightarrow g(x)$. Тогда
            
            $\int\limits_{E} f_n + g_n = \int\limits_{E} f_n + \int\limits_{E} g_n$, сделаем предельный переход, значит при $n \rightarrow +\infty$
            
            $\int\limits_{E} f + g = \int\limits_{E} f + \int\limits_{E} g$
            
        \subsection{Следствие}
        
            Пусть $f$, $g$~--- суммируемые на множестве $E$, тогда $f + g$ тоже суммируема и $\int\limits_{E} f + g = \int\limits_{E} f + \int\limits_{E} g$.
            
            \subsubsection{Доказательство}
            
                $(f + g)_{\pm} \leqslant | f + g | \leqslant |f| + |g|$.
                
                $h := f + g$,
                
                $h_+ - h_- = f_+ - f_- + g_+ - g_-$,
                
                $h_+ + f_- + g_- = h_- + f_+ + g_+$,
                
                $\int h_+ + \int f_- + \int g_- = \int h_- + \int f_+ \int g_+$,
                
                $\int h_+ - \int h_- = \int f_+ - \int f_- + \int g_+ - \int g_-$, тогда
                
                $\int h = \int f + \int g$.
    
    \newpage
        
    \section{Теорема об интегрировании положительных рядов}
    
        $u_n \geqslant 0$ почти везде, измеримы на $E$. Тогда
        
        $\int\limits_{E} \left( \sum\limits^{+\infty}_{i = 1} u_n \right) d \mu = \sum\limits^{+\infty}_{n = 1} \int\limits_{E} u_n d \mu$.
        
        \subsection{Доказательство}
        
            Очевидно по теореме Леви.
            
            $S(x) = \sum\limits^{+\infty}_{n = 1} u_n(x)$ и $p \leqslant S_N \leqslant S_{N + 1} \leqslant \ldots$ и $S_N \rightarrow S(X)$.
            
            $\lim\limits_{n \rightarrow +\infty} \int\limits_{E} S_N = \int\limits_{E} S$,
            
            $\lim\limits_{n \rightarrow +\infty} \sum\limits^n_{k = 1} \int\limits_{E} u_k(x) = \int\limits_{E} S(x) d \mu$.
            
        \subsection{Следствие}
        
            $u_n$~--- измеримая функция, $\sum\limits^{+\infty}_{n = 1} \int\limits_{E} | u_n | < +\infty$. Тогда
            
            $\sum u_n$~--- абсолютно сходится почти везде на $E$.
            
            \subsubsection{Доказательство}
            
                $S(x) = \sum\limits^{+\infty}_{n = 1} | u_n(x) |$
                
                $\int\limits_{E} S(x) = \sum\limits^{+\infty}_{n = 1} \int\limits | u_n(x) | < +\infty$, значит $S(x)$ конечна почти всюду.
    
    \newpage
    
    \section{Абсолютная непрерывность интеграла}
    
        $f$~--- суммируемая функция, тогда верно:
        
        $$\forall \varepsilon > 0 : \exists \delta > 0 : \forall E \in \mathcal{A} : \mu E < \delta : \left| \int\limits_{E} f \right| < \varepsilon$$.
        
        \subsection{Доказательство}
        
            $X_n = X \left( f \geqslant n \right)$, $X_n \supset X_{n + 1} \supset \ldots$ и $\mu \left( \bigcap\limits^{+\infty}_{n = 1} X_n \right) = 0$.
            
            Тогда $\forall \varepsilon > 0 : \exists n_{\varepsilon} : \int\limits_{X_{n_{\varepsilon}}} |f| < \frac{\varepsilon}{2}$ ($A \mapsto \int\limits_{A} |f|$~--- мера, тогда $\int\limits_{\bigcap X_n} |f| = 0$ и по непрерывности меры сверху).
            
            $\delta := \frac{\varepsilon}{2 n_{\varepsilon}}$, берём $E : \mu E < \delta$.
            
            $\left| \int\limits_{E} f \right| \leqslant \int\limits_{E} |f| = \int\limits_{E \cap X_{n_{\varepsilon}}} |f| + \int\limits_{E \setminus X_{n_{\varepsilon}}} |f| \leqslant \int\limits_{X_{n_{\varepsilon}}} |f| + n_{\varepsilon} \mu E < \frac{\varepsilon}{2} + n_{\varepsilon} \frac{\varepsilon}{2 n_{\varepsilon}} = \varepsilon$.
            
        \subsection{Следствие}
        
            $e_n$~--- измеримое множество, $\mu e_n \rightarrow 0$, $f$~--- суммируемая. Тогда $\int\limits_{e_n} f \rightarrow 0$.
 
    \newpage
    
    \section{02.03.2020}
    
        $f_n \rightrightarrows f$ по мере то же самое, что и $\mu X( |f_n - f| \geqslant \varepsilon) \rightarrow 0$. Ещё есть способ $\int\limits_X |f_n - f| d \mu \rightarrow 0$. Можно ли вывести хоть какую-нибудь импликацию.
        
        $\Rightarrow$ нельзя, пример: $f_n(x) = \dfrac{1}{nx}$ в $(\mathbb{R}, \lambda)$, тогда $f_n \rightrightarrows 0$ по мере. а $\int \left| \dfrac{1}{nx} \right| d \mu = +\infty$.
        
        $\Leftarrow$ можно: $\mu X ( |f_n - f| \geqslant \varepsilon) = \int\limits_{x_n} 1 d \mu \leqslant \int\limits_{x_n} \dfrac{|f_n - f|}{\varepsilon} d \mu \leqslant \dfrac{1}{\varepsilon} \int\limits_X |f_n - f| \rightarrow 0$.
        
        Хотим доказать подобие $f_n \rightarrow f$, то $\int f_n \rightarrow \int f$.
        
        \subsection{Теорема Лебега о мажорированной сходимости}
        
            $f_n$, $f$~--- измеримые, почти везде конечные функции. $f_n \xRightarrow[\mu]{} f$. Также существует $g$, что:
            
            \begin{enumerate}
            
                \item $\forall n : |f_n| \leqslant g$ почти везде;
                
                \item $g$~--- суммируема на $X$ ($g$~--- мажоранта).
                
            \end{enumerate}
        
            Тогда $\int\limits_X |f_n - f| d \mu \rightarrow 0$, и тем более $\int\limits_X f_n \rightarrow \int\limits_X f$.
            
            \subsubsection{Доказательство}
            
                $f_n$~--- суммируема в силу первого утверждения про $g$, $f$~--- суммируема по следствию теоремы Рисса. Тем более $\left| \int\limits_X f_n - \int\limits_X f \right| \leqslant \left| \int\limits_X f_n - f \right| \leqslant \int |f_n  - f|$.
                
                \begin{enumerate}
                
                    \item $\mu X < +\infty$. Фиксируем $\varepsilon > 0$. $X_n := X(|f_n - f| \geqslant \varepsilon)$, $\mu X_n \rightarrow 0$.
                    
                        $\int\limits_X |f_n - f| = \int\limits_{x_n} + \int\limits_{x_n^c} \leqslant \int\limits_{x_n} 2g + \int\limits_{x_n^c} \varepsilon_0 \leqslant \int\limits_{x_n} 2g + \int\limits_x \varepsilon < \varepsilon (1 + \mu X)$. (при больших $n$ выражение $\int\limits_{x_n} 2g \leqslant \varepsilon$).
                        
                    \item $\mu X = +\infty$, $\varepsilon > 0$. 
                    
                        Утверждение: $\exists A$~--- измеримое, $\mu A$~--- конечное, $\int\limits_{X \setminus A} g < \varepsilon$.
                    
                        \textit{Доказательство}
                        
                            $\int G = \sup \left\{ \int g_n : h - \text{ступенчатая функция} 0 \leqslant h \leqslant g \right\}$
                            
                        $\exists h_0 : \int\limits_X g - \int\limits_X h_0 < \varepsilon$, $A := \mathrm{supp \ } h_0$. (где supp~--- носитель (support))
                        
                        $\int\limits_{X \setminus A} g + \int\limits_A g - h_0 < \varepsilon$.
                        
                        $\int\limits_X |f_n - f| = \int\limits_A + \int\limits_{X \setminus A} \leqslant \int\limits_A |f_n - f| + 2 \varepsilon < 3 \varepsilon$ при больших $n$.
                        
                \end{enumerate}
                
        \subsection{Теорема Лебега о мажорированной сходимости почти везде}
        
            $(X, \mathcal{A}, \mu)$, $f_n$, $f$~--- измеримые, $f_n \rightarrow f$~--- почти везде.
            
            Существует такая $g$, что:
            
            \begin{enumerate}
            
                \item $|f_n| \leqslant g$ почти везде;
                
                \item $g$~--- суммируема.
                
            \end{enumerate}
            
            \subsubsection{Доказательство}
            
                $f_n$, $f$~--- суммируемая, тем более~--- как и раньше.
                
                $h_n := \sup( |f_n - f|, |f_{n + 1} - f|, \ldots )$, $h_n$ убывает. $0 \leqslant h_n \leqslant 2g$.
                
                $\lim\limits_{n \rightarrow +\infty} h_n(x) = \overline{\lim} |f_n - f| = 0$ почти везде.
                
                $2g - h \geqslant 0$, возрастают, тогда по теореме Леви $\int\limits_X 2g - h \rightarrow \int\limits_X 2g$, значит $\int\limits_X h_n \rightarrow 0$, тогда $\int\limits_X |f_n - f| \leqslant \int\limits_X h_n \rightarrow 0$.
                
        \subsection{Теорема Фату}
        
            $(X, \mathcal{A}, \mu$, $f_n \geqslant 0$~--- измеримые, $f_n \rightarrow f$ почти везде. Если $\exists C > 0$, что $\forall n : \int\limits_X f_n \leqslant C$, то $\int\limits_X f \leqslant C$.
            
            \subsubsection{Замечание}
            
                Вообще говоря $\int\limits_X f_n \not\rightarrow \int\limits_X f$.
                
            \subsubsection{Доказательство}
            
                $g_n = \int (f_n, f_{n + 1}, \ldots)$.
                
                $g_n$ возрастает, $g_n \rightarrow f$ почти везде. $\lim g_n = \underline{\lim} f_n = f$ почти везде.
                
                $\int\limits_X g_n \leqslant \int\limits_X f_n \leqslant C$, тогда $\int F \leqslant C$.
                
            Примерчик
            
            $f_n = n \cdot \chi_{[0, \frac{1}{n}} \rightarrow 0$ почти везде.
            
            $\int\limits_{\mathbb{R}} f_n = 1$, $\int f = 0$.
            
            Положительность важна:
            
            $f_n \geqslant 0$, тогда $\int -f_n \leqslant -1$, но $\int f = 0 \geqslant -1$.
            
            \subsubsection{Следствие}
            
                $f_n \xRightarrow[\mu]{} f$ ($f_{n_k} \rightarrow f$).
                
            \subsubsection{Следствие 2}
            
                $f_n \geqslant 0$, измеримая. Тогда
                
                $\int\limits_X \underline{\lim} f_n \leqslant \underline{\lim} \int\limits_X f_n$.
                
                \text{Доказательство}
                
                    $\int\limits_X g_n \leqslant \int\limits_X f_n \leqslant C$.
                
                    Берём $n_k$
                    
                    $\underline{\lim} \left( \int\limits_X f_n \right) = \lim\limits_{k \rightarrow +\infty} \left( \int\limits_X f_{n_k} \right)$.
                    
                    $\int\limits_X f_{n_k} \rightarrow \lim \left( \int\limits_X f_n \right)$, а $\int\limits_x g_n \rightarrow \int\limits_X \underline{\lim} f_n$.
\newpage

\part{Произведение мер}

\newpage

    \section{Произведение мер}
    
        $(X, \mathcal{A}, \mu)$ и $(Y, \mathcal{B}, \nu)$~--- пространства с мерой.
   
        $\mathcal{A} \times \mathcal{B} = \left\{ A \times B, A \in \mathcal{A}, B \in \mathcal{B} \right\}$~--- семейство подмножеств в $X \times Y$.
        
        $\mathcal{A}$, $\mathcal{B}$~--- полукольца, значит и $\mathcal{A} \times \mathcal{B}$~--- полукольцо.
                
        $\mathcal{A} \times \mathcal{B}$~--- полукольцо \textit{измеримых прямоугольников} (на самом деле это не всегда так).
            
            
        Тогда введём меру на $A \times B$~--- $\mu_0 (A \times B) = \mu(A) \cdot \nu(B)$.
        
        Обозначим $(X \times Y, A \otimes B, \mu \times \nu)$ как произведение пространств с мерой.
        
    \newpage
    
    \section{Теорема о произведении мер}
        
        \begin{enumerate}
        
            \item $\mu_0$~--- мера на полукольце $\mathcal{A} \times \mathcal{B}$;
            
            \item $\mu$, $\nu$~--- $\sigma$-конечное, значит $\mu_0$~--- $\sigma$-конечное.
            
        \end{enumerate}
        
        \subsection{Доказательство}
        
            \begin{enumerate}
            
                \item 

                    Проверим счётную аддитивность $\mu_0$. $\chi_{A \times B} (x, y) = \chi_A(x) \cdot \chi_B(y)$, $(x, y) \in X \times Y$.
            
                    $P = \bigsqcup\limits_{\text{сч.}} P_k$~--- измеримые прямоугольники. $P = A \times B$ и $P_k = A_k \times B_k$, $\chi_P = \sum \chi_{P_k}$.
            
                    $\chi_A(x) \chi_B(y) = \sum\limits_k \chi_{A_k}(x) \chi_{B_k}(y)$. Интегрируем по $\nu$ (по пространству $Y$).
            
                    $\chi_A(x) \cdot \nu(B) = \sum \chi_{A_k}(x) \nu (B_k)$. Интегрируем по $\mu$.
            
                    $\mu A \cdot \nu B = \sum \mu A_k \cdot \nu B_k$.
            
                \item
                
                    $X = \bigcup X_k$, $Y = \bigcup Y_j$, где $\mu X_k$ и $\nu Y_j$~--- конечные, $X \times Y = \bigcup\limits_{k, j} X_k \times Y_j$.
    
                    $\left( \mathbb{R}^m, \mathcal{M}^m, \lambda_m \right)$ и $\left( \mathbb{R}^n, \mathcal{M}^n, \lambda_n \right)$.
    
                    $\left( X \times Y, \mathcal{A} \otimes \mathcal{B}, \mu_0 \right)$, где $\mathcal{A} \times \mathcal{B}$~--- полукольцо.
    
                    Запускаем теорему о продолжении меры.
    
                    $\rightsquigarrow \left( X \times Y, \mathcal{A} \otimes \mathcal{B}, \mu \right)$, где $\mathcal{A} \times \mathcal{B}$~--- $\sigma$-алгебра.
    
                    $\mu$, $\nu$~--- $\sigma$-конечная, следовательно продолжение определено однозначно.
            
            \end{enumerate}
            
        \subsection{Замечание}
    
            Произведение мер ассоциативно.
    
        \subsection{Дополнительная теорема (без доказательства)}
    
            $\lambda_{m + n}$ есть произведение мер $\lambda_m$ и $\lambda_n$.
        
    \newpage
    
    \section{Сечения множества}
    
        $X$, $Y$ и $C \subset X \times Y$, $C_x = \left\{ y \in Y : (x, y) \in C \right\} \subset Y$~--- сечение множества $C$, аналогично определим $C^y = \left\{ x \in X : (x, y) \in C \right\}$.
    
        Допустимы объедения, пересечения и т.п.
    
    \newpage

    \section{Принцип Кавальери}
    
        $(X, \mathcal{A}, \mu)$ и $(Y, \mathcal{B}, \nu)$, а также $\mu$, $\nu$~--- $\sigma$-конечные и полные. 
        
        $m = \mu \times \nu$, $C \in \mathcal{A} \otimes \mathcal{B}$. Тогда:
        
        \begin{enumerate}
        
            \item при почти всех $x \in X$ сечение $C_x \in \mathcal{B}$;
            
            \item $x \mapsto \nu (C_x)$~--- измерима (почти везде) на $X$;
            
            \item $m C = \int\limits_{X} \nu (C_x) d \mu(x)$.
            
        \end{enumerate}
        
        \subsection{Замечание}
        
            \begin{enumerate}
            
                \item $C$~--- измеримая $\not\Rightarrow$ что $\forall x : C_x$~--- измеримое.
                
                \item $\forall x$, $\forall y$, $C_x$, $C^y$~--- измеримы $\not\Rightarrow$ что $C$~--- измеримо (пример можно взять из Серпинскиго).
                
            \end{enumerate}
            
        \subsection{Доказательство}
        
            $D$~--- класс множеств $X \times Y$, для который принцип Кавальери верен.
            
            \begin{enumerate}
            
                \item $D \times \mathcal{B} \subset D$, $C = A \times B$, $C_x = 
                                                                        \begin{cases}
                                                                            B & x \in A \\ 
                                                                            \varnothing & x \notin A
                                                                        \end{cases}$.
                
                    $x \longmapsto C_x: \nu B \cdot \chi_A(x)$.
                    
                    $\int\limits_{X} \nu B \chi_A(x) d \mu(x) = \mu A \cdot \nu B = m C$.
                    
                \item $E_i$~--- дизъюнктные, $E_i \in D$. Тогда $\bigsqcup E_i \in D$.
                
                    $(E_i)_x$~--- измеримые при почти всех $x$. 
                    
                    При почти всех $x$ все сечения $(E_i)_x$, $i = 1, 2, \ldots$~--- измеримые.
                    
                    $E_x = \bigsqcup (E_i)_x$~--- измеримые при почти всех $x$.
                    
                    $\nu E_x = \sum \nu (E_i)_x$, значит $x \mapsto \nu E_x$ измеримая функция.
                    
                    $\int\limits_{X} \nu E_x d \mu = \int\limits_{X} \sum \nu (E_i)_x d \mu = \sum \int\limits_{X} \nu (E_i)_x d \mu = \sum m E_i = m E$
                
                \item $E_i \in D$, $\ldots \supset E_i \supset E_{i + 1} \supset \ldots$, $E = \bigcap\limits^{+\infty}_{i = 1} E_i$, $m E_i < +\infty$. Тогда $E \in D$.
                
                    $\int\limits_{X} \nu (E_i)_x d\mu = m E_i < +\infty \Rightarrow \nu (E_i)_x$~--- почти везде конечны.
                    
                    $(E_i)_x \supset (E_{i + 1})_x \supset \ldots$, $E_x = \bigcap\limits^{+\infty}_{i = 1} (E_i)_x \Rightarrow E_x$~--- измеримое при почти всех $x$.
                    
                    При почти всех $x$ (для тех $x$, для который $\nu (E_i)_x$~--- конечные сразу все $i$ или при $i = 1$), поэтому можно утверждать, что $\nu E_x = \lim\limits_{i \rightarrow +\infty} \nu (E_i)_x \Rightarrow x\mapsto \nu E_X$~--- измерима.
                    
                    $\int\limits_{X} \nu E_x d \mu = \int\limits_{X} \lim (\nu E_i)_x = \lim\limits_{i \rightarrow +\infty} \int\limits_{X} \nu (E_i)_x d \mu = \lim m E_i = m E$ (по непрерывности сверху меры $m$).
                    
                    Перестановка пределов доказывается из теоремы Лебега, которую ещё не доказывали $|\nu (E_i)_x | \leqslant \nu (E_1)_x$~--- суммируемая функция.
                    
                    Мы доказали, что если $A_{ij} \in \mathcal{A} \times \mathcal{B}$, то $\bigcap\limits_{j} \left( \bigcup\limits_{i} A_{ij} \right) \in D$.
        
                    $m E = \inf \left( \sum m P_k, \ E \subset \bigcup P_k \right)$.
    
                \item $m E = 0 \Rightarrow E \in D$. $H = \bigcap\limits_{j} \bigcup\limits_{i} P_{ij}$, $m H = 0$ ($P_{ij} \in \mathcal{A} \times \mathcal{B}$), тогда $E \subset H$ ($H \in D$).
                
                    $0 = m H = \int\limits_{X} \nu H_x d \mu \Rightarrow \nu H_x = 0$ при почти всех $x$, но $E_x \subset H_x \Rightarrow$ при почти всех $x$ $\nu E_x = 0$, значит и $\int \nu E_x = 0 = m E$.
                    
                \item $C \in \mathcal{A} \otimes \mathcal{B}$, $m C < +\infty \Rightarrow C \in D$.
                
                    Для множества $C$ существует множество $e$, что $m e = 0$ и $H = \bigcap \bigcup P_{ij}$ и $C = H \setminus e$, $C_x = H_x \setminus e_x$ и $m C = m H$.
                    
                    $\nu e_x = 0$ при почти всех $x$, значит $\nu C_x = \nu H_x - \nu e_x$ при почти всех $x$.
                    
                    $\int\limits_{X} \nu C_x d \mu = \int\limits_{X} \nu H_x - \nu e_x = \int\limits_{X} \nu H_x - \int\limits_{X} \nu e_x = mH = m C$.
                    
                \item $C$~--- произвольное, $m$-измеримое множество, $X = \bigsqcup X_k$ и $Y = \bigsqcup Y_j$, тогда $C = \bigsqcup\limits_{i, j} \left( C \bigcap \left( X_i \times Y_j \right) \right) \in D$ по пункту $2$. ($\mu X_k$, $\mu Y_j$~--- конечные).
                
            \end{enumerate}
        
        \subsection{Следствие}
        
            $C \in Q \otimes B$, $P_1(C) := \left\{ x : C_x \neq \varnothing \right\}$, тогда если $P_1(C)$~--- измеримое в $X$, тогда $m C = \int\limits_{P_1(C)} \nu C_x d \mu x$.
        
        \subsection{Замечание}
        
            Из того, что $C$ измеримое $\not\Rightarrow$ что его проекция измерима.

    \newpage
    
    \section{Совпадение определенного интеграла и интеграла Лебега}
        
        $f : [a, b] \rightarrow \mathbb{R}$, непрерывное. Тогда $\int\limits^b_a f(x) dx = \int\limits_{[a, b]} f d \lambda_1$.
            
        \subsection{Доказательство}
            
            Достаточно доказать для $f \geqslant 0$. 
                
            $f$~--- непрерывно $\Rightarrow C = \Pi \Gamma \left(f, [a, b] \right)$ измеримо в $\mathbb{R}^2$ (почти очевидно).
                
            $C_x = [0, f(x)]$ (или $\varnothing$) $\Rightarrow$ измеримость $\lambda_1 C_x = f(x)$.
                
            $\int\limits^b_a f(x) dx = \lambda_2 \left( \Pi \Gamma \left( f, [a, b] \right) \right) = \int\limits_{[a, b]} f(x) d \lambda_1 (x)$.
                
        \subsection{Замечание}
            
            $f \geqslant 0$ измеримое, значит $\lambda_2 \Pi \Gamma (f, [a, b]) = \int\limits_{[a, b]} f(x) d \lambda_2(x)$.
            
            $f : X \times Y \rightarrow \overline{\mathbb{R}}$, $C \in X \times Y$, $C_x$, $f_x : C_x \rightarrow \mathbb{R}$, т.е. $y \mapsto f(x, y)$, аналогично $f^y : C^y \rightarrow \overline{\mathbb{R}}$.
    
    \newpage
    
    \section{Теорема Тонелли}
    
        $(X, \mathcal{A}, \mu)$, $(Y, \mathcal{B}, \nu)$ и $\mu$, $\nu$~--- $\sigma$-конечные и полные, а также $m = \mu \times \nu$.
        
        $f : X \times Y \rightarrow \overline{\mathbb{R}}$, $f \geqslant 0$, измеримая. Тогда
        
        \begin{enumerate}
        
            \item при почти всех $x$ функция $f_x$~--- измерима почти везде на $Y$ (аналогично при почти всех $y$ функция $f^y$ также измерима на $X$);
            
            \item $x \mapsto \varphi(x) = \int\limits_Y f_x(y) d \nu (y) = \int\limits_Y f(x, y) d \nu (y)$~--- измерима почти везде на $X$ (аналогично $y \mapsto \psi(y) = \int\limits_X f(x, y) d \mu(x)$~--- измерима почти везде на $Y$);
            
            \item $\int\limits_{X \times Y} f(x, y) d \mu = \int\limits_{Y} \left( \int\limits_{X} f (x, y) d \mu (x) \right) d \nu (y) = \int\limits_X \left( \int\limits_Y f(x, y) d \nu (y) \right) d \mu (x)$.
            
        \end{enumerate}
        
        \subsection{Доказательство}
        
            \begin{enumerate}
            
                \item $f = \chi_c$, $C \subset X \times Y$, измеримая. $f_x = \chi_{C_x} (y)$. $C_x$~--- измеримое при почти всех $x \Rightarrow f_x$~--- измеримая при почти всех $x$.
                
                    $\varphi(x) = \int\limits_{Y} \chi_{C_x} (y) d \nu (y) = \nu (C_x)$ ($x \mapsto \nu C_x$~--- измерима по принципу Кавальери).
                    
                    $\int\limits_{X} \varphi(x) = \int\limits_{X} \nu C_X = m C = \int\limits_{X \times Y} \chi_C d m$.
                
                \item $f = \sum\limits_{\text{кон.}} a_k \chi_{C_k}$, $f \geqslant 0$.
                
                    $f_x = \sum a_k \chi_{(C_k)_x} (y)$.
                    
                    $x \mapsto \int f_x(y) d \nu(y) = \sum a_k \nu (C_k)_x$~--- измеримая (отдельные слагаемые~--- измеримые, значит и вся сумма измеримая).
                    
                    $\int\limits_X \left( \int\limits_{Y} f_x(y) d \nu \right) d \mu = \sum a_k \int\limits_X \nu (C_k)_x d \mu = \sum a_k m C_k = \int\limits_{X \times Y} f d m$
                    
                \item $f \geqslant 0$, $g_n$~--- ступенчатые, что $\ldots \leqslant g_n \leqslant g_{n + 1} \leqslant \ldots$, $\lim\limits_{n \rightarrow +\infty} g_n = f$.
                
                    $f_x = \lim\limits_{n \rightarrow +\infty} (g_n)_x$~--- измерима как предел измеримых функций.
                    
                    $\varphi(x) = \int\limits_Y f_x(y) d \nu (y) = \lim\limits_{n \rightarrow +\infty} \int\limits_{Y} g_n d \nu = \lim\limits_{n \rightarrow +\infty} \varphi_n(x)$, значит $\varphi(x)$ измерима из-за измеримости $\varphi_n$ (Теорема Леви).
                
                    $g_n \leqslant g_{n + 1} \leqslant \ldots \Rightarrow \varphi_n(x) \leqslant \varphi_{n + 1} (x) \leqslant \ldots$
                    
                    $\int\limits_{X} \varphi(x) = \lim\limits_{n \rightarrow +\infty} \int\limits_{X} \varphi_n(x) = \lim\limits_{n \rightarrow} \int\limits_{X \times Y} g_n d m = \int\limits_{X \times Y} f dm$ (по теореме Леви)
                    
            \end{enumerate}
            
        \textit{Везде должна быть приговорка \glqqпри почти всех $x$\grqq }.
        
    \newpage
    
    \section{Теорема Фубини}
    
        $(X, \mathcal{A}, \mu)$, $(Y, \mathcal{B}, \nu)$ и $\mu$, $\nu$~--- $\sigma$-конечные и полные.
        
        $f : X \times Y \rightarrow \overline{\mathbb{R}}$, суммируемая. Тогда
        
        \begin{enumerate}
        
            \item при почти всех $x$ функция $f_x$~--- суммируемая почти везде на $Y$ (аналогично при почти всех $y$ функция $f^y$ также измерима на $X$).
            
            \item $x \mapsto \varphi(x) = \int\limits_Y f_x(y) d \nu (y) = \int\limits_Y f(x, y) d \nu (y)$~--- суммируемая почти везде на $X$ (аналогично $y \mapsto \psi(y) = \int\limits_X f(x, y) d \mu(x)$~--- суммируемая почти везде на $Y$).
            
            \item $\int\limits_{X \times Y} f(x, y) d \mu = \int\limits_{Y} \left( \int\limits_{X} f (x, y) d \mu (x) \right) d \nu (y) = \int\limits_X \left( \int\limits_Y f(x, y) d \nu (y) \right) d \mu (x)$
            
        \end{enumerate}
        
        \textit{без доказательства} 
    
        \subsubsection{Следствие}
        
            $\int\limits_C f = \int\limits_{X \times Y} f \chi_C = \int\limits_X \left( \int\limits_Y f \cdot \chi_C \right) d \mu = \int\limits_{P_1(C)} \left( \int\limits_{C_x} f(x, y) d \nu(y) \right) d \mu (x)$.
            
            $P_1(C)$~--- проекция, измеримая, $\left\{ x : C_x \neq \varnothing \right\}$.
    
    \newpage
    
    \section{Какая-то нужная штука для лекции 02.03.2020, потом удалю}
    
        $B(0, 1) \subset \mathbb{R}^m$, Хотим найти $\lambda_m B(0, 1) = \alpha_m$.
    
        $\lambda_m B(0, R) = \alpha_m \cdot R^M$.
    
        $x_1^2 + x_2^2 + \ldots + x_m^2 \leqslant 1$.
    
        интеграл обычного кружочка: $\int \chi_B d \lambda_2 = \int\limits^1_{-1} \int\limits^{\sqrt{1 - x^2}}_{-\sqrt{1 - x^2}} 1 dy dy dx = \int\limits^1_{-1} 2 \sqrt{1 - x^2} dx = \pi$
    
        $\alpha_m = \int\limits_{\mathbb{R}^m} \chi_B = \int\limits^1_{-1} \left( \int\limits_{B(0, \sqrt{1 - x_1^2}) \subset \mathbb{R}^{m -1}} 1 d \nu \right) dx_1 = \int\limits^1_{-1} (1 - x_1^2)^{\frac{m - 1}{2}} \alpha_{m - 1} d x_1$.
    
        $B(x, y) = \int\limits^1_0 t^{x - 1} (1 - t)^{y - 1} dt$.
    
        $B(x, y) = \dfrac{\Gamma(x) \Gamma(y)}{\Gamma(x + y)}$, $\Gamma(n) = (n - 1)!$, $\Gamma(x + 1) = \Gamma(x) \cdot x$.
    
        Тогда объём шара в $\mathbb{R}^m$ равен $\alpha_{m - 1} 2 \int\limits^1_0 (1 - t)^{\frac{m - 1}{2}} t^{-\frac{1}{2}} dt = B(\frac{1}{2}, \frac{m + 1}{2}) \alpha_{m - 1}$. Тогда объём шара можно переписать как $\dfrac{\Gamma (\frac{1}{2}) \Gamma (\frac{m + 1}{2})}{\Gamma (\frac{m}{2} + 1) \alpha_{m - 1}}$.
    
\newpage

\part{Замена переменных в интеграле}

\newpage

    \section{Образ меры при отображении}
    
        $(X, \mathcal{A}, \mu)$ и $(Y, \mathcal{B}, )$ (пространство и алгебру изобрели, а меру нет).
    
        $\Phi : X \rightarrow Y$, $\forall B \in \mathcal{B}$ $\Phi^{-1}(B)$~--- измеримо ($\in \mathcal{A}$).
    
        $\nu : \mathcal{B} \rightarrow \overline{\mathbb{R}}$, $E \in \mathcal{B}$, $\nu E := \mu(\Phi^{-1}(E))$~--- это мера на $\mathcal{B}$, а также \textit{образ меры $\mu$ при отображении $\Phi$}.

        \subsection{Замечание 1}
        
            $\nu E = \int\limits_{\Phi^{-1}(E)} 1 d \mu$.
            
            $\nu \left( \bigsqcup B_i \right) = \mu \left( \Phi^{-1} \left( \bigsqcup B_i \right) \right) = \mu \left( \bigsqcup \Phi^{-1}(B_i) \right) = \sum \mu \Phi^{-1}(B_i) = \sum \nu B_i$.
    
        \subsection{Замечание 2}
        
            $f$~--- измерима относительна $\mathcal{B}$, тогда $f \circ \Phi$~--- измерима относительна $\mathcal{A}$. 
            
            $X \left( f \left( \Phi(x) \right) < a \right) = \Phi^{-1} \left( Y (f < a) \right)$.
    
    \newpage
    
    \section{Взвешенный образ меры}
        
        $\omega : X \rightarrow \overline{\mathbb{R}}$, $\omega \geqslant 0$, измеримая. 
        
        Тогда $\nu(B) := \int\limits_{\Phi^{-1}(B)} \omega d \mu$~--- мера, которая назначает \textit{взвешенный образ меры $\mu$}, где $\omega$~--- её вес.
        
    \newpage
    
    \section{Теорема о вычислении интеграла по взвешенному образу меры}
    
        $\Phi : X \rightarrow Y$~--- измеримое отображение, $\omega : X \rightarrow \overline{\mathbb{R}}$, $\omega \geqslant 0$~---измеримая на $X$. $\nu$~--- взвешенный образ меры $\mu$ ($\omega$~--- её вес). Тогда
        
        $\forall f \geqslant 0$~--- измеримой на $Y$ верно, что $f \circ \Phi$~--- измерима на $X$ и выполняется следующее свойство:
        
        $\int\limits_Y f(y) d \nu(y) = \int\limits_X f (\Phi(x)) \omega(x) d \mu(x)$.
            
        \subsection{Замечание}
        
            То же верно для случая $f$~--- суммируемая.
    
        \subsection{Доказательство}
        
            \begin{enumerate}
            
                \item $f = \chi_B$, $B \in \mathcal{B}$. Тогда $\left(f \circ \Phi \right)(x) = \begin{cases} 1 & \Phi(X) \in B \\ 0 & \Phi(x) \notin B \end{cases} \ = \chi_{\Phi^{-1}(B)}$.
                
                    Доказывать нечего \smiley{} : $\nu B = \int\limits_{\Phi(B)} \omega d \mu$;
                    
                \item $f$~-- ступенчатая, для каждой ступеньки~--- правда, и по линейности интеграла получаем результат;
                
                \item $f \geqslant 0$~--- измеримая. Теорема об аппроксимизации измеримых функций ступенчатыми плюс предельный переход по теореме Леви;
                
                \item $f$~--- измеримая, значит $|f|$~--- всё верно.
                
            \end{enumerate}
            
        \subsection{Следствие}
        
            $f$~--- суммируема на $Y$, $B \in \mathcal{B}$, $\int\limits_B f d \nu(y) = \int\limits_{\Phi^{-1}}(B) \left(f \circ \Phi\right) w d \mu$.
            
            Частный случай: $X = Y$, $\mathcal{A} = \mathcal{B}$, $\Phi = \mathrm{id}$, $\omega \geqslant 0$~--- измерима.
            
    \newpage
    
    \section{Плотность одной меры по отношению к другой}
    
        $\nu B = \int\limits_B \omega(x) d \mu(x)$, тогда $\omega$~--- \textit{плотность меры $\nu$ относительно меры $\mu$}.
            
        \subsection{Замечание}
        
            $\int\limits_X f(x) d \nu(x) = \int\limits_X f(x) \omega(x) d \mu(x)$.
    
    \newpage
            
    \section{Критерий плотности}
    
        $(X, \mathcal{A}, \mu)$, $\nu$~--- ещё одна мера на $\mathcal{A}$, $\omega \geqslant 0$~--- измеримая. Тогда
        
        $\omega$~--- плотность $\nu$ относительно $\mu$ $\Longleftrightarrow \forall A \in \mathcal{A}$ верно: $\inf\limits_{A} \omega \cdot \mu A \leqslant \nu A \leqslant \sup\limits_{A} \omega \cdot \mu A$ ($0 \cdot \infty = 0$).
        
        \subsubsection{Доказательство}
        
            \begin{itemize}
            
                \item $\Rightarrow$ Очевидно (интеграл $\mu A$ обладает этими свойствами из-за плотностей);
                
                \item $\Leftarrow$ Считаем, что $\omega > 0$. Для $\omega = 0$ получаем: $e := X(\omega = 0)$, $\nu e = 0 = \int\limits_e \omega d \mu$, тогда $\nu (A) = \int\limits_{A} \omega d \mu = 0$. 
                
                Теперь пусть $\omega > 0$, то $q \in (0, 1)$. $A_j := A(q^j \leqslant \omega \leqslant q^{j - 1})$, $j \in \mathbb{Z}$, $A = \bigsqcup\limits_{j \in \mathbb{Z}} A_j$.
                
                    $q^j \mu A_j \leqslant \nu A_j \leqslant q^{j - 1} \mu A_j$.
                    
                    $q^j \mu A_j \leqslant \int\limits_{A_j} \omega d \mu \leqslant q^{j - 1} \mu A_j$.
                    
                    $q \int\limits_A \omega d \mu = q \sum \int\limits_{A_j} \leqslant \sum q^j \mu A_j \leqslant \nu A \leqslant \dfrac{1}{q} \sum q^j \mu A_j \leqslant \dfrac{1}{q} \int\limits_A \omega$.
                    
                    Устремим $q \rightarrow 1$ и получим доказательство равенства.
                    
            \end{itemize}
           
    \newpage
    
    \section{Единственность плотности}
    
        $f$, $g$~--- суммируемые на $X$, $\forall A$~--- измеримых верно: $\int\limits_A f = \int\limits_a g$. Тогда $f = g$ почти везде.
        
        \subsubsection{Доказательство}
        
            $h = f - g$, $\forall A$~--- измеримых, $\int\limits_A h = 0$.
            
            $A_+ = X(h \geqslant 0)$, $A_- = X(h < 0)$, $A_+ \bigcap A_- = \varnothing$.
            
            $\int\limits_{A_+} |h| = \int\limits_{A_+} h = 0$.
            
            $\int\limits_{A_-} |h| = - \int\limits_{A_-} h = 0$.
            
            $X = A_+ \bigsqcup A_-$, $\int\limits_{X} |h| = 0$, тогда $h = 0$.
            
        \subsection{Следствие}
        
            Плотность $\nu$ относительно $\nu$ определена однозначно с точностью до $\mu$ почти везде.
    
    \newpage
        
    \section{Лемма об образе малых кубических ячеек}
    
        $\Phi : O \subset \mathbb{R}^m \rightarrow \mathbb{R}^m$, $a \in O$. $\Phi$~--- дифференцируема $G$ в окрестности точки $a$, $\det \Phi'(a) \neq 0$. Пусть $c > | \det \Phi'(a) |$. 
        
        Тогда существует такое $\delta > 0$, что для любого куба $Q \subset B(a, \delta)$, $a \in Q$ верно, что $c \cdot \lambda Q > \lambda \Phi(Q)$.
        
        \subsubsection{Доказательство}
        
            $L := \Phi'(a)$~--- обратимое линейное отображение. 
            
            $\Phi(x) = \Phi(a) + L (x - a) + o(x - a)$.
            
            $a + L^{-1}(\Phi(x) - \Phi(a)) = x + o(x - a)$ (увеличили в константу, поэтому \textit{о маленькое} остаётся \textit{о маленьким}).
            
            $\forall \varepsilon > 0$ можно записать шар $B_{\varepsilon}(a)$, что при $x \in B_{\varepsilon}(a)$ $\left| \psi(x) - x \right| < \dfrac{\varepsilon}{\sqrt{m}} |x - a|$.
            
            $Q \subset B_{\varepsilon}$, $a \in Q$~--- куб со стороной $h$, при $x \in Q : | \psi(x) - x| < \varepsilon h$. $|x_i - a_i| \leqslant h$.
            
            $x$, $y \in Q$, тогда $\left| \psi(x)_i - \psi(y)_i \right| = \left| \psi(x)_i - x_i \right| + \left| \psi(y)_i - y_i \right| + |x_i - y_i| \leqslant | \psi(x) - x | + |\psi(y) - y| + h < (1 + 2 \varepsilon)h$.
            
            $\psi(Q)$~--- содержится в кубе со стороной $(1 + 2 \varepsilon) h$, тогда $\lambda \psi(Q) \leqslant (1 + 2 \varepsilon)^m \lambda Q$.
            
            $\lambda \Phi(Q) \leqslant (1 + 2 \varepsilon)^m | \det L | \lambda Q < C \lambda Q$.
            
            Берём $\varepsilon : (1 + 2 \varepsilon) | \det L | < C$, где $\delta$~--- радиус $B_{\varepsilon} (a)$.
            
    $\lambda A = \inf\limits_{\text{G - открытое}, A \subset G} \lambda G$
            
    \newpage
    
    \section{Теорема об образе меры Лебега при диффеоморфизме}
    
        \subsection{Лемма}
    
            $f : \underset{\text{откр.}}{O} \subset \mathbb{R}^m \rightarrow \mathbb{R}$, $O$~--- непрерывное. $A$~--- измеримое, $A \subset Q \subset \overline{Q} \subset O$.
        
            Тогда $\int\limits_{A \subset G \text{открытое}} \left( \lambda (G) \sup\limits_{G} f \right) = \lambda A \sup\limits_{A} f$.
            
            Без доказательства.
        
        \subsection{Теорема}
        
            $\Phi : O \subset \mathbb{R}^m \rightarrow \mathbb{R}^m$~--- диффеоморфизм. $A \in \mathcal{M}^m$, $A \subset O$. Тогда
        
            $\lambda \Phi(A) = \int\limits_{A} \left| \det \Phi'(a) \right| d \lambda$.
        
        \subsubsection{Доказательство}
        
            $\nu A := \lambda \Phi(A)$. Верно ли, что $J_{\Phi} (x) := \left| \det \Phi'(x) \right|$~--- это плотность $\nu$ по отношению к $\mu$.
            
            Достаточно проверить, что $\forall A$ верно: $\inf\limits_{A} J_{\Phi} \cdot \lambda A \leqslant \nu A \leqslant \sup\limits_{A} J_{\Phi} \cdot \lambda A$.
            
            Достаточно проверить правое неравенство. Левое~--- правое для $\Phi^{-1}$ и $\widetilde{A} = \Phi(A)$.
            
            $\lambda \Phi^{-1} \left( \widetilde{A} \right) \leqslant \sup J_{\Phi^{-1}} \cdot \lambda \widetilde{A}$.
            
            $\lambda A \leqslant \sup \left| \det (\Phi^{-1})' \right| \lambda \Phi(A)$.
            
            $\sup \dfrac{1}{\left| \det \Phi' \right|}$
            
            $\dfrac{1}{\inf \left| \det \Phi' \right|}$
            
            \begin{enumerate}
            
                \item $A$~--- кубическая ячейка, $\overline{A} \subset O$. От противного: пусть оказалось, что $\lambda Q \sup J_{\Phi} < \nu Q$. Возьмём $c > \sup\limits_{Q} J_{\Phi}$, так, что $\lambda Q \cdot c < \nu Q$. Значит существует такая часть $Q_i$, что $\lambda Q_i \cdot c < \nu Q_i$. $\lambda Q_n \cdot c < nu Q_n$, $a = \bigcap \overline{Q_n}$, накроем точку $a$ этим кубиков. $c > \left| \det \Phi'(a) \right|$, тогда $\nu Q_n = \lambda \Phi(Q_n)$. Получили, что $\lambda \Phi(Q_n) > c \lambda Q_n$, а по лемме нужно наоборот.
            
                \item Оценка $\nu A \leqslant \sup J_{\Phi} \lambda A$, верна для случая, когда $A$~--- открытое множество.
                
                    $\nu Q \leqslant \sup\limits_{A} J_{\Phi} \lambda Q$.
                    
                    Суммируя по $Q$: $\nu A \leqslant \sup\limits_{A} J_{\Phi} \lambda A$.
                    
                    Что было в лемме (и что мы потеряли):
                    
                    $\inf\limits_{A \subset G} \left( \lambda G \cdot \sup\limits_{G} f \right) = \lambda A \cdot \sup\limits_{A} f$.
                    
                    $G$~--- открытое, тогда
                    
                    $\nu G \leqslant \sup\limits_{G} J_{\Phi} \cdot \lambda G$.
                    
                    $\nu A \leqslant \nu G \leqslant \lambda \lambda A \sup\limits_{A} f$.
                    
            \end{enumerate}

    \newpage
    
    $\forall A \in \mathcal{M}^m$, $\Phi(A)$~--- измерима
    
    $\lambda \Phi(A) = \int\limits_A \left| \det \Phi'(x) \right| d \lambda (x)$.
    
    $\Phi: X \rightarrow Y$
    
    $\nu(E) = \int\limits-{\Phi^{-1}(E)} \omega d \mu$.
    
    $E = \Phi(A)$.
    
    \newpage
    
    \section{Теорема о гладкой замене переменной в интеграле Лебега}
    
        $\Phi: O \subset \mathbb{R}^m \rightarrow \mathbb{R}^m$~--- диффеоморфизм, $f$ ~--- измеримое, $f \geqslant 0$, $\mathcal{O} = \Phi \left( O \right)$. Тогда
            
        $\int\limits_{\mathcal{O}} f(y) dy = \int\limits_O f \left(\Phi(x)\right) \left| \det \Phi'(x) \right| dx$.
            
        То же верно для суммируемой функции $f$.
        
        \subsection{Доказательство}
        
            Следует из теоремы об образе меры Лебега.
        
    \newpage
    
    \section{Сферические координаты в $\mathbb{R}^m$}
    
        $r$~--- расстояние от центра до точки
                    
        $\varphi_1$, $\varphi_2$, $\ldots$, $\varphi_{m - 1}$~--- соответствующие углы, определяются по индукции на меньшие подпространства.
                    
        $x_1 = r \cos \varphi_1$;
                    
        $x_2 = r \sin \varphi_1 \cos \varphi_2$;
                    
        $\vdots$
                    
        $x_m = r \sin \varphi_1 \sin \varphi_2 \ldots \sin \varphi_{m - 1}$.
                    
        $x_1, \ldots, x_m$. Выразим последние две переменные через угол $\varphi_{m - 1}$ и какое-то расстояние $\rho_{m - 1}$.
                    
        $x_1, \ldots, x_{m - 2}$, $\rho_{m - 1}$, $\varphi_{m - 1}$, тогда
                    
        $x_{m - 1} = \rho_{m - 1} \cos \varphi_{m - 1}$, а $x_m = \rho_{m - 1} \sin \varphi_{m - 1}$.
    
        $x_{m - 2} = \rho_{m - 2} \cos \varphi_{m - 2}$.
                    
        $\vdots$
                    
        Пусть осталось только $x_1$, тогда $x_1 = r \cos \varphi_1$ и $\rho_2 = r \sin \varphi_1$, т.е. $\rho_1 = r$.
                    
        $\int dx_1 \ldots dx_m = \int \rho_{m - 1} dx_1 \ldots dx_{m - 2} d\rho_{m - 1} d \varphi_{m - 1} = \int \rho^2_{m - 2} \sin \varphi_{m - 2} dx_1\ldots dx_{m - 3} d \rho_{m - 2} d \varphi_{m - 2} d \varphi_{m - 1} =$
        
        $= \int \rho^3_{m - 3} \sin^2 \varphi_{m - 3} \sin \varphi_{m - 2} dx_1 \ldots = \int r^{m - 1} \sin^{m - 2} \varphi_1 \sin^{m - 3} \varphi_2 \ldots \sin \varphi_{m - 2} \ldots$
                    
        $r^{m - 1} sin^{m - 2} \varphi_1 \sin^{m - 3} \varphi_2 \ldots \sin \varphi_{m - 2}$~--- это Якобиан.
                
    \newpage
    
    \section{Формула для Бета-функции}
    
        $B(s, t) = \int\limits^1_0 x^{s - 1} (1 - x)^{t - 1} dx = \dfrac{\Gamma(s) \Gamma(t)}{\Gamma (s + t)}$.
            
        \subsubsection{Доказательство}
            
            По определению гаммы-функции:
            
            $\Gamma(s) \Gamma(t) = \int\limits^{+\infty}_0 x^{s - 1} e^x \left( \int\limits^{+\infty}_0 y^{t - 1} e^{-y}dy \right) dx = \int\limits^{+\infty}_0 x^{s - 1} e^{-x} \int\limits_X (u - x)^{t - 1} e^{-u + x} du dx$, где $y = u - x$, 
            
            $\int\limits^{+\infty}_0 du \int\limits^u_0 dx x^{s - 1} (u - x)^{t - 1} e^{-u}$, заменим $x = uv$ и получим
            
            $\int\limits^{+\infty}_0 du \int\limits^1_0 dv u^{s - 1} v^{s - 1} u^{t - 1} (1 - v)^{t - 1} u e^{-u} = \int\limits^{+\infty}_0 du u^{s + t - 1} e^{-u} \int\limits^1_0 v^{s - 1}(1 - v)^{t - 1}dv = \Gamma(s + t) B(s, t)$.
       
    \newpage
    
    \section{Объем шара в $\mathbb R^m$}
    
        $\lambda_m B(0, R) = \int\limits_{x_1^2 + \ldots + x_m^2 = R^2} 1 dx$, введём сферические координаты.
        
        $\int\limits^R_0 dr \int\limits^{\pi}_0d\varphi_1 \ldots \int\limits^{\pi}_0 d \varphi_{m - 2} \int\limits^{2 \pi}_0 d \varphi_{m - 1} r^{m - 1} \sin^{m - 2} \varphi_1 \sin^{m - 3} \varphi_2 \ldots \sin \varphi_{m - 2}$, а дальше воспользуемся бетой-функцией.
        
        Пример как вычислять $\sin$ в какой-то степени:
        
        $\int\limits^{\pi}_0 \left(\sin \varphi_k \right)^{m - 1 - k} = 2 \int\limits^{\pi / 2}_0 t^{\frac{m - 1 - k}{2} - \frac{1}{2}} (1 - t)^{-0.5} dt = B \left( \frac{m - k}{2}, \frac{1}{2} \right) =\dfrac{\Gamma\left( \frac{m - k}{2} \right) \Gamma\left( \frac{1}{2} \right)}{\Gamma \left( \frac{m - k}{2} + \frac{1}{2} \right)}$.
   
\newpage

\part{Функция распределения}

\newpage

    \section{Теорема о вычислении интеграла по мере Бореля---Стилтьеса (с леммой)}
    
        \subsection{Определение}
        
            $\left( X, \mathcal{O}, \mu \right)$, $h : X \rightarrow \overline{\mathbb{R}}$~--- измеримая, пространство конечное.
    
            Пусть $\forall t \in \mathbb{R}$, $\mu X(h < t) < +\infty$.
    
            $H(t) := \mu X(h < t)$~--- функция распределения функции $h$ по $\mu$ $(H : \mathbb{R} \rightarrow \mathbb{R})$.
    
            Очевидно, что $H$ возрастает, $h : X \rightarrow \overline{\mathbb{R}}$, $\nu := h(\mu)$, $\nu(A) = \mu \left( h^{-1}(A) \right)$.
    
            Пусть $h$~--- измеримая, тогда $\forall B \in \mathcal{B} \left( \mathbb{R} \right)$, $h^{-1} \left( \mathcal{B} \right)$~--- измеримая.
    
            $\mu_H[a, b) = H(b - 0) - H(a - 0)$~--- \textit{мера Бореля-Стилтьеса}.
    
        \subsection{Лемма}
    
            $h : X \rightarrow \overline{\mathbb{R}}$~--- измеримая, почти везде конечная.
        
            $H$~--- функция распределения (корректно заданная), $\forall t$ $\mu X(h < t) < +\infty$.
        
            Тогда на $\mathcal{B}$, $\mu_H$ совпадает с $h(\mu)$.
        
        \subsubsection{Доказательство}
        
            $\mu_h[a, b) = H(b - 0) - H(a - 0) = H(b) - H(a)$~--- непрерывность меры снизу.
            
            $H(b) - H(a) = \mu X(a \leqslant h < b) = \mu \left( h^{-1} [a, b) \right) = \nu [a, b]$, где $\nu = h(\mu)$
            
            Значит $\mu_H = \nu$ на $\mathcal{B}$.
            
        \subsection{Теорема}
    
            $f : \mathbb{R} \rightarrow \mathbb{R}$, $\geqslant 0$, измеримое по Борелю.
        
            $h : X \rightarrow \overline{\mathbb{R}}$, измеримая, почти везде конечная, с функцией распределения $H$.
        
            $\mu_H$~--- мера Бореля-Стилтьеса. Тогда
        
            $\int\limits_X f \left( h(x) \right) d \mu(x) = \int\limits_{\mathbb{R}} f(t) d \mu_H (t)$.
        
        \subsubsection{Доказательство}
        
            По теореме о взвешенном образе меры:
            
            $(X, \mathcal{A}, \mu)$, $(Y = \mathbb{R}, \mathcal{B}, h(\mu))$,
            
            $\Phi = h : X \rightarrow Y$, $\omega = 1$.
            
            $\int\limits_Y f(y) d \nu = \int\limits_X f(\Phi(x)) 1 d \mu(x)$.
            
        Путь $f \geqslant 0$, измеримая, $\mathbb{R} \rightarrow \mathbb{R}$.
    
        $\int\limits_{\mathbb{R}^m} f \left( | x | \right) d \lambda_m = \int\limits_0^{+\infty} f(t) d \mu_H$ при $h(x) = |x|$, где $H(r) = \mu \mathbb{R}^m \left( |x| < r \right) = \alpha_m r^m$.
    
        $\mu_H [a, b) = H(b) - H(a) = \int\limits^b_a H'(t) dt = \int\limits^b_a m \alpha_m t^{m - 1} dt$.
    
        $\mu_H$ и мера $\nu : \nu(A) = \int\limits_A m \alpha_m t^{m - 1} dt$, значит $\mu_h = \nu$ на $\mathcal{B}$. 
    
        $\int\limits^{+\infty}_0 f(t) m \alpha_m t^{m - 1} dt$.
    
            \subsubsection{Следствие}
        
                Мы проверили, что $g$ возрастает, $g \in C^1 \left( \mathbb{R} \right)$ и $M_g(A) = \int\limits_A g'(x) dx$.
            
\newpage

\part{Ряды Фурье}

\newpage

    \section{Интегральные неравенства Гельдера и Минковского}
    
        \begin{enumerate}
        
            \item Неравенство Гёльдера:
        
                $p$, $q > 1$, $\dfrac{1}{p} + \dfrac{1}{q} = 1$, заданы почти везде, измеримы.
            
                $(X, \mathcal{A}, \mu)$, $f$, $g : X \rightarrow \mathbb{C}$ ($\mathbb{R}$). Тогда
            
                $\int\limits_X |fg| d \mu \leqslant \left( \int\limits_X |f|^p \right)^{1/p} \left( \int\limits_X |g|^q \right)^{1/q}$
            
            \item Неравенство Минковского
        
                $(X, \mathcal{A}, \mu)$, $f$, $g : X \rightarrow \mathbb{C}$~--- измерима почти везде, конечна, $1 \leqslant p < +\infty$. Тогда
            
                $\left( \int\limits_X |f + g|^p \right)^{1/p} \leqslant \left( \int\limits_X |f|^p \right)^{1/p} + \left( \int\limits_X |g|^p \right)^{1/p}$
            
        \end{enumerate}
        
    \newpage

    \section{Интеграл комплекснозначной функции}
    
        $(X, \mathcal{A}, \mu)$, $f : X \rightarrow \mathbb{C}$, $f(x) = g(x) + i h(x)$.
        
        $f$~--- измерима $\Longleftrightarrow g = \mathrm{Re} f$ и $h = \mathrm{Im} f$~--- измеримые.
            
        $f$~--- суммируемая $\Longleftrightarrow g = \mathrm{Re} f$ и $h = \mathrm{Im} f$~--- суммируемые.
            
        $\int\limits_X f = \int\limits_X g + i \int\limits_X h$.
            
        \subsection{Вывод}
        
            $\left| \int\limits_X f d \mu \right| \leqslant \int\limits_X |f| d\mu$.
            
    \newpage
    
    \section{Пространство $L^p(E,\mu)$}
    
        $L^p(X, \mu)$, $1 \leqslant p < \infty$
        
        $\mathcal{L}^p (X, \mu) = \left\{ f : X \xrightarrow[\text{п.в.}]{} \overline{\mathbb{R}} (\overline{\mathbb{C}}), f \text{~--- измерима}, \int\limits_X |f|^p d \mu < +\infty \right\}$
        
        \begin{itemize}
            
            \item $\mathcal{L}^p (X, \mu)$~--- линейное пространство~--- по н. Минковского;
            
            \item Введём норму $\| f \| = \left( \int\limits_X |f|^p \right)^{1/p}$;
            
            \item $f$ эквивалентно $g$ если $f(x) = g(x)$ при почти всех $x$
                
        \end{itemize}
        
        $L^p$~--- уберём из $\mathcal{L}$ все одинаковые функции, оставив только одного представителя из каждого класса эквивалентности.
            
    \newpage
    
    \section{Существенный супремум}
    
    $f : X \xrightarrow[\text{п.в.}]{} \overline{\mathbb{R}}$, $\mathrm{ess}\sup f = \inf \left\{ A \in \overline{\mathbb{R}} : f(x) \leqslant A \ \text{п.в.} \right\}$.
    
        \subsection{Свойства}
    
            \begin{enumerate}
        
                \item $\mathrm{ess} \sup f \leqslant \sup f$;
            
                \item $f(x) \leqslant \mathrm{ess} \sup f$ при почти всех $x$;
            
                \item $\left| \int\limits_\mathbb{R} fg \right| \leqslant \mathrm{ess} \sup |f| \cdot \int\limits_X |g|$.
            
            \end{enumerate}
        
            \subsubsection{Доказательство}
    
                \begin{enumerate}
        
                    \item Очевидно
            
                    \item $M = \mathrm{ess}\sup f$
            
                        $\forall n \in \mathbb{N}$ верно $f(x) \leqslant M + \frac{1}{n}$ почти везде.
            
                    \item Очевидно $\left| \int\limits_X fg \right| \leqslant \int\limits_X |fg|$,
            
                        $|fg| \leqslant M|g|$ почти везде.
            
                \end{enumerate}
        
    \newpage
    
    \section{Пространство $L^\infty(E,\mu)$}
        
        $\mathcal{L}^{\infty} (X, \mu) = \left\{ f : X \xrightarrow[\text{п.в.}]{} \mathbb{R} (\mathbb{C}), f \text{~--- измерима}, \mathrm{ess}\sup |f| < +\infty \right\}$
    
        $f$, $g \in \mathcal{L}^{\infty} \Rightarrow f + g \in \mathcal{L}^{\infty}$.
    
        т.е. $\mathcal{L}^{\infty}$~--- линейное пространство, норма $\| f \|_{\infty} = \mathrm{ess}\sup\limits_x |f|$.
    
        $\mathrm{ess}\sup |f + g| \leqslant \mathrm{ess}\sup |f| + \mathrm{ess}\sup |g|$.
    
        \subsection{Замечание}
    
            $\| fg \|_1 \leqslant \| f \|_p \| g \|_q$~--- неравенство Гёльдера (можно брать $p = 1$ и $q = +\infty$).
        
            $f \in \mathcal{L}^p (X, \mu)$, $1 \leqslant p \leqslant +\infty$, $\Rightarrow f$~--- почти всюду конечно $\Rightarrow$ можно считать, что $f$ задана почти всюду на $X$ и всюду конечна.
    
    \newpage
    
    \section{Теорема о вложении пространств $L^p$}
    
        $X$, $\mu X < +\infty$, $1 \leqslant s < r \leqslant +\infty$. Тогда 
        
        \begin{enumerate}
            
            \item $L^r(X, \mu) \subset L^s(x, \mu)$;
            
            \item $\| f \|_s \leqslant (\mu X)^{\frac{1}{s} - \frac{1}{r}} \| f \|_r$
            
        \end{enumerate}
        
        \subsection{Доказательство}
        
            \begin{enumerate}
            
                \item следует из $2$;
                
                \item $r = \infty$~--- очевидно
                
                    $r$~--- конечно, тогда:
                
                    $\| f \|_s = \left( \int\limits_X |f|^s \right)^{\frac{1}{s}} \leqslant \left( \int\limits_X \| f ||^s_{\infty} \right)^{\frac{1}{s}}$
                    
                    $|f| \leqslant \mathrm{ess}\sup f = \| f \|_{\infty} = \| f \|_{\infty} \mu X^{1/s}$
                    
                    $\| f \|^s_s = \int\limits_X |f|^s 1 d \mu$ по Гёльдеру получаем неравенство
                    
                    $\left( \int\limits_X \left( |f|^s \right)^{r/s} \right)^{s/r} \left( \int\limits_X 1 \right)^{\frac{r - s}{r}} = \left( \int\limits_x |f|^r \right)^{s/r} \left( \mu X \right)^{1 - \frac{s}{r}}$.
                
            \end{enumerate}
            
        \subsection{Следствие}
        
            $\mu E < +\infty$, $1 \geqslant s < r \geqslant +\infty$.
            
            $f_n$, $f \in L^s$, $f_n \rightarrow f$ на $L^r$. Тогда $f_n \rightarrow f$ на $L^s$.
            
        \subsubsection{Доказательство}
        
            очевидно, потому что $\| f \|_s \leqslant \mu E^{\frac{1}{s} - \frac{1}{r}} \| f \|_r$.

\newpage

\part{Поверхностный интеграл}

    \section{Измеримое множество на простом гладком двумерном многообразии в $\mathbb{R}^3$}
    
        $M$~--- просто гладкое двумерное многообразие в $\mathbb{R}^3$, $\varphi : \underset{\text{откр.}}{O} \subset \mathbb{R}^2 \rightarrow \mathbb{R}^3$~--- параметризация.
        
        $E \subset M$~--- измеримое (по Лебегу), если его $\varphi^{-1}(E)$~--- измерим в $\mathbb{R}^2$.
        
    \newpage
    
    \section{Мера Лебега на простом гладком двумерном многообразии в $\mathbb{R}^3$}
    
        $\mathcal{A}_M = \left\{ E \subset M, E \text{~--- изм.} \right\}$~--- $\sigma$-алгебра.
        
        Мера Лебега на $\mathcal{A}_M$: $S(E) = \iint\limits_{\varphi^{-1}(E)} \left| \varphi'_u \times \varphi'_v \right| du dv$.
             
    \newpage
        
    \section{Поверхностный интеграл первого рода}
        
        $M$~--- простое двумерное гладкое многообразие, $\varphi$~--- гладкая параметризация, $f : M \rightarrow \overline{\mathbb{R}}$, $f \geqslant 0$, измеримая. Тогда
        
        $\iint\limits_M f ds$~--- Поверхностный интеграл $\mathrm{I}$ рода и вычисляется следующим образом:
        
        $\iint\limits_M f ds = \iint\limits_{\varphi^{-1} M} f(x(u, v), y(u, v), z(u, v)) | \varphi'_u \times \varphi'_v | dv du$.
        
    \newpage
    
    \section{Кусочно-гладкая поверхность в $\mathbb{R}^3$}
    
        $M \subset \mathbb{R}^3$~--- кусочно-гладкое многообразие в $\mathbb{R}^3$
        
        $M$~--- объекты конечного числа элементов:
        
        \begin{itemize}
        
            \item Простые двумерные гладкие многообразия;
            
            \item Гладкие кривые~--- простые $k$-мерные многообразия в $\mathbb{R}^3$;
            
            \item Точки.
            
        \end{itemize}
        
        $M = \bigsqcup M_i \bigsqcup l_i \bigsqcup p_i$.
        
        $S(E) = \sum S(E \bigcap M_i)$.
        
\newpage

\part{Преобразование Фурье}

    \section{Теорема о сходимости в пространствах $L^p$ и по мере}
    
        $1 \leqslant p < +\infty$, $f_n$, $f \in L^p(X, \mu)$. Тогда верны следующие утверждения:
        
        \begin{enumerate}
        
            \item $f_n \rightarrow f$ в $L^p$, тогда $f_n \rightrightarrows f$ по мере $\mu$.
            
            \item $f_n \rightrightarrows f$ по мере $\mu$ (либо $f_n \rightarrow f$ почти везде).
                
                Если $\exists g \in L^p : |f_n| \leqslant g$. Тогда $f_n \rightarrow f$ в $L^p$.
            
        \end{enumerate}
        
        \subsection{Доказательство}
            
            \begin{enumerate}
            
                \item $X_n(\varepsilon) := X \left( |f_n - f| \geqslant \varepsilon \right)$.
            
                    $\mu X_n(\varepsilon) = \int\limits_{X_n(\varepsilon)} 1 \leqslant \dfrac{1}{\varepsilon^p} \int\limits_{X_n(\varepsilon)} |f_n - f|^p d \mu \leqslant \dfrac{1}{\varepsilon^p} \| f_n - f \|^p_p \rightarrow 0$.
            
                \item $f_n \rightrightarrows f$, тогда $f_{n_k} \rightarrow f$ п.в.. Тогда $|f| \leqslant g$ п.в. $|f_n - f|^p \leqslant (2g)^p$, $\| f_n - f \|^p_p = \int\limits_X |f_n - f|^p d\mu \rightarrow 0$ по теореме Лебега.
                
            \end{enumerate}
    
    \newpage
    
    \section{Полнота $L^p$}
    
        $L^P(X, \mu)$, $1 \leqslant p < +\infty$~--- полное.
        
        \subsubsection{Доказательство}
        
            $f_n$~--- фундаментальная. 
            
            Для $\varepsilon = \frac{1}{2}$ $\exists N_1$ при $n = n_1 > N_1$, $\forall k > n_1$ $\| f_{n_1} - f_k \| < \frac{1}{2}$.
            
            Для $\varepsilon = \frac{1}{4}$ $\exists N_2 > n1$ при $n = n_2 > N_2$, $\forall k > n_2$ $\| f_{n_2} - f_k \| < \frac{1}{4}$.
            
            $\varepsilon = \dfrac{1}{2^m}$ $\exists N_m > n_m$ при $n = n_m > N_m$, $\forall k > n_m$ $\| f_{n_m} - f_k \| < \dfrac{1}{2^m}$.
            
            Таким образом, $\sum\limits_{k = 1}^{+\infty} \| f_{n_{k + 1}} - f_{n_k} \|_p < 1$.
            
            Рассмотрим $S(x) = \sum\limits_{k = 1}^{+\infty} | f_{n_{k + 1}}(x) - f_{n_k}(x) | \in [0, +\infty]$.
            
            $S_n$, $\| S_n \|_p \leqslant \sum \| f_{n_{k + 1}} - f_{n_k} \|$
            
            $S_n$, $\| S_n \|_p \leqslant \sum\limits^N_{k = 1} \| f_{n_{k + 1}} - f_{n_k} \|_p < 1$.
            
            $\int\limits_X S_n^p \leqslant 1$, по т. Фату $\int S^p \leqslant 1$, тогда $S^p$~--- сходится, значит $S$ конечно почти везде, тогда
            
            $\sum \left( f_{n_{k + 1}}  f_{n_k} \right)$~--- сходится почти везде.
            
            $f(x) := f_{n_1} + \sum\limits_{k = 1}^{+\infty} \left( f_{n_{k + 1}} - f_{n_k} \right)$~--- сходится с потрам
            
            $f_{n_1} + \sum\limits_{k = 1}^{m - 1} \left( f_{n_{k + 1}} - f_{n_k} \right) = f_{n_m}$.
            
            $f_{n_m} \rightarrow f$ почти везде.
            
            Проверим, что $\| f_n - f \|_p \rightarrow 0$.
            
            $\forall \varepsilon > 0$ $\exists N$ $\forall m, n > N$ $\| f_n - f_m \|_p < \varepsilon$
            
            $\| f_n - f_{n_k} \|^p_p = \int\limits_X |f_n - f_{n_k}|^p d\mu < \varepsilon^p$ верно при всех больших $k$.
            
            Тогда по теорему Фату: $\int\limits_X |f_n - f|^p d\mu < \varepsilon^P$, т.е. $\| f_n - f \|_p < \varepsilon$, т.е. $f_n \rightarrow f$ в $L^p$.
    
    \newpage
    
    \section{Плотность в $L^p$ множества ступенчатых функций}
    
        \subsection{Определение}
        
            $Y$~--- множество, $\mathcal{A} \subset Y$~--- (всюду) плотное множество, если $\forall y \in Y : \forall U(y)$ верно $U(y) \bigcap \mathcal{A} \neq \varnothing$.
        
            Пример: $\mathcal{A} = \mathbb{Q} \subset Y = \mathbb{R}$.
        
        \subsection{Лемма} 
    
            $(X, \mathcal{A}, \mu)$, $1 \leqslant p \leqslant +\infty$
        
            Тогда
        
            $\left\{ f \in L^p : f \text{~--- ступ.} \right\}$~--- плотно $L^p$.
        
            \subsubsection{Замечание}
    
                $p < +\infty$, $\varphi \in L^p$~--- ступенчатая, тогда $\mu X(\varphi \neq 0) < +\infty$.
        
            \subsubsection{Доказательство}
        
                \begin{enumerate}
            
                    \item $p = +\infty$, $f \in L^{\infty}$, подменим $f$ на множество меры $0 : |f| \leqslant \| f \|_{\infty}$ всюду.
                
                        $\exists$ ступ. $\varphi_n \rightrightarrows f_+$, $\psi_m \rightrightarrows f_-$, т.е. $\| \varphi_n - f_+ \|_{\infty} \rightarrow 0$, $\varphi_n \rightarrow f_+$ в $C^{\infty}$, $\psi_m \rightarrow f_-$.
                
                    \item $p < +\infty$, $f \geqslant 0$, $\exists \varphi_n$~--- ступенчатая, $\varphi_n \rightarrow f$ всюду.
                
                        $\| \varphi_n - f \|^p_p = \int\limits_X |\varphi_n - f|^p d \mu \rightarrow 0$, $| \varphi - f|^p \leqslant |f|^p$.
                
                \end{enumerate}
                
    \newpage
            
    \subsection{Определение}
        
        $X$~--- \textit{топологическое} пространство, если $\forall F_1$, $F_2$~--- замкнутых подмножеств, $F_1 \bigcap F_2 = \varnothing$.
            
        Если $\exists$ открытые $U(F_1)$, $U(F_2)$, которые не пересекаются, то это свойство $X$ называются \textit{нормальностью}. (дополнительно требуется, чтобы $\forall y \in X$ $\left\{ y \right\}$~--- замкнутое).
            
    \subsection{Лемма Урысона}
        
        Будет дописано позже.
        
        $X$~--- норм, $F_0$, $F_1$~--- замкнуты, $F_0 \bigcap F_1 = \varnothing$.
        
        Тогда $\exists f : X \rightarrow \mathbb{R}$, $0 \leqslant f \leqslant 1$, непрерывное.
        
        $f\big|_{F_0} = 0$, $f\big|_{F_1} = 1$.
        
        \subsection{Доказательство}
        
            Переформулируем нормальность:
            
            $\forall F_1$~--- замкнутого, $\subset G$~--- открытого, $\exists U(F_1)$~--- открытое, что выполняется $F_1 \subset U(F_1) \subset \overline{U(F_1)} \subset G$.
            
            \begin{enumerate}
            
                \item $F_0 \subset U(F_0) \subset \overline{U(F_1)} \subset F_1^C$
                
                \item $\overline{G_0} \subset U(\overline{G_0}) \subset G_1$
                
                \item $\overline{G_0} \subset U'(\overline{G}_0) \subset \overline{U'(\overline{G_0})} \subset G_{1/2}$
                
                $G_{1/2} \subset U(\overline{G}_{1/2}) \subset \overline{U} \subset G_1$, где $U(\overline{G}_{1/2}) = G_{3/4}$.
                
            \end{enumerate}
            
            $f$~--- непрерывна, значит $f^{-1}(a, b)$~--- открыто. Достаточно проверить, что:
            
            \begin{enumerate}
            
                \item $f^{-1}(-\infty, s)$~--- открыто;
                
                \item $f^{-1}(-\infty, s)$~--- замкнуто.
                
            \end{enumerate}
            
            $f^{-1}(a, b) = f^{-1}(-\infty, b) \setminus f^{-1}(-\infty, a)$.
            
            \begin{enumerate}
            
                \item $\forall s : f^{-1} (-\infty, s) = \bigcup\limits_{q \in s, q\text{-дв. рац.}} G_q$~--- открыто.
                
                    $\subset$ $f(y) < S$, где $f(y) = \inf \left\{ q : x \in G_q \right\}$.
                    
                    $\supset$ $x \in \text{ЛЧ}$, $f(x) = S_0 < q_1 < S$, $x \in G_{q_1}$.
                
            \end{enumerate}
            
\newpage

\part{Поверхностный интеграл II рода}

    \newpage

    \section{Финитная функция}
    
        \textit{Финитная функция}~--- функция, равная $\mathbf{0}$ вне некоторого шара, и непрерывная в $C_0 \left( \mathbb{R}^m \right)$.
        
        Очевидно, что $\forall p \in [1, +\infty) : C_0 \left( \mathbb{R}^m \right) \subset L^p \left( \mathbb{R}^m, \lambda_m \right)$.
     
    \newpage

    \section{Сторона поверхности}
    
        Поверхность~--- простое гладкое двумерное многообразие.
    
        Сторона поверхности (гладкой)~--- непрерывное векторное поле единичных нормалей.
        
        Если не существует непрерывного поля единичных нормалей, то такая поверхность~--- односторонняя.
        
    \newpage
        
    \section{Задание стороны поверхности с помощью касательных реперов}
    
        \textit{Репер}~--- Пара ЛНЗ касательных векторов.
        
        Способ задания стороны~--- задать поле касательных реперов.
        
    \newpage
    
    \section{Интеграл II рода}
    
        $\Omega$~--- двусторонняя поверхность в $\mathbb{R}^3$, $F : \Omega \rightarrow \mathbb{R}^3$.
        
        $n_0$~--- сторона поверхности.
        
        Тогда интегралом $\mathrm{II}$ рода (поля $F$ на $\Omega$) называют:
        
        $\int\limits_{\Omega} \langle F, n_0 \rangle ds$.
        
        \subsubsection{Замечания}
        
            \begin{enumerate}
        
                \item поменяем сторону~--- поменяем знак;
                
                \item Не зависит от параметризации;
                
                \item Обозначения: $F = (P, Q, R)$
                
                    $\int\limits_{\Omega} = \int\limits_{\Omega} P dy dz + Q dz dx + R dx dy$.
                
                    $x(u, v)$, $y(u, v)$, $z(u, v)$, тогда
                    
                    $(x'_u, y'_u, z'_u) \times (x'_v, y'_v, z'_v) = \vec{n}$
                
                    $dy dz = (y'_u du + y'_v dv) \wedge (z'_u du + z'_v dv) = du \wedge dv (y'_u z'_v - y'_v z'_u)$
                    
                    $\wedge$~--- косо-коммутативная операция
                    
                    $da \wedge db = -db \wedge da$
                    
                    $da \wedge da = -da \wedge da = 0$.
                    
            \end{enumerate}
     
    \newpage
    
    \section{Плотность в $L^p$ множества финитных непрерывных функций}
    
        $\left( \mathbb{R}^m, \mathcal{M}^m, \lambda_m \right)$, $E \subset \mathbb{R}^m$~--- измеримая.
        
        Тогда множество финитных функций (непрерывных) плотно в $L^p \left( E, \lambda_m \right)$
            
        \subsection{Доказательство}
        
            $g \in L^p(E, \mu)$
            
            $\forall \varepsilon > 0 : \exists f \in C_0 \left( \mathbb{R}^m \right)$, $\| g - f \big|_E \|_p < \varepsilon$. Пусть $g = 0$ вне $E$, то $\| g - f \big|E \|_{2^p(E, \mu)} \leqslant \| g - f \| < \varepsilon$ в $L^p \left( \mathbb{R}^m \right)$.
            
            $g = g^+ - g^-$, $g^+$~--- приблизим ступенчатыми, $\exists$ ступ. $h : \| g^+ - h \| < \varepsilon$.
            
            $h = \sum c_k \chi_{a_k}$. Каждую $\chi_{A_k}$ приблизим финитной непрерывной функцией:
            
            $\forall \varepsilon > 0 : \exists$ замкнутая $F_k \subset A_k \subset G_k$(откр.), $\lambda_m \left( G_k \setminus F_k \right) < \left(\dfrac{\varepsilon}{|c_k|\cdot q}^p \right)$.
            
            По лемме Урысона $\exists f_k : 0 \leqslant f_k \leqslant 1$, $f = 1$ на $F_k$, $f = 0$ на $\mathbb{R}^m \setminus G_k$.
            
            $\| g^+ - \sum c_k f_k \|_p \leqslant \| g^+ - h \|_p + \| h - \sum c_k f_k \| \leqslant \varepsilon + \sum |c_k| \cdot \| \chi_{A_k} - f_k \| \leqslant \int \left| \chi_{A_k} - f_k \right|^p \leqslant \varepsilon + \sum \dfrac{\varepsilon}{q} = 2 \varepsilon$
            
            $\int\limits_{G_k \setminus F_k} 1^p < \left( \dfrac{\varepsilon}{|c_k|q} \right)^p$.
            
            $1 \leqslant p < +\infty$.
            
        \subsection{Замечание}
        
            \begin{enumerate}
            
                \item В $L^{\infty} \left( \mathbb{R}^m \right)$ этот факт не работает.
                
                    $L^{\infty} \left( [0, 2] \right)$ функцию $\chi_{[0, 1]}$ не приблизить непрерывной.
                    
                \item В $L^p \left( E, \lambda_m \right)$ плотны:
                
                    \begin{itemize}
                    
                        \item Линейная комбинация характеристических функций ячеек;
                
                        \item Гладкие финитные функции;
                        
                        \item Рациональные линейные комбинации рациональных ячеек;
                        
                        \item Просто непрерывные функции.
                        
                    \end{itemize}
                    
            \end{enumerate}
    
    \newpage
    
    \section{Теорема о непрерывности сдвига}
    
        \subsection{Необходимое определение}
        
            $L^p [0, T]$, $T \in \mathbb{R}$, можем понимать как пространство $T$-периодических функций $\left( \mathbb{R} \rightarrow \mathbb{R} \right)$, $\int\limits^T_0 f = \int\limits^{a + T}_a f$.
    
            $C [0, T]$~--- пространство непрерывных функций, $\| f \| = \max\limits_{x \in [0, T]} | f(x) |$.
    
            $\widetilde{C} [0, T]$~--- пространство непрерывных $T$-пер. функций.
        
            $f \in \widetilde{C} [0, T] \Rightarrow f$~--- равномерно непрерывные.
    
            $\widetilde{C} [0, T]$ плотно в $L^P [0, T]$, $p < +\infty$.
    
        \subsection{Формулировка теоремы}
        
            $f_h(x) := f(x + h)$.
        
            \begin{enumerate}
        
                \item $f$~--- равномерно непрерывная на $\mathbb{R}^m \Rightarrow \| f_h - f \| \rightarrow 0$ при $n \Rightarrow 0$;
            
                \item $1 \leqslant p < +\infty$, $f \in L^p \left( \mathbb{R}^m \right) \Rightarrow \| f_n - f \|_p \rightarrow 0$ при $n \rightarrow 0$;
            
                \item $f \in \widetilde{C} [0, T] \Rightarrow \| f_n \ f \|_{+\infty} \rightarrow 0$;
            
                \item $1 \leqslant p < +\infty$ $f \in L^p [0, T] \Rightarrow \| f_n - f \|_p \rightarrow 0$.
            
            \end{enumerate}
        
        \subsection{Доказательство}
        
            $1$ и $3$ очевидные утверждения по определению равномерной непрерывности.
            
            $\forall \varepsilon > 0 : \exists \delta : \forall x, x' : |x - x'| < \delta$ $| f(x) - f(x')| < \varepsilon$
            
            $\forall |h| < \delta : \| f_h - f \|_{\infty} \leqslant \varepsilon$.
        
            $g$~--- финитно непрерывная: $\| f - g \|_p < \dfrac{\varepsilon}{3}$.
            
            $\| f_h - f \|_p \leqslant \| f_h - g_h \|_p + \| g_h - g \|_p + \| g - f \|_p \leqslant \dfrac{2 \varepsilon}{3} + \| g_h - g \|_p$
            
            $g = 0$ вне $B(0, r)$, пусть $|h| < 1$, тогда $\| g_h - g \|_p = \| g_h - g\|_{L^p \left( B(0, r + 1) \right)} \leqslant \| g_h - g \|_{+\infty} \cdot \lambda B^{1/p}$
            
            и 4) $\| g_h - g \|_p \leqslant \| g_h - g \|_{\infty} T^{1/p}$
            
\newpage

    \section{Формула Грина}
    
        $D$~--- компактное, связное, односвязное, множество в $\mathbb{R}^2$, ограниченное кусочно-гладкой кривой.
        
        На $\partial D$ направление ''против часовой стрелки''.
        
        \subsection{Теорема}
    
            $D \subset \mathbb{R}^2$~--- см выше.
        
            $P$, $Q$~--- векторные поля, гладкие в $U(D)$. Тогда
        
            $\iint\limits_D \left( \dfrac{\partial Q}{\partial x} - \dfrac{\partial P}{\partial y} \right) dx dy = \int\limits_{\partial D} P dx + Q dy$.
        
        \subsection{Доказательство}
        
            $D$~--- кривая $4$-угольника относительно $OX$, а также относительно $OY$.
            
            Рассмотрим поле $(P, \mathbf{0})$ (для $(\mathbf{0}, Q)$ аналогично).
            
            ПЧ: $-\iint\limits_{D_b} \dfrac{\partial P}{\partial y} dx dy = \int\limits_{\partial D} Pdx + \mathbf{0} dy$
            
            ЛЧ: $-\int\limits^b_a d \int\limits^{f_2(x)}_{f_1(x)} \dfrac{\partial P}{\partial y} dy = - \int\limits^b_a P(x, y) \bigg|^{y = f_2}_{y = f_1} dx = \int\limits^b_a P(x, f_1(x))dx - \int\limits^b_a P(x, f_2(x)) dx$.
            
            ПЧ: $\int\limits_{\gamma_1} + \int\limits_{\gamma_2} + \int\limits_{\gamma_3} + \int\limits_{\gamma_4}$, $\int\limits_{\gamma_1} = \int\limits^b_a P(x, f(x)) \cdot 1 + 0 \cdot f'(x) dx$, $\int\limits_{\gamma_2} = \int\limits_{\gamma_4} = 0$, $\int\limits_{\gamma_3} = \int\limits^a_b P(x, f(x)) dx$.
            
    \newpage
    
    \section{Формула Стокса}
    
        $\Omega$~--- двусторонняя, гладкая поверхность, $\overline{n_0}$~--- сторона.
        
        $\partial \Omega$~--- кусочно-гладкая кривая с согласованной ориентацией.
        
        $(P, Q, R)$~--- гладкое векторное поле в $U(\Omega)$. Тогда 
        
        $\int\limits_{\partial \Omega} P dx + Q dy + R dz = \iint\limits_{\Omega} (R'_y - Q'_z) dy dz + (P'_z - R'_x) dzdx + (Q'_x - P'_y) dxdy$.
        
        \subsection{Доказательство}
        
            Считаем, что поверхность $C^r$-гладкая.
            
            Достаточно проверить для $(P, 0, 0)$.
            
            $\int\limits_{\partial \Omega} P dx = \iint P'_z dzdx - P'_y dxdy$.
            
            $\int\limits_{\partial \Omega} P dx = \int P(x(u, v), y(u, v), z(u, v)) \left( \dfrac{\partial x}{\partial u} du + \dfrac{\partial x}{\partial v} dv \right)$ и по формуле Грина получаем
            
            $\int\limits_L P x'_u du + P x'_v dv = \iint\limits_G \dfrac{\partial}{\partial u} (P x'_v) - \dfrac{\partial}{\partial v} (P x'_u) du dv = \iint \left( P'_x x'_u + P'_y y'_u + P'_z z'_u \right) x'_v + P x''_vv - \left(P'_x x'_v + P'_y y'_v + P'_y z'_v \right) x'_u - P x''_uv du dv = \iint P'_x \mathbf{0} + P'_y (x'_v y'_u - x'_u y'_v) + P'_z (x'_v z'_u - x'_u z'_v) = \iint\limits_{G} P'_z dzdx - P'_y dxdy$
            
            Получили что хотели.
            
    \newpage
    
    \section{Формула Гаусса-Остроградского}
    
        $V = \left\{ (x, y, z) : (x, y) \in \Omega \text{ и } f(x, y) \leqslant z \leqslant F(x, y) \right\}$
        
        $\Omega\underset{\text{замкн.}} \subset \mathbb{R}^2$, $\partial \Omega$~--- кусочно-гладкая кривая, $f$, $F \in C^1 \left( \Omega \right)$.
        
        $R : U(V) \rightarrow \mathbb{R}$, $R \in C^1$. Тогда
        
        $\iiint_V \dfrac{\partial R}{\partial z} dx dy dz = \iint\limits_{\partial V^+} R dx dy$.
        
        \subsection{Доказательство}
        
            $\iiint\limits_V \dfrac{\partial R}{\partial z} dx dy dz = \iint\limits_{\Omega} dx dy \int\limits_{f(x, y)}^{F(x, y)} \dfrac{\partial R}{\partial z} dz = \iint\limits_{\Omega} R(x, y, F(x, y)) - \iint\limits_{\Omega} R(x, y f(x, y)) dx dy = \iint\limits_{\text{график F (верх)}} R dx dy + \iint\limits_{\text{график f (низ)}} R dx dy$.
            
            $0 = \iint\limits_{\text{цил. } \partial V} R dx dy$.
        
        \subsection{Следствие}
        
            $\iint\limits_V \left( \dfrac{\partial P}{\partial x} + \dfrac{\partial Q}{\partial y} + \dfrac{\partial R}{\partial z} \right) dx dy dz = \iint\limits_{\partial V^+} P dydz + Q dzdx + R dxdy$.
    
    \newpage
    
    \section{Соленоидальность бездивергентного векторного поля}
    
        \subsection{Дивергенция}
    
            $\mathrm{div} A$~--- это функция $\dfrac{\partial P}{\partial x} + \dfrac{\partial Q}{\partial y} + \dfrac{\partial R}{\partial z}$.
        
            $\mathrm{div} A(a) = \lim\limits_{r \rightarrow 0} \dfrac{1}{\lambda_3 B} \iiint\limits_{B(a, r)} \mathrm{div} A dx dy dz = \lim\limits_{r \rightarrow 0} \dfrac{1}{\lambda_3 B} \iint\limits_{S(a, r)} \langle (P, Q, R), n_0 \rangle dS$.
            
        \subsection{Ротор}
    
            $(P, Q, R) \in C^1$~--- ротор (вихрь).
        
            $\mathrm{rot} A = (R'_y - Q'_z, P'_z - R'_x, Q'_x - P'_y)$.
    
        $\mathrm{rot} V = 0$, $\gamma = 0$. Тогда $\int\limits_{\gamma} P dx + Q dy + R dz = 0 $.
        
        $\gamma$~--- путь от $A$ до $B$. Тогда $\int\limits_{\gamma}$~--- зависит от $A$ и $B$, но не от самого пути.
        
        Если $O \subset \mathbb{R}^3$~--- не односвязная, $\mathrm{rot} V = 0$. $I(v, \gamma)$ не зависит от $\gamma$
    
        $\int\limits_{\Omega} \mathrm{rot} V = \int\limits_{\gamma_2} V + \int\limits_{\gamma_1} V$.
    
        $\mathrm{div} (P, Q, R) = 0$.
    
        $\forall V : \iint\limits_{\partial V} \langle (P, Q, R), n_0 \rangle dS = 0$.
    
        \subsection{Вспомогательная теорема}
    
            $V$~--- поле. Если $\mathrm{rot} V = 0$ и область односвязная, то поле гладкое.
    
            $\mathrm{rot} = 0$~--- дифференциальный критерий потенциальности $\Leftrightarrow$ поле локально-потенциальное $\Leftrightarrow V$~--- потенциальное.
        
        \subsection{Соленоидальное поле}
    
            Поле $V$~--- \textit{соленоидальное} в $\Omega$ если существует векторный потенциал, т.е. существует такое векторное поле $B$, что $\mathrm{rot} B = V$.
        
        \subsection{Теорема}
    
            $\Omega$~--- параллелепипед, $(A_1, A_2, A_3) = A$~--- соленоид в $\Omega \Leftrightarrow \mathrm{div} A = 0$ в $\Omega$.
    
            \subsubsection{Доказательство}
        
                $\Rightarrow$ Тривиально $\mathrm{div} \mathrm{rot} B = 0$~--- упражнение.
            
                $\Leftarrow$. 
            
                    $\mathrm{div} A = 0$. Ищем векторный потенциал $B : \mathrm{rot} B = A$.
            
                    $B = (P, Q, R)$, $R'_y - Q'_z = A_1$, $P'_z - R'_x = A_2$, $Q'_z - P'_y = A_3$.
                
                    Забавный факт: можем подменить $B$ на $B_1$, что $\mathrm{B - B_1} = 0$ и $B - b_1$~--- потенциал $f$.
                
                    Пусть $R = 0$, тогда $-Q'_z = A_1$, $P'_z = A_2$ и $Q'_x - P'_y = A_3$. $P(x, y, z) = \int\limits^z_{z_0} A_2 (x, y, z) dt$
                
                    $Q(x, y, z) = - \int\limits^z_{z_0} A_1 dz + \varphi(x, y)$.
                
                    $I(y) = \int\limits^b_a f(x, y) dx$, $I'_(y)= \int\limits^b_a f'_y (x, y) dx$.
                
                    $\varphi'_x -\int\limits^z_{z_0} \dfrac{\partial A_1}{\partial x} - \int\limits^z_{z_0} \dfrac{\partial A_2}{\partial y} = A_3$.
                
                    $\mathrm{div} = 0$ по условию, тогда $\varphi'_x + \int\limits^z_{z_0} \dfrac{\partial A_3}{\partial z} dz = A_3$.
                    
                    $\varphi'_x(x, y) + A_3(x, y, z) - A_3(x, y, z_0) = A_3$.
                
                    $\varphi'_x(x, y) = A_3(x, y, z_0)$.
                
                    $\varphi = \int\limits^x_{x_0} A_3(x, y, z_0) dx + g(y)$.
                
\newpage
    
\part{Гильбертовы пространства}

\newpage
    
    \section{Гильбертово пространство}
    
        \textit{Гильбертово пространство} $\mathcal{H}$~--- линейное пространство со скалярным произведением (и соответствующей нормой), полное (как линейное нормированное пространство).
        
    \newpage
    
    \section{Теорема о свойствах сходимости в Гильбертовом пространстве}
    
        Пусть $x, y$ лежат в Гильбертовом пространстве. Тогда верны следующие свойства:
        
        \begin{enumerate}
        
            \item $x_n \rightarrow x_0$, $y_n \rightarrow y_0$. Тогда $\langle x_n, y_n \rangle \rightarrow \langle x_0, y_0 \rangle$.
            
            \item $\sum x_k$~--- сходится. Тогда $\forall y \in \mathcal{H} : \langle \sum\limits_{k = 1}^{+\infty} x_k, y \rangle = \sum\limits_{k = 1}^{+\infty} \langle x_k, y \rangle$.
            
            \item $\sum x_k$~--- ортогональный ряд. Тогда $\sum x_k$~--- сходится $\Longleftrightarrow \sum \| x_k \|^2 < +\infty$ и при этом $\| \sum x_k \|^2 = \sum \| x_k \|^2$.
            
        \end{enumerate}
        
        \subsection{Доказательство}
        
            \begin{enumerate}
            
                \item $| \langle x_n, y_n \rangle - \langle x_0, y_0 \rangle | = | \langle  x_n, y_n \rangle - \langle x_n, y_0 \rangle + \langle x_n, y_0 \rangle - \langle x_0, y_0 \rangle | \leqslant | \langle x_n, y_n - y_0 \rangle | + | \langle x_n - x_0, y_0 \rangle | \leqslant \| x_0 \| \| y_n - y_0 \| + \| x_n - x_0 \| \| y_0 \| \rightarrow 0$ при $n \rightarrow +\infty$.
                
                \item $S_N = \sum\limits^N_{k = 1} x_k$, тогда $\langle \sum\limits_{k = 1}^N x_k, y \rangle = \sum\limits_{k = 1}^N \langle x_n, y \rangle$. При устремлении к бесконечности получаем необходимое равенство.
                
                \item $S_N = \sum\limits_{k = 1}^N x_k$, $\| S_N \|^2 = \langle \sum\limits_{k = 1}^N x_k, \sum\limits_{k = 1}^N x_k \rangle = \sum \langle x_k, x_l \rangle = \sum\limits_{k = 1}^n \langle x_k, x_k \rangle = \sum\limits_{k = 1}^N \| x_k \|^2 = \sum_N$.
                
                Аналогично $\| S_N - S_M \|^2 = \left| \sum_N - \sum_M \right|$
                
                $S_n$ и $\sum_N$~--- фундаментальны одновременно.
                
            \end{enumerate}
    
    \newpage
        
    \section{Ортогональная система (семейство) векторов}
    
        $\left\{ e_k \right\}$~--- \textit{ортогональная система (семейство) векторов}, если $e_k \in \mathcal{H}$, что $\forall i, j : i \neq j : e_i \perp e_j$, $e_k \neq 0$. 
        
    \newpage
    
    \section{Ортонормированная система}
    
        Если ортогональная система $\left\{ e_k \right\}$, для которой $\forall k : \| e_k \| = 1$~--- ортонормированная система векторов.
        
        \subsection{Замечание} 
        
            Если $\left\{ e_k \right\}$~--- ортогональная система, то $\left\{ \dfrac{e_k}{\| e_k \|} \right\}$~--- ортонормированная система.
        
    \newpage
        
    \section{Теорема о коэффициентах разложения по ортогональной системе}
    
        $\left\{ e_k \right\}$~--- ортонормированная система в $\mathcal{H}$, $x \in \mathcal{H}$, $\sum\limits_{k = 1}^{+\infty} c_k e_k = x$. Тогда
        
        \begin{enumerate}
        
            \item ортонормированная система~--- ЛНЗ;
            
            \item $c_k = \dfrac{\langle x, e_k \rangle}{\| e_k \|^2}$;
            
            \item $c_k e_k$~--- ортогональная проекция $x$ на прямую $\left\{ t e_k | t \in \mathbb{R} \right\}$, т.е. $x = c_k e_k + z$, где $z \perp e_k$.
            
        \end{enumerate}
        
        \subsection{Доказательство}
        
            \begin{enumerate}
            
                \item $\sum\limits_{k = 1}^N \alpha_k e_k = 0$.
                
                    Умножим $e_j$ $1 \leqslant j \leqslant N$, $\langle \sum\limits_{k = 1}^N \alpha_k e_k, e_j \rangle = \sum \alpha_k \langle p_k, p_j \rangle \Rightarrow \alpha_j = 0$.
                    
                \item $\langle x, e_m \rangle = \langle \sum\limits_{k = 1}^{+\infty} c_k e_k, e_m \rangle = \sum c_k \langle e_k, e_m \rangle = c_m \langle e_m, e_m \rangle$.
                
                \item $\langle x - c_k e_k, e_k \rangle = \langle x, e_k \rangle - c_k \| e_k \|^2 = 0$.
                
            \end{enumerate}
                
\newpage

\part{Ряды Фурье}

\newpage

    \section{Коэффициенты Фурье}
    
        $\left\{ e_k \right\}$~--- ортогональная система векторов в $\mathcal{H}$, $x \in \mathcal{H}$.
        
        $c_k(x) := \dfrac{\langle x, e_k \rangle}{\| e_k \|^2}$~--- \textit{коэффициенты Фурье} вектора $x$ по системе $\left\{ e_k \right\}$.
        
        $\sum c_k(x) e_k$~--- ряд Фурье в выражениях $x$. При перенормировке $\left\{ e_k \right\}$ ряд Фурье не меняется.
            
    \newpage
    
    \section{Ряд Фурье в Гильбертовом пространстве}
    
        $\sum c_k(x) \cdot e_k$ называется рядом Фурье вектора $x$ по ортогональной системе $\left\{ e_k \right\}$.
        
    \newpage
    
    \section{Теорема о свойствах частичных сумм ряда Фурье. Неравенство Бесселя}
    
        $\left\{ e_k \right\}$~--- ортогональная система $\mathcal{H}$, $x \in \mathcal{H}$, $n \in \mathbb{N}$. $S_n := \sum\limits_{k = 1}^n c_k(x) e_k$, $\mathcal{L} := \Lin(e_1, \ldots, e_n)$.
        
        Тогда верны следующие свойства:
        
        \begin{enumerate}
        
            \item $S_n$~--- проекция $x$ на $S$. $x = S_n + z$, где $z \perp \mathcal{L}$.
            
            \item $S_n$~--- элемент наилучшего приближения для $x$ в $\mathcal{L}$.
            
                $\| x - S_n \| = \min\limits_{y \in \mathcal{L}} \| x - y \|$.
            
            \item $\| S_n \| \leqslant \| x \|$.
            
        \end{enumerate}
        
        \subsection{Доказательство}
        
            $z := x - S_n$, $\langle x, e_k \rangle = \langle x, e_k \rangle - \langle S_n, e_k \rangle = \langle x, e_k \rangle - \langle \sum\limits_{i = 1}^n c_i(x)e_i, a_k \rangle = \langle x_i, e_k \rangle - \sum c_i(x) \langle e_i, e_k \rangle = 0$
            
            $x = S_n + z$, $z \perp \mathcal{L}$.
            
            $y \in \mathcal{L}$, $\| x - y \|^2 = \| S_n - y + z \|^2 = \| S_n - y \|^2 + \| z \|^2 \geqslant \| z \|^2 = \| S_n - x \|^2$
            
            $\| x \|^2 = \| S_n \|^2 + \| z \|^2 \geqslant \| S_n \|^2$.
            
        \subsection{Неравенство Бесселя}
        
            В условиях теоремы выполняется следующее равенство:
            
            $\sum\limits_{k = 1}^{+\infty} \left| C_k(x) \right|^2 \| e_k \|^2 \leqslant \| x \|^2$.
            
            из 3 свойства следует $\| x \|^2 \geqslant \sum\limits_{k = 1}^n \left| c_k(x) \right|^2 \| e_k \|^2$ для любого $n$.
        
    \newpage
    
    \section{Теорема Рисса~--- Фишера о сумме ряда Фурье. Равенство Парсеваля}
    
        $\left\{ e_k \right\}$~--- ортогональная система в $\mathcal{H}$, $x \in \mathcal{H}$. Тогда выполняеются следующие утверждения:
        
        \begin{enumerate}
        
            \item Ряд Фурье $x$ сходится в $\mathcal{H}$.
            
            \item $x = \sum\limits^{+\infty}_{k = 1} c_k(x) e_k + z$, где $\forall k : z \perp e_k$.
            
            \item $x = \sum\limits_{k = 1}^{+\infty} c_k(x) e_k \Longleftrightarrow \sum \left| c_k(x) \right|^2 \| e_k \|^2 = \| x \|^2$ (равенство Парсеваля).
            
        \end{enumerate}
            
        \subsection{Доказательство}
        
            $\sum x_k$~--- ортогональный~--- сх $\Longleftrightarrow \sum \| x_k \|$~--- сходится.
            
            Р.Ф.~--- сходится $\Longleftrightarrow \sum \left| c_k(x) \right|^2 \| e_k \|^2$~--- сходится~--- это всё верно по неравенству Бесселя.
            
            $z : x - \sum c_k e_k$, $\langle z, e_n \rangle = \langle x, e_n \rangle - \sum = \langle x, e_n \rangle - c_n \langle e_n, e_n \rangle$.
            
            $\Rightarrow$~--- очевидно из предыдущей теоремы пункта $3$.
            
            $\Leftarrow \| x \|^2 = \| \sum c_k(x) p_k \|^2 + \| z \|^2 = \sum \left| c_k(x) \right|^2 \| e_k \|^2 + \| z \|^2 \Rightarrow z = 0$
            
    \newpage
    
    \section{Базис, полная, замкнутая ОС}
    
        \begin{enumerate}
        
            \item ортогональная система векторов~--- \textit{базис}, если $\forall x \in \mathcal{H} : x = \sum c_k(x) e_k$.
            
            \item ортогональная система векторов \textit{полная}, если не $\exists z : z \perp \left\{ e_k \right\}$.
            
            \item ортогональная система векторов \textit{замкнутая} если $\forall x \in \mathcal{H}$ выполняется уравнение замкнутости, т.е. $\sum \left| c_k(x) \right|^2 \| e_k \|^2 = \| x \|^2$.
            
        \end{enumerate}
            
    \newpage
    
    \section{Теорема о характеристике базиса}
    
        $\left\{ e_k \right\}$~--- ортогональная система векторов, тогда эквивалентны следующие утверждения:
            
        \begin{enumerate}
        
            \item $\left\{ e_k \right\}$~--- базис.
            
            \item $\forall x, y \in \mathcal{H}$ выполняется обобщающее уравнение замкнутости:
            
                $\langle x, y \rangle = \sum\limits_{k = 1}^{+\infty} c_k(x) \overline{c_k(y)} \| e_k \|^2$.
                
            \item $\left\{ e_k \right\}$~--- замкнутая ортогональная система.
            
            \item $\left\{ e_k \right\}$~--- полная ортогональная система.
            
            \item $\Lin(e_1, e_2, e_3, \ldots)$~--- плотное в пространстве $\mathcal{H}$.
            
        \end{enumerate}
        
        \subsection{Доказательство}
        
            \begin{itemize}
            
                \item $1 \Rightarrow 2$) $x = \sum c_k(x) P_k$, $\dfrac{\langle y, e_k \rangle}{\| e_k \|^2} = c_k(y)$. $\langle x, y \rangle = \sum c_k(x) \langle e_k, y \rangle = \sum c_k(x) \overline{c_k(y)} \| e_k \|^2$
                
                \item $2 \Rightarrow 3$) $y := x$.
                
                \item $3 \Rightarrow 4$) $z \perp e_k : \forall k$, $c_k(z) = 0$.
                
                    Уравнение замкнутой системы: $\| z \|^2 = \sum \left| c_k(z) \right|^2 \| e_k \|^2 = 0$.

                \item $4 \Rightarrow 1$) По теореме Рисса-Фишера $x = \sum\limits_{k = 1}^{+\infty} c_k(x) e_k + z$, $z \perp e_k \forall k$, то по условию $z = 0$, значит это и есть базис.
                
                \item $4 \Rightarrow 5$) $\mathcal{L} = Cl ( \Lin ( e_1, e_2, \ldots )) ?= \mathcal{H}$.
                    
                    Если $\neq$, то $\exists x \in \mathcal{H} \setminus \mathcal{L}$, тогда $x = \sum c_k(x) e_k + z$, $z \perp e_k \forall k \Rightarrow z = 0 \Rightarrow x \in \mathcal{L}$.
                    
                \item $5 \Rightarrow 4$) $y \perp e_k \forall k$, $y \perp \mathcal{L} = \mathcal{H}$, $y \perp y$, что значит $\langle y, y \rangle = 0$.
                
            \end{itemize}
        
\newpage

\part{Интегралы, зависящие от параметра}

\newpage

    \subsection{Несобственный интеграл в $\mathbb{R}$}
    
        $\int\limits^b_a f(x) dx = \int\limits-{[a, b]} f d \lambda_1$.
        
        $\int\limits^{\rightarrow b}_a f dx = \lim\limits_{B \rightarrow b - 0} \int\limits^B_a f dx$~--- \textit{несобственный интеграл}.
        
        Здесь $f$~--- локально суммируемая, т.е. $\forall B \in [a, b) : f$~--- суммируемая на $[a, B]$. (возможно, что $b = +\infty$.
        
    \subsection{Теорема}
    
        $\int\limits^{\rightarrow b}_a f dx$~--- абсолютно сходится $\Longleftrightarrow f$~--- суммируемая на $[a, b)$.
        
        \subsubsection{Доказательство}
        
            \begin{itemize}
            
                \item $\Leftarrow)$ 
                
                    $f$~--- суммируемая $\Rightarrow \int\limits_{[a, b)} |f| d\lambda$~--- конечный, тогда $\int\limits^{\rightarrow b}_a$ существует.
                
                    $\int\limits^B_a |f| \leqslant \int\limits_{[a, b)} |f|$.
                    
                \item $\Rightarrow)$
                
                    $\lim\limits_{B \rightarrow b - 0} \int\limits^B_a |f| d \lambda = \int\limits_{[a, b)} |f| d \lambda$~--- в силу непрерывности меры снизу $g \geqslant 0$.
                    
                    Измеримость $E \mapsto \int\limits_{E} g dx$.
                
            \end{itemize}
            
    $f : X \times Y \rightarrow \overline{\mathbb{R}}$
    
    $X$~--- пространство с мерой, $y \subset Y_0$~--- метрическое пространство (или даже метризуемое).
    
    Считаем, что $\forall y : f( \cdot, y )$~--- суммируемая на $X$.
    
    \newpage
    
    \section{Предельный переход под знаком интеграла при наличии равномерной сходимостичвп}
    
        $\mu X < +\infty$, $\varphi : X \rightarrow \mathbb{R}$, $f(x, y) \rightrightarrows \varphi$ при $y \rightarrow y_0$ ($y_0 \in Y_0$ или $y_0$~--- предельная точка $Y$). Тогда
        
        $\varphi$~--- суммируемая на $X$, $\lim\limits_{y \rightarrow y_0} \int\limits_X f(x, y) d \mu = \int\limits_X \varphi d \mu$.
        
        \subsection{Доказательство}
        
            По Гейне выбираем $y_n \rightarrow y_0$ при больших $n : \forall x : \left| f(x, y) - \varphi(x) \right| < 1 \Rightarrow \varphi$~--- суммируемая.
        
            $\left| \int\limits_X F(x, y) d \mu - \int\limits_X \varphi d \mu \right| \leqslant \int\limits_X | f - \varphi | d \mu \leqslant \sup\limits_{x \in X} \left| f (x, y_0) - \varphi(x) \right| \mu X \xrightarrow[n \rightarrow +\infty]{} 0$. 
    
    \newpage
    
    \subsection{определение}
    
        $f : X \times Y \rightarrow \overline{\mathbb{R}}$ (как выше).
        
        $y_0 \in Y$, $f$~--- \textit{удовлетворяет условию} $L_{\mathrm{loc}}(y_0)$, если $\exists g : X \rightarrow \overline{\mathbb{R}}$~--- суммируемая, а также существует $U(y_0)$, что для почти всех $x \in X$ и $\forall y \in Y(y_0) : \left| f(x, y) \right| \leqslant g(x)$.
        
    \newpage
    
    \subsection{Теорема Лебега о мажорирующей сходимости}
    
        $f : X \times Y \rightarrow \overline{\mathbb{R}}$, $\varphi : X \rightarrow \overline{\mathbb{R}}$, что $\lim\limits_{y \rightarrow y_0} f(x, y) = \varphi(x)$ при почти всех $x$, $f$~--- удовлетворяет $L_{\mathrm{loc}} (y_0)$. Тогда 
        
        $\varphi$~--- суммируемая, $\lim\limits_{y \rightarrow y_0} \int\limits_X f(x, y) d \mu = \int\limits_X \varphi d \mu$.
    
        \subsubsection{Доказательство}
        
            Из теоремы Лебега по Гейне $y_n \rightarrow y_0$, при почти всех $x$, при $y \in U(y_0)$ верное $\left| f(x, y) \right| \leqslant g(x)$, для больших $n$ получаем $\left| f(x, y_n) \right| \leqslant g(x)$, при $n \rightarrow +\infty$ $\left| \varphi(x) \right| \leqslant g(x) \Rightarrow \varphi$~--- суммируемая.
            
            $\int\limits_X f(x, y_n) d \mu \xrightarrow[n \rightarrow +\infty]{} \int\limits_X \varphi d \mu$.
            
        \subsubsection{Следствие}
        
            $f$~--- при почти всех $x$~--- непрерывно по $y$ в точке $y_0$, $f$~--- удовлетворяет $L_{\mathrm{loc}}(y_0)$. Тогда
            
            $J(y) := \int\limits_X f(x, y) d \mu (x)$~--- непрерывна в $y_0$.
            
            $\varphi \leftarrow -f(x, y_0)$.
            
    \subsection{Правило Лейбница}
    
        $Y \subset \mathbb{R}$~--- промежуток.
        
        $f : X \times Y \rightarrow \overline{\mathbb{R}}$, $\forall y : f (\cdot, y)$~-- суммируемая на $X$. 
        
        Пусть:
        
        \begin{enumerate}
        
            \item для почти всех $x$ и $\forall y \in Y : \exists f'_y(x, y)$.
            
            \item $f'_y$~--- удовлетворяет $L_{\mathrm{loc}}(y_0)$.
            
        \end{enumerate}
        
        $J(y) = \int\limits_X f(x, y) d \mu(x)$. Тогда
        
        $J(y)$~--- дифференцируемая в $y_0$ и $J'(y) = \int\limits_X f'_y(x, y) d \mu (x)$.
        
        \subsubsection{Доказательство}
        
            $F(x, h) := \dfrac{f(x, y_0 + h) - f(x, y_0)}{h} \xrightarrow[h \rightarrow 0]{} f'_y (x, y_0)$.
            
            $\dfrac{J(y_0 + h) - J(y_0)}{h} = \int\limits_X F(x, h) d \mu \xrightarrow[h \rightarrow 0]{} \int\limits_X f'_y(x_0, y_0) d \mu$.
            
            $L_{\mathrm{loc}} (h = 0)$, $\left| F(x, h) \right| = \left| f'_y (x, y_0 \rightarrow \theta h ) \right|$ по теорема Лагранжа, и $\left| f'_y (x, y_0 \rightarrow \partial h ) \right| \leqslant g(x)$ по условию $L_{\mathrm{loc}}(y_0)$ для $f'_y$ из $2$ пункта.
            
\newpage

\part{Тригонометрические ряды Фурье}

    \subsection{Тригонометрический полином порядка $n$}
    
        $T_n(x) = \dfrac{a_0}{2} + \sum\limits_{k = 1}^n a_k \cos {kx} + b_k \sin {kx}$~--- \textit{тригонометрический полином не выше $n$}.
        
        $\dfrac{a_0}{2} + \sum\limits_{k = 1}^{+\infty} a_k \cos {kx} + b_k \sin {kx}$~--- \textit{тригонометрический ряд}.
        
        $\cos{kx} = \dfrac{e^{ikx} + e^{-ikx}}{2}$, $\sin{kx} = \dfrac{e^{ikx} - e^{-ikx}}{2i}$.
        
        $S_n = \sum\limits_{k = -n}^n c_k e^{ikx}$~--- тригонометрический полином в экспоненциальной форме.
        
        $\sum\limits_{k \in \mathbb{Z}} c_k e^{ikx} = \lim S_n(x)$~--- тригонометрический ряд в экспоненциальной форме.
        
        $e^{ikx} = \cos {nx} + i \sin {nx}$.
        
    \newpage
    
    \section{Лемма о вычислении коэффициентов тригонометрического ряда}
    
        Дан тригонометрический ряд (вещественный или комлексный), $S_n$, также известно, что $S_n \rightarrow f$ в $L^1 [-\pi, \pi]$. Тогда
        
        $a_k = \dfrac{1}{\pi} \int\limits^{\pi}_{-\pi} f(t) \cos kt dt$ (работает и при $k = 0$).
        
        $b_k = \dfrac{1}{\pi} \int\limits^{\pi}_{-\pi} f(t) \sin kt dt$.
        
        $c_k = \dfrac{1}{2 \pi} \int\limits^{\pi}_{-\pi} f(t) e^{-ikt} dt$.

        \section{Доказательство}
        
            Докажем только формулу $1$, остальные доказываются аналогично.
            
            Возьмём $n > k$, тогда $\int\limits_{-\pi}^{\pi} S_n(t) \cos {kt} = \int\limits^{\pi}_{-\pi} \left( \dfrac{a_0}{2} + \sum a_l \cos {lt} + b_l \sin {lt} \right) \cdot \cos {kt} dt = \int\limits^{\pi}_{-\pi} a_l \cos^2 {kt} = \pi a_k$.
            
            $\left| \int\limits^{\pi}_{-\pi} S_n(t) \cos {kt} - \int\limits^{\pi}_{-\pi} f(t) \cos {kt} \right| \leqslant \int\limits^{\pi}_{-\pi} \left| S_n(t) - f(t) \right| \left| \cos {kt} \right| dt \leqslant \int\limits^{\pi}_{-\pi} \left| S_n(t) - f(t) \right| dt = \| S_n - f \|_1 \xrightarrow[n \rightarrow +\infty]{} 0$.
      
    \newpage
    
    \subsection{Определение}
    
        $f \in L^1 [-\pi, \pi]$, $a_k(f)$, $b_k(f)$, $c_k(f)$, полученные по формуле из леммы~--- \textit{это назначенные коэффициенты Фурье функции $f$}.
        
        Ряд $\dfrac{a_0(f)}{2} + \sum\limits_{k = 1}^{+\infty} a_k(f) \cos {kx} + b_k(f) \sin {kx}$~--- \textit{ряд Фурье функции $f$}.
        
        Также можно рассматривать $\sum\limits_{k \in \mathbb{Z}} c_k(f) c^{ikx}$~--- тоже ряд Фурье.
        
        \subsubsection{Замечание}
        
            $f \in L_1 = L^1 [-\pi, \pi]$~--- чётна.
            
            $b_k(f) = \dfrac{1}{\pi} \int\limits^{\pi}_{-\pi} f(t) \sin {kt} t dt = 0$.
            
            $a_k(f) = \dfrac{2}{\pi} \int\limits^{\pi}_0 f(t) \cos {kt} t dt$.
            
            если $f$~--- нечётная, то меняем местами $a$ и $b$ ($a_k = 0$, $b_k(f) = \dfrac{2}{\pi} \ldots$).
            
        \subsubsection{Еще шаманство}
        
            Для $f \in L^1 [0, \pi]$ можно считать ряд Фурье по синусам или по косинусам.
            
            $f \backsim \dfrac{a_0}{2} + \sum a_k \cos {kx}$, $f \backsim \sum b_k(f) \sin {kx}$.
    
    \newpage
    
    \section{Теорема Римана-Лебега}
    
        $E \subset \mathbb{R}$, $f \in L^1(E, \lambda_1)$. Тогда
        
        $\int\limits_E f(t) e^{i \lambda t} dt \xrightarrow[\lambda \rightarrow +\infty]{} 0$.
        
        $\int\limits_E f(t) \cos{t} \xrightarrow[\lambda \rightarrow +\infty]{} 0$ (аналогично для $\sin$).
        
        \subsection{Следствие}
        
            $a_k(f)$, $b_k(f)$, $c_k(f) \xrightarrow[k \rightarrow +\infty]{} 0$.
            
        \subsection{Доказательство}
        
            Пусть $f = 0$ вне $E$, $f \in L^1 (\mathbb{R})$.
            
            $\int\limits_{\mathbb{R}} f(t) e^{i \lambda t} dt$ при $t = \tau + \dfrac{\pi}{\lambda}$ равно $\int\limits_{\mathbb{R}} f \left( \tau + \dfrac{\pi}{\lambda} \right) e^{i \lambda \tau + i \pi} d \tau = - \int\limits_{\mathbb{R}} f \left( t + \dfrac{\pi}{\lambda} \right) e^{i \lambda t} dt$.
            
            $2 \int\limits_{\mathbb{R}} f(t) e^{i \lambda t} = \int\limits_{\mathbb{R}} \left( f(t) - f(t + \dfrac{\pi}{\lambda}) \right) e^{i \lambda t} dt$
            
            $\left| \int\limits_{\mathbb{R}} f(t) e^{i \lambda t} \right| \leqslant \dfrac{1}{2} \int\limits_{\mathbb{R}} \left| f(t) - f(t + \dfrac{\pi}{\lambda} \right| \cdot \left| e^{i \lambda t} \right| dt = \dfrac{1}{2} \| f - f_{\pi / \lambda} \| \xrightarrow[h \rightarrow 0]{} 0$.
    
    \newpage
    
    \subsection{Модуль непрерывности}
    
        $w(f, h) = \sum\limits_{x, y \in E, |x - y| < h} | f(x) - f(y) |$~--- \textit{модуль непрерывности}.
        
    Пусть $f$~--- дифференцируема на $[a, b]$, тогда $| w(f, h)| \leqslant \max |f'| h$.
    
    \subsection{Теорема}
    
        \begin{enumerate}
        
            \item $f \in \widetilde{C} [-\pi, \pi]$. Тогда $|a_k(f)|$, $|b_k(f)|$, $2|c_k(f)| \leqslant w(f, \frac{\pi}{k})$.
            
        \end{enumerate}
        
        \subsubsection{Доказательство}
        
            Как в теореме Римана-Лебега делаем рассуждение $[-\pi, \pi]$.
            
            $\int\limits^{\pi}_{-\pi} f(t) \cos {kt} dt = - \int\limits^{\pi}_{-\pi} f(\tau + \dfrac{\pi}{k}) \cos {kt}$ (сделали замену), тогда $\pi w(f, \frac{\pi}{k}$.
            
\newpage

\part{05.05.2020}
    
    \subsection{Равномерно сходящийся интеграл}
    
        $J(y) = \int\limits^{\rightarrow b}_a f(x, y) d \mu (x)$, $f : \langle a, b \rangle \times Y \rightarrow \overline{\mathbb{R}}$, локально суммируемая.
        
        Интеграл $J(y)$ равномерно сходится на $Y \Longleftrightarrow \int\limits^t_a f(x, y) dx \rightrightarrows J(y)$ при $t \rightarrow b - 0$.
        
        $\left| \int\limits^t_a f(x, y) dx - J(y) \right| \xrightarrow[t \rightarrow b - 0]{} 0$.
        
        $\sup\limits_y \left| \int\limits^{\rightarrow b}_a f(x, y) dx \right| \xrightarrow[t \rightarrow b - 0]{} 0$.
    
    \subsection{Что-то похожее на признак Вейерштрасса}
    
        $\left| f(x, y) \right| \leqslant g(x)$ и $\int\limits^b_a g(x)$ конечен, тогда интеграл $\int\limits^{\rightarrow b}_a g(x) dx$~--- равномерно сходится.
        
    \subsection{Ложное воспоминание Констранина Петровича}
    
        $f : T \times Y \rightarrow \mathbb{R}$, $T \subset \widetilde{T}$, $Y \subset \widetilde{Y}$~--- метрические пространства (метризуемые)
        
        $t_0$~--- предельная точка $T$, $y_0$~--- предельная точка $Y$. Пусть
        
        \begin{enumerate}
        
            \item $\forall t \in T : \exists \text{ кон. } L(t) = \lim\limits_{y \rightarrow y_0} f(t, y)$.
            
            \item $\forall y \in Y : \exists \text{ кон. } J(y) = \lim\limits_{t \rightarrow t_0} f(t, y)$.
            
            \item Хотя бы один из пределов~--- равномерный.
            
        \end{enumerate}
    
        Тогда существует конечный $\lim\limits_{t \rightarrow t_0} L(t) = \lim\limits_{y \rightarrow y_0} J(y)$.
        
        $f_n(x)$, $\lim\limits_{x \rightarrow x_0} f_n(x) = a_n$, $f_n(x) \rightrightarrows S(x)$, тогда $\exists \text { кон. } \lim\limits_{x \rightarrow x_0} S(x) = \lim\limits_{n \rightarrow +\infty} a_n$.
        
    \subsection{Теорема}
    
        $f : X \times Y \rightarrow \overline{\mathbb{R}}$, $(X, \mathcal{A}, \mu)$~--- пространство с мерой.
        
        $Y \subset \widetilde{Y}$~--- метрическое пространство (или $Y$~--- м.п., или $\widetilde{Y}$~--- метризуемое)
        
        $Y_0 \in \widetilde{Y}$~--- п. т. $Y$.
        
        \begin{enumerate}
        
            \item при почти всех $x : \exists f_0(x) = \lim\limits_{y \rightarrow y_0} f(x, y)$.
            
            \item $f$~--- локально суммируемая, т.е. суммируемая на каждом $(a, t) : t < b$. $\int\limits^t_a f(x, y) \rightarrow \int\limits^t_a f_0(x)$.
            
            \item $|forall y : \exists J(y) = \int\limits^{\rightarrow b}_a f(x, y) d \mu (x)$~--- равномерно сходится на $Y$.
            
        \end{enumerate}
        
        Тогда $\int\limits^{\rightarrow b}_a f(x, y) d \mu(x) \xrightarrow[y \rightarrow y_0]{} \int\limits^{\rightarrow b}_a f_0(x) d \mu(x)$.
        
        \subsubsection{Доказательство}
        
            Это ложное вспоминание с точностью до обозначений.
            
            $T = (a, b)$, $T_0 = \overline{\mathbb{R}}$, $t_0 = b$.
            
            $f(t, y) = \int\limits^t_a f(x, y) d \mu (x)$, $L(t) = \int\limits^t_a f_0(x) d \mu (x)$.
            
            Переход конечный $\leftrightarrow$ интеграл равномерно сходится.
            
        \subsubsection{Следствие}
        
            $1 \leftrightarrow 1'$ при почти всех $x$ $y \mapsto f(x, y)$, непрерывна в точке $y_0$.
            
            Тогда заключение: $J(y)$ непрерывен в точке $y_0$.
            
    \subsection{Определение}
    
        $E = \langle a, b \rangle$, $M \in \mathbb{R}$, $\alpha \in (0, 1)$.
        
        $\mathrm{Lip}_M(\alpha) = \left\{ f : E \rightarrow \mathbb{R} : \forall x, y \in E : \left| f(x) - f(y) \right| \leqslant M | x - y | \right\}$~--- класс Липшеца.
    
        \subsubsection{Пример}
        
            $f$~--- дифференцируема, $\forall x : \left| f'(x) \right| \leqslant M$, $f \in \mathrm{Lip}_M(1)$.
            
            $\left| f(x) - f(y) \right| = | f'(X) | x - y | \leqslant M | x - y |$.
           
    \subsection{Следствие}
    
        $0 \alpha \leqslant 1$, $f \in \mathrm{Lip}_M (\alpha)$. Тогда при $k \neq 0$
        
        $\left| a_k(f) \right|$, $\left| b_k(f) \right|$, $2 \left| c_k(f) \right| \leqslant \dfrac{M \pi^{\alpha}}{k^{\alpha}}$.
        
    \subsection{Утверждение}
    
        $f \in \widetilde{C}^1 [a, b]$. Тогда
        
        $a_k(f') = k b_k(f)$, $b_k(f') - ka_k(f)$, $c_k(f') = i k c_k(f)$.
        
        $2 \pi C_k(f') = \dfrac{1}{2 \pi} \int\limits^{\pi}_{-\pi} f'(x) e^{-ikx}  = f(x) e^{-ikx} \bigg|^{\pi}_{-\pi} + ik   \int\limits^{\pi}_{-\pi} f(x)e^{-ikx}$
    
    \subsection{Следствие}
    
        \begin{enumerate}
        
            \item $f \in \widetilde{C}^{(r)} [-\pi, \pi]$. Тогда $|a_k(f)|$, $|b_k(f)|$, $|c_k(f)| \leqslant \dfrac{\mathrm{const}}{|k|^2}$.
        
            \item $f \in \widetilde{C}^{(r)}$, $f^{(r)} \in \mathrm{Lip}_M (\alpha)$. Тогда $\ldots \leqslant \dfrac{\mathrm{const}}{|k|^{r + \alpha}}$.
            
        \end{enumerate}
        
        $a_k(f) = \dfrac{1}{k^r} a_k \left( f^{(r)} \right)$
        
    \subsection{Ядро Дирихле}
    
        $D_n(t) = \dfrac{1}{\pi} \left( \dfrac{1}{2} + \sum\limits_{k = 1}^n \cos {kt} \right)$~--- \textit{ядро Дирихле}, $n = 0, 1, \ldots$.
        
    \subsection{Ядро Фейера}
    
        $\Phi_n(t) = \dfrac{1}{n + 1} \sum\limits_{k = 0}^n D_k(t)$.
        
    \subsection{Свойства}
    
        \begin{enumerate}
        
            \item $D_n(t) = dfrac{\sin \left( \left(n + \dfrac{1}{2} \right) t \right)}{2 \pi \sin {t / 2}}$.
            
            \item $\Phi_n(t) = \dfrac{1}{2 \pi (n + 1)} \cdot \dfrac{\sin^2 \left( \dfrac{n + 1}{2} t \right)}{\sin^2 {t / 2}}$.
            
            \item $D_n$, $\Phi_n$~--- чётные, $\Phi_n \geqslant 0$, $\int\limits^{\pi}_{-\pi} D_n = 1$, $\int\limits^{\pi}_{-\pi} \Phi_n = 1$.
            
            \item $f \in L^1 [-\pi, \pi]$, тогда $S_n(f, x) = \int\limits_{-\pi}^{\pi} f(x + t) D_n(t) dt$.
            
        \end{enumerate}
    
        \subsubsection{Доказательство}
        
            $2 \sin \dfrac{\pi}{2} \cos {kt} = \sin \left(k + \dfrac{1}{2} \right) t - \sin \left( K - \dfrac{1}{2} \right) t$.
            
            $2 \sin \dfrac{t}{2} D_n = \dfrac{1}{\pi} \left( \sin \dfrac{\pi}{2} + \sum \sin \left(k + \dfrac{1}{2} \right) t - \sin \left( k - \dfrac{1}{2} \right) t \right)$
        
            $2 \pi (n + 1) \Phi_n = \sum\limits_{k = 0}^n \dfrac{\sin \left(k + \dfrac{1}{2} \right) t}{\sin {t / 2}} = \dfrac{sin^2 \dfrac{n + 1}{2} t }{\sin^2 \dfrac{t}{2}}$.
            
            $2 \sum\limits_{k = 0}^n \sin \dfrac{t}{2} \sin \left( k + \dfrac{1}{2} \right) t = 2 \sum \cos {kt} - \cos \left( k + 1 \right) t = (1 - \cos (n + 1) t ) = 2 \sin^2 \left( \dfrac{n + 1}{2} t \right)$
            
            $A_k(f, x) \dfrac{1}{\pi} \int\limits^{\pi}_{-\pi} f(x + t) \cos {kt} dt$
    
    \subsection{Интеграл Дирихле}
    
        $\int\limits^{\pi}_{-\pi} f(x + t) Dn(t) dt$~--- \textit{интеграл Дирихле}.
    
    \newpage
    
    \section{Принцип локализации Римана}
    
        $f$, $g \in L_1$, $x_0 \in \mathbb{R}$, $\delta > 0$. $f(x) = g(x)$ на $(x_0 - \delta, x_0 + \delta)$. Тогда
        
        ряды Фурье $f$ и $g$ ведут себя одинаково, т.е. $S_n(f, x_0) - S_n(g, x_0) \xrightarrow[n \rightarrow +\infty]{} 0$.
        
        \subsection{Доказательство}
        
            $h := f - g$, $h = 0$ в $(x_0 - \delta, x_0 + \delta)$, $S_n(h, x_0)$, проверим, что $S_n(h, x_0) \rightarrow 0$.
        
            $S_n(h, x_0) = \int\limits^{\pi}_{-\pi} h(x_0 + t) D_n(t)$.
        
            $\dfrac{\sin \left(n + \dfrac{1}{2} \right) t}{\sin \dfrac{t}{2}} = \mathrm{ctg} \dfrac{t}{2} \sin{t} + \cos {nt}$.
        
            $S_n(h, x_0) = \dfrac{1}{\pi} \int\limits^{\pi}_{-\pi} h(x_0 + t) \ctg \dfrac{t}{2} \sin {nt} + h(x_0 + t) \cos {nt} dt = b_n(h_1) + a_n(h_2) \xrightarrow[h \rightarrow +\infty]{} 0$. по теореме Римана-Лебега.
        
            Равенство выполняется в случае $h_1$, $h_2 \in L_1$, $h_2 \in L_1$~--- очевидно.
        
            $\int\limits^{\pi}_{-\pi} |h_2| = \int\limits^{\pi}_{-\pi} | h(x_0 + t) | dt$.
        
            $h(x_0 + t) \mathrm{ctg} \dfrac{t}{2}$ при $|t| < \delta : h_1 = 0$.
        
            $|t| > \delta : |h_1| \leqslant |h(x_0 + t)| \cdot \mathrm{ctg} \dfrac{\delta}{2}$.
        
            \subsection{Замечания}
    
                \begin{enumerate}
        
                    \item В условиях теоремы пусть $[a, b] \subset (x_0 - \delta, x_0 + \delta)$. Тогда
            
                    $S_n(h, x) \rightrightarrows 0$ при $n \rightarrow +\infty$ на $[a, b]$.
                
                    \item $x_0$, $\delta$. Для определения ряда Фурье нужен весь $[-\pi, \pi]$. $A$ для ''поведения'' ряда Фурье существенна лишь окрестность $x_0$.
            
                    \item $f \in L^1 [0, \pi]$, 
            
                    $f \backsim \sum b_k(f) \sin {kx}$.
            
                    $\backsim \sum a_k(f) \cos {kx}$.
            
                \end{enumerate}
        
            Эти различия ведут себя одинаково на $[0, \pi]$.
        
\newpage

\section{До свидания, теория меры}

    $(a, b)$
    
    Сумм. $(a, t)$
    
    $\lim\limits_{t \rightarrow b-0} \int\limits^t_a f(x) d \mu(x)$.
    
    $\int\limits^{\rightarrow b}_a f(x, y) d \mu(x)$~--- равномерно сходится, если $\int\limits^t_a \rightrightarrows \int\limits^{\rightarrow b}_a$, если $\sup\limits_{y \in Y} \left| \int\limits^{\rightarrow b}_t f(x, y) d \mu(x) \right| \xrightarrow[t \rightarrow b - 0]{} 0$.
    
    \subsection{Теорема об интегрировании по параметру}
    
        $f : (a, b) \times Y \rightarrow \overline{\mathbb{R}}$~--- суммируемая по мере $\lambda_1 \times \mu$ на каждом множестве вида $(a, t) \times Y$, где $a < t < b$. $\mu Y < +\infty$. Пусть $J(y) = \int\limits^{\rightarrow b}_a f(x, y) dx$~--- равномерно сходится на $Y$. Тогда
        
        \begin{enumerate}
        
            \item $J(y)$~--- суммируемая на $Y$.
            
            \item $\int\limits^{\rightarrow b}_a \left( \int\limits_Y f(x, y) d \mu(y) \right) dx$~--- сходится.
            
            \item $\int\limits_Y \int\limits^{\rightarrow b}_a f(x, y) d \mu(x) = \int\limits^{\rightarrow b}_a \left( \int\limits_Y f(x, y) dy \right)$.
            
        \end{enumerate}
        
        \subsubsection{Доказательство}
        
            Проверим свойство $1$.
            
            $J_t(y) = \int\limits^t_a f(x, y) dx$, $a < t < b$, $y \in Y$~--- суммируемая на $Y$ по теореме Фубини.
            
            $\left| J(y) - J_t(y) \right| = \int\limits_t = \left| \int\limits^{\rightarrow b}_t f(x, y) dx \right| \leqslant 1$ $\forall y$ при $t$ близких к $b$ (следует из равномерной сходимости), значит $J(y)$~--- суммируемая (поскольку $\mu Y < +\infty$).
            
            Остальные свойства сами собой получаются.
            
            $x \mapsto \int\limits_Y f(x, y) d \mu(y)$~--- суммируемая по $x$ на промежутке $(a, t)$ (по теореме Фубини).
            
            По теореме Фубини $\int\limits^t_a \left( \int\limits_Y f(x, y) d \mu(y) \right) dx = \int\limits_Y \int\limits^t_u f = \int\limits_Y \int \left( \int\limits^{\rightarrow b}_a f dx \right) d \mu(y) - \int\limits_Y \left( \int\limits^{\rightarrow b}_t f dx \right) d \mu(Y)$.
            
            $\left| \int\limits^t_a \left( \int\limits_Y f \right) - \int\limits_Y \left( \int\limits^{\rightarrow b}_a f \right) \right| \leqslant \left| \int\limits_Y \left( \int\limits^{\rightarrow b}_t f dx \right) dy \right| \leqslant \int\limits_Y \left| \int\limits^{\rightarrow b}_t f dx \right| dy \leqslant \mu Y \sup\limits_{y \in Y} \left| \int\limits^{\rightarrow b}_t f(x, y) dx \right| \xrightarrow[t \rightarrow b - 0]{} 0$.
        
    \subsection{Правило Лейбница для несобственный интегралов}
    
        $f : [a, b) \times \langle c, d \rangle \rightarrow \mathbb{R}$~--- непрерывная.
        
        $\forall y : J(y) = \int\limits^{\rightarrow b}_a f(x, y) dx$~--- сходится.
        
        Пусть $\forall x : \forall y : \exists f'_y(x, y)$~--- непрерывная функция, $[a, b) \times \langle c, d \rangle$.
        
        Пусть $I(y) = \int\limits^{\rightarrow b}_a f'_y (x, y) dx$~--- равномерно сходится на $\langle c, d \rangle$. Тогда 
        
        \begin{enumerate}
        
            \item $J(y) \in C^1 \langle c, d \rangle$.
        
            \item $J'(y) = I(y)$, т.е. $\dfrac{d}{dy} \left( \int\limits^{\rightarrow b}_a f(x, y) dx \right) = \int\limits^b_a f'_y(x, y) dx$
            
        \end{enumerate}
        
        \subsubsection{Доказательство}
        
            $I$~--- непрерывно зависит от $y$ (по теореме о непрерывности несобственного интеграла).
            
            $s_0$, $s \in \langle c, d \rangle$, $\int\limits^s_{s_0} I(y) dy = \int\limits^s_{s_0} \left( \int\limits^{\rightarrow b}_a f'_y dx \right) dy$.
            
            $[x, y) \in [a, t] \times [s_0, s]$, $f'_y$
            
            По предыдущей теореме меняем порядок и получаем
            
            $\int\limits^{\rightarrow b}_a \left( \int\limits^s_{s_0} f'_y(x, y) dy \right) dx = \int\limits^{\rightarrow b}_a f(x, s) - f(x, s_0) dx = J(s) - J(s_0) \Rightarrow J(s)$~--- дифференцируема, по теореме Барроу $J'(S) = I(S)$.
            
\newpage

\part{11.05.2020}

    $D_n := \dfrac{1}{\pi} \left( \dfrac{1}{2} + \sum\limits^n_{k = 1} \cos {kx} \right)$
    
    $S_n(f, x) = \int\limits^{\pi}_{-\pi} f(x + t) D_n(t) dt$.
    
    \subsection{Признак Дины}
    
        $f \in L_1$, $x_0 \in \mathbb{R}$, $S \in \mathbb{R} ( \mathbb{C} )$.
        
        $\int\limits^{\pi}_0 \dfrac{\left| f(x_0 + t) - 2 S + f(x_0 - t) \right|}{t} dt < +\infty$. Тогда
        
        ряд Фурье в $x_0$ сходится к $S$, или $S_n(f, x_0) \xrightarrow[n \rightarrow +\infty]{} S$.
        
        \subsubsection{Доказательство}
        
            Обозначим $\varphi(t) = f(x_0 + t) - 2S + f(x_0 - t)$.
            
            Если $D_n$~--- четный, то $\int\limits^{\pi}_{-\pi} D_n = 1$.
            
            $S_n(f, x_0) - S = \int\limits^{\pi}_{-\pi} f(x_0 + t) D_n(t) - \int\limits^{\pi}_{-\pi} S D_n(t) dt = \int\limits^{\pi}_{-\pi} \left( f(x + t) - S \right) D_n(t) dt = \int\limits^0_{-\pi} + \int\limits^{\pi}_0 = \int\limits^{\pi}_0 \varphi(t) D_n(t) dt$, для $t \in s - t$. Тогда как в предположении теоремы получаем
            
            $\int\limits^{\pi}_0 \varphi(t) D_n(t) dt = \dfrac{1}{n} \int\limits_0^{\pi} \dfrac{\varphi(t)}{2} \left( \ctg \dfrac{t}{2} \sin {nt} + \cos {nt} \right) = b_n(h_1) + a_n \left( \dfrac{\varphi(t)}{2} \right)$ (в кавычках).
            
            $h_1(t) = \dfrac{\varphi(t)}{2} \cdot \ctg \dfrac{t}{2}$.
            
            $h_1(t) = $, $\dfrac{\varphi(t)}{2} \ctg \dfrac{t}{2}$ для $t \in (0, \pi)$ или $0$, если $t \in (-\pi, 0)$.
            
            в кавычках $\dfrac{\varphi(t)}{2} = \dfrac{\varphi(t)}{2}$ для $t \in (0, \pi)$ и 0 в противном случае.
            
            Теперь проблема с $h_1 \in L_1$.
            
            $h_1$, $\dfrac{\varphi}{2} \in L_1$ (в кавычках).
            
            в кавычках $\dfrac{\varphi}{2} \in L_1$~--- очевидно.
            
            $\left| \dfrac{\varphi(t)}{2} \ctg \dfrac{t}{2} \right| \leqslant \dfrac{| \varphi(t) |}{2 \cdot t / 2} = \dfrac{| \varphi(t) |}{t}$
            
            $| \tg x | > |x|$ при $x \in (0, \pi / 2)$, а $\ctg x < x$.
        
        \subsection{Замечания}
        
            \begin{enumerate}
            
                \item $* \Leftrightarrow \forall \delta > 0 : \int\limits^{\delta}_0 \dfrac{|\varphi(t)|}{t} dt < +\infty \leqslant \int\limits^{\pi}_{\delta} \dfrac{|f(x_0 + t)| + |f(x_0 - t)| + 2S}{t} \leqslant \dfrac{1}{\delta} \left( \| f \| + \| f \| + 2 S \pi \right)$.
                
                \item $f(x) = \dfrac{1}{\ln |x|}$~--- непрерывна в $0$.
                
                    $x_0 = 0$, $S := 0$, то $\int \dfrac{| f(t) + f(-t) - 2S|}{t} dt$,тогда $-\int\limits^{\pi}_0 \dfrac{2}{t \ln t} dt$~--- расходится.
                
            \end{enumerate}
            
        \subsection{Следствие}
            
            $f \in L_1$, $x_0 \in [-\pi, \pi]$. Пусть существуют $4$ предела: $f(x_0 \pm 0)$, $\alpha_{\pm} := \lim\limits_{t \rightarrow \pm 0} \dfrac{f(x_0 + t) - f(x_0 \pm 0)}{t}$. (односторонняя производная). Тогда
            
            $S_n(f, x_0) \rightarrow \dfrac{1}{2} \left( f(x_0 + 0) + f(x_0 - 0) \right)$.
            
            Берём $S = \dfrac{1}{2} \left( f(x_0 + 0) + f(x_0 - 0) \right)$.
            
            $(*) : \int\limits^{\pi}_0 \dfrac{f(x_0 + t) + f(x_0 - t) - f(x_0 + 0) - f(x_0 - 0)}{t} \xrightarrow[t \rightarrow 0]{} \alpha_+ + \alpha_-$.
            
            т.е. интеграл (*) не является несобственном в нуле.
            
        \subsection{Следствие 2}
        
            $f \in L_1$, $f$~--- непрерывна в $x_0$, а также $\exists f'_{\pm} (x_0)$. Тогда
            
            $S_n(f, x_0) \rightarrow f(x_0)$.
            
            \subsubsection{Доказательство}
            
                $f(x_0 \pm 0) = f(x_0$, $\alpha_{\pm} = f'_{\pm} (x_0)$.
                
        \subsection{Пример}
        
            $f(x) = x$ на $[-\pi, \pi]$, она нечётная, тогда $a_k(f) = 0$.
            
            $b_k(f) = \dfrac{1}{\pi} \int\limits^{\pi}_{-\pi} x \sin {kx} = \dfrac{2}{\pi} \int\limits^{\pi}_0 = x \dfrac{-\cos{kx}}{k} \bigg|^{\pi}_0 + \dfrac{2}{\pi k} \int\limits^{\pi}_0 \cos {kx} dx = \dfrac{2}{k} (-1)^{k - 1}$.
            
            Ряд Фурье: $S(f, x) = \sum \dfrac{2}{k} (-1)^k \sin {kx}$.
            
            $x = \dfrac{\pi}{2}$, $\left( \dfrac{1}{1} - \dfrac{1}{3} + \dfrac{1}{5} - \dfrac{1}{7} + \ldots \right) = \dfrac{\pi}{4}$.
            
            $x = \pi$, $\sum = 0$ (полусумма $\pi$ и $-\pi$ равна $0$ по признаку Дини).
            
            $\sum \| \dfrac{2}{k} (-1)^k \sin {kx} \|^2_2 = \| x \|^2_2$.
            
            $\dfrac{4}{k^2} \int\limits^{\pi}_{-\pi} \sin^2 kx dx$.
            
            $\int\limits^{\pi}_{-\pi} x^2 dx = \dfrac{x^3}{3} \big|^{\pi}_{-\pi} = \dfrac{2 \pi^3}{3}$.
            
            $\sum \dfrac{4 \pi}{k^2} = \dfrac{2 \pi^3}{3}$, $\sum \dfrac{1}{k^2} = \dfrac{\pi^2}{6}$.
            
        \subsection{Конфетка}
        
            Пусть $f \in L_1$, тогда:
            
            \begin{enumerate}
            
                \item Четная.
                
                \item $\int\limits^{\pi}_{-\pi} f = 0$.
                
                \item $\forall q \in \mathbb{Q} : f = 0$ в окрестности точки $\pi q$.
                
                \item $0 < \int\limits^{\pi}_{-\pi} |f|^2 dx < +\infty \Rightarrow f \in L_2, f \neq 0 \Rightarrow$ ряд Фурье $f$ нетривиальный.
                
            \end{enumerate}
            
            $a_k = a_k(f)$. Тогда
            
            $\forall m \in \mathbb{N} : s\sum\limits^{+\infty}_{k = 0} a_{km} = 0$.
            
            \subsubsection{Доказательство}
            
                $\sum a_k \cos {kx} \leftrightarrow f$.
                
                $x_0 := \dfrac{2 \pi}{n} i$; в окрестности $x_0$ $f = 0$ удовлетворяет признаку Дини.
            
                $\sum a_k \cos \dfrac{2 \pi}{n} i k = 0$, $i = 0, 1, 2, \ldots, n - 1$.
                
                Сложим по $i$ : $\sum a_k \left( \sum\limits_{i = 0}^{n - 1} \cos \left( \dfrac{2 \pi}{n} k j \right) \right) = 0$.
                
                $\cos 2 \pi \dfrac{0 k}{n} + \cos 2 \pi \dfrac{k}{n} + \cos 2 \pi \dfrac{2 k}{n} + \cos 2 \pi \dfrac{3k}{n} + \ldots + \cos 2 \pi \dfrac{(n - 1) k}{n}$.
                
                Это сумма $x$ координат и векторов, и она не меняется при повороте на $\dfrac{2 \pi k}{n} \Rightarrow $ сумма векторов равна $0$, значит и сумма $x$ координат равна $0$. (рассуждение содержательно только при $k$ не делящемся на $n$).
                
                При $k$ делящемся на $n$ сумма равна $n$.
                
\newpage

\part{Свёртки и аппроксимативные единицы}
    
    \subsection{Определение}
    
        Свёртка двух функций из $L^1 [-\pi, \pi]$, $f$, $K \in L_1$, 
        
        $(f * K)(x) = \int\limits^{\pi}_{-\pi} f(t) K(x - t) dt = \int\limits^{\pi}_{-\pi}f(x - t) K(t) dt$.
        
    \subsection{Корректность определения}
    
        $g(x, t) = f(x - t) K(t)$. Проверим, что функция~--- измеримая $\mathbb{R}^2 \rightarrow \overline{\mathbb{R}}$.
        
        Давайте рассмотрим функции попроще, т.е. $\varphi(x, t) = f(x - t)$~--- измеримая ли она? $\mathbb{R}^2 \rightarrow \overline{\mathbb{R}}$.
        
        $\mathbb{R}^2 (\varphi < a)$, $E_a := \mathbb{R} \left( f(x) < a \right)$~--- измеримая по Лебегу в $\mathbb{R}$.
        
        $f(x - t) < a$.
        
        $(x, t) \mapsto (x - t, t)$.
        
        $\mathbb{R}^2 ( \varphi < a) \mapsto E_a \times \mathbb{R}$.
        
        $\varphi$~--- измеримая, $K(t) : \mathbb{R}^2 \rightarrow \overline{\mathbb{R}}$~--- измеримо, $(t, x) \mapsto K(t)$.
        
        $\iint\limits_{[-\pi, \pi] \times [-\pi, \pi]} | g(x, t) | dx dt = \int\limits^{\pi}_{-\pi} dt \left( | K(t) | \int\limits^{\pi}_{-\pi} | f(x - t) | dx \right) = \int\limits^{\pi}_{-\pi} f(x - t) K(t) dt$.
        
        Таким образом, свёртка определена при почти всех $x$, и результат свёртки также лежит в $L^1$ (всё это следует из теоремы Фубини).
        
    \subsection{Коэффициент Фурье свёртки}
    
        $c_k(f * K) = 2 \pi c_k(f) c_k(K)$.
        
        $2 \pi c_k(f * k) = \int\limits^{\pi}_{-\pi} \left( \int\limits^{\pi}_{-\pi} f(x - t) K(t) dt \right) e^{-ikx} dx$, $f(x - t) K(t) e^{-ikx}$~--- суммируемая на $[-\pi, \pi] \times [-\pi, \pi]$, тогда $\iint\limits_{[-\pi, \pi] \times [-\pi, \pi]} f(x - t) K(t) e^{-ik(x - t)} e^{-ikt} dx dt = \int\limits^{\pi}_{-\pi} dt \left( \int\limits^{\pi}_{-\pi} K(t) e^{-ikt} \int\limits^{\pi}_{-\pi} f(x - t) e^{-ik(x - t)} dx \right) = (2 \pi)^2 c_k(K) c_k(f)$.
        
        $\widetilde{c_k} (f * K) = \widetilde{c_k} (f) \widetilde{c_k} (K)$.
        
        $L^1[-\pi, \pi] \xrightarrow[]{\widetilde{c}}$.
        
        $f \mapsto ( \ldots, \widetilde{c_{-2}(f)}, \widetilde{c_{-1}(f)}, \widetilde{c_0}(f), \widetilde{c_1(f)}, \ldots )$.
        
        $f * g \mapsto ( \ldots, \widetilde{c_{-1}(f)}, \widetilde{c_{-1}(g)}, \widetilde{c_0(f)} \widetilde{c_0(g)}, \widetilde{c_1}, \ldots )$.
        
    \subsection{Ещё одно свойство}
    
        $f \in L^p [-\pi, \pi]$, $K \in L^q [-\pi, \pi]$. $1 \leqslant p \leqslant +\infty$, $\dfrac{1}{p} + \dfrac{1}{q} = 1$. Тогда
        
        $f * K$ непрерывна, $\| f * K \|_{\infty} \leqslant \| f \|_p \cdot \| K \|_q$ (*).
        
        \subsubsection{Доказательство}
        
            Неравенство (*)~--- неравенство Гёльдера.
            
            $\int\limits^{\pi}_{-\pi} f(x - t) K(t) dt \leqslant \left( \int\limits^{\pi}_{-\pi} |f|^p \right)^{1/p} \left( \int\limits^{\pi}_{-\pi} |K|^q \right)^{1/q}$. 
            
            $p = 1, +\infty$, 
            
            $q = +\infty, 1$.
            
            $\left| (f * K) (x + h) - (f * K)(x) \right| = \left| \int\limits^{\pi}_{-\pi}  \left( f(x + h - t) - f(x - t) \right) \right| \leqslant \| k \|_q \cdot \| f_h - f \|_p \xrightarrow[h \rightarrow 0]{} 0$ (по теореме о непрерывности сдвига, но с оговоркой, что теореме о непрерывности сдвига не работает для случая $p = +\infty$, если $p = +\infty$, то поменяем $p$ и $q$ местами, работает из-за симметричности свёртки). 
            
\newpage

\part{18.05.2020}

    $(f * K)(x) = \int\limits^{\pi}_{-\pi} f(x - t) K(t) dt$.
    
    \subsection{Теорема}
    
        $f \in L^p [-\pi, \pi] (1 \leqslant p \leqslant +\infty), K \in L_1$. Тогда $f * K \in L^p$.
        
        $\| f * K \|_P \leqslant \| K \|_1 \| f \|_p$.
        
        При $p = +\infty$ тоже очевидно.
        
        Докажем при $1 < p < +\infty$. Возьмём $\dfrac{1}{p} + \dfrac{1}{q} = 1$.
        
        $\left| \int\limits^{\pi}_{-\pi} f(x - t) k(t) dt \right|^p \leqslant \left( \int\limits^{\pi}_{-\pi} \left| f(x - t) \right| \left| K(t) \right|^{1/p} \left| K(t) \right|^{1/q} \right)$. и это не превосходит по Гёльдеру
        
        $\left( \int\limits^{\pi}_{-\pi} \left| f(x - t) \right|^p \left| K(t) \right| dt \right) \left( \int\limits^{\pi}_{-\pi} \left| K(t) \right| dt \right)^{p/q} = \| K \|^{p/q}_1$.
        
        $\| f * K \|^p_p = \int\limits^{\pi}_{-\pi} \left| \int\limits^{\pi}_{-\pi} f(x - t) K(t) dt \right|^p dx \leqslant \| K_1 \|^{p/q} \int\limits^{\pi}_{-\pi} \int\limits^{\pi}_{-\pi} \left| f(x - t) \right|^p \left| K(t) \right| dt dx = \| K \|^{p/q + 1}_1 \| f \|^p_p = \| K \|^p_1 \| f \|^p_p$.
        
    \subsection{Определение}
    
        $E_{\delta} := [-\pi, \pi] \setminus (-\delta, \delta)$, $0 \leqslant \delta < \pi$.
               
            $D \in \mathbb{R}$, $h_0 \in \overline{\mathbb{R}}$~--- предельная точка $D$.
            
            Семейство функций $\left\{ K_h \right\}_{h \in D}$~--- аппроксимативная единица, если выполнены следующие аксиомы:
            
            \begin{enumerate}
            
                \item $\forall h \in D : K_h \in L^1 ([-\pi, \pi])$, $\int\limits^{\pi}_{-\pi} K_h = 1$.
                
                \item $\exists M > 0 : \forall h \in D : \| K_h \|_1 \leqslant m$.
                
                \item $\forall \delta in (0, \pi) : \int\limits_{E_{\delta}} | K_h | dt \rightarrow 0$, $h \rightarrow h_0$.
                
            \end{enumerate}
            
            \subsubsection{Замечание}
            
                Если $K_n \geqslant 0$, то из аксиомы $1$ следует аксиома $2$ ($M = 1$).
                
            \subsubsection{Суррогатная аксиома $3$}
            
                $K_h \in L^{\infty} [-\pi, \pi]$ и $\forall \delta \in (0, \pi) : \mathrm{ess}\sup\limits_{x \in E_{\delta}} \left| K_h(t) \right| \xrightarrow[h \rightarrow h_0]{} 0$.
                
                Очевидно, из суррогатной аксиомы $3$ следует обычная аксиома $3$.
            
            \subsubsection{Вывод}
            
                Сочетание аксиом $1$, $2$ и суррогатной $3$~--- усиленная аппроксимативная единица.
                
            \subsubsection{Замечание}
            
                $K_h$~--- аппроксимативная единица, $\left\{ \dfrac{|K_h|}{\|K_h\|_1} \right\}_{h \in D}$~--- тоже аппроксимативная единица (из аксиомы 1 $K_h \Rightarrow \| K_h \|_1 \geqslant 1$).
                
    \subsection{Свойства аппроксимативной единицы}
    
        $K_h$~--- аппроксимативная единица. Тогда
        
        \begin{enumerate}
        
            \item $f \in \overline{C} [-\pi, \pi] \Rightarrow f * K_h \rightrightarrows f$, $h \rightarrow h_0$.
            
            \item $f \in L_1 \Rightarrow f * K_h \xrightarrow[\text{в }L_1]{} f$, т.е. $\| f * K_h - f \|_1 \xrightarrow[h \rightarrow h_0]{} 0$.
            
            \item $K_h$~--- усиленная аппроксимативная единица, $f \in L_1$~--- непрерывно в точке $x$. Тогда $f * K_h$ непрерывна в точке $x$, $(f * K_h)(x) \xrightarrow[h \rightarrow h_0]{} f(x)$.
            
        \end{enumerate}
        
        \subsubsection{Доказательство}
        
            $f * K_h^{(n)} - f(x) = \int\limits^{\pi}_{-\pi} \left( f(x - t) - f(x) \right) K_h(t) dt$ (аксиома $1$).
                
            \begin{itemize}
            
                \item $1$ пункт
                    
                    $f$~--- равномерно-непрерывная: $\forall \varepsilon > 0 : \exists \delta > 0 : \forall t : |t| < \delta : \forall x : \left| f(x - t) - f(x) \right| < \dfrac{\varepsilon}{2 M}$ (аксиома $2$).
                
                    Фиксируем $\varepsilon$:
                    
                        $f * K_h(x) - f(x) = \int\limits^{\delta}_{-\delta} + \int\limits_{E_{\delta}} = I_1 + I_2$.
                        
                        $\left| I_1 \right| \leqslant \int\limits^{\delta}_{-\delta} \left| f(x - t) - f(x) \right| \left| K_h(t) \right| dt \leqslant \dfrac{\varepsilon}{2 m} \cdot \int\limits^{\delta}_{-\delta} | K_h(t) | \leqslant \dfrac{\varepsilon}{2 M} \cdot \| K_h \|_1 \leqslant \dfrac{\varepsilon}{2}$
                        
                        $\left| I_2 \right| \leqslant \int\limits_{E_{\delta}} | \ldots | \leqslant 2 \| f \|_{\infty} \cdot \int\limits_{E_{\delta}} |k_h| dt \rightarrow 0$ (по аксиоме $3$), т.е. $\exists U(h_0) : \forall L \in U(h_0) : \left| I_p \right| < \dfrac{\varepsilon}{2}$.
                        
                        $\left| f * K_h - f \right| \leqslant \left| I_1 \right| + \left| I_2 \right| < \varepsilon$.
                        
                \item $3$ пункт
                
                    $f * K_h$~--- непрерывен по свойству свёртки.
                    
                        $f$~--- равномерно-непрерывная: $\forall \varepsilon > 0 : \exists \delta > 0 : \forall t : |t| < \delta : \left| f(x - t) - f(x) \right| < \dfrac{\varepsilon}{2 M}$ (аксиома $2$).
                        
                        $\left| I_1 \right| \leqslant \int\limits^{\delta}_{-\delta} \left| f(x - t) - f(x) \right| \left| K_h(t) \right| dt \leqslant \dfrac{\varepsilon}{2 m} \cdot \int\limits^{\delta}_{-\delta} | K_h(t) | \leqslant \dfrac{\varepsilon}{2 M} \cdot \| K_h \|_1 \leqslant \dfrac{\varepsilon}{2}$
                        
                        $\left| I_2 \right| \leqslant \int\limits_{E_{\delta}} | \ldots | \leqslant \mathrm{essup} |K_h| \int\limits_{E_{\delta}} |f(x - t)| * |f(x)| dt \leqslant \mathrm{essup} |K_h| \left( \| f \|_1 + 2 \pi | f(x) | \right)$
                        
                        $\left| f * K_h - f \right| \leqslant \left| I_1 \right| + \left| I_2 \right| < \varepsilon$.
                
                \item $2$ пункт
                
                    $\| f * K_h(x) - f(x) \|_1 = \int\limits^{\pi}_{-\pi} \left| \int\limits^{\pi}_{-\pi} \left( f(x - t) - f(x) \right) K_h(t) dt \right| dx \leqslant \int\limits^{\pi}_{-\pi} \int\limits^{\pi}_{-\pi} \left| f(x - t) - f(x) \right| \cdot |K_h| dt dx$.
                    
                    $g(t) := \int\limits^{\pi}_{-\pi} \left| f(x + t) - f(x) \right| dx = \int\limits^{\pi}_{-\pi} | K_h(t) | g(-t) dt = \| K_h \|_1 \cdot \int\limits^{\pi}_{-\pi} g(-t) \dfrac{|K_h(t)|}{\| K_h \|_1} dt$
                    
                    $g(t)$~--- непрерывная по теореме о непрерывности сдвига, $\| K_h \|_1 \leqslant M$ по аксиоме $2$.
                    
                    $\left| \int\limits^{\pi}_{-\pi} | f(x_0 + t) - f(x) | - | f(x_0 + t_0) - f(x) | \right| \leqslant \int\limits^{\pi}_{-\pi} \left| f(x_0 + t) - f(x + t_0) \right| dt$.
                    
            \end{itemize}
            
\end{document}
